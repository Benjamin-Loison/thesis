\section*{Structure}

This thesis is structured as follows. Chapter~\ref{chapter:introduction} gives
an introduction to the problem of Blockchain Interoperability which we treat in
this work and places our work in the broader context of the related scientific
literature. Chapter~\ref{chapter:background} introduces the reader to the
prerequisites required to understand the work, including the cryptographic
primitives we use, the Bitcoin Backbone and Ouroboros models in which we work, and the
technical details of existing cryptocurrency implementations which we build
upon, with a focus on Bitcoin and Ethereum. A reader familiar with
these concepts can skip this chapter. We then devote the next three chapters to
discuss how to disemminate information from open blockchains of both consensus
mechanisms, which form the heart of this thesis. Chapter~\ref{chapter:work} puts forth our Proofs of Proof-of-Work
construction, which allows consuming information from proof-of-work-based
blockchains. In Chapter~\ref{chapter:superlight}, we use them to build
superlight clients, including SPV nodes, full nodes, and superlight miners which
can replace full miners while maintaining only logarithmic consensus state.  Chapter~\ref{chapter:variable} extends our results in the Variable Difficulty and Bounded Delay model. Chapter~\ref{chapter:stake} puts forth our Proofs of
Proof-of-Stake construction, which allows consuming information from
proof-of-stake-based blockchains. Chapter~\ref{chapter:sidechains} discusses
leveraging proofs-of-proofs to achieve interoperability between blockchains by
allowing the free passing of information from one blockchain to another enabling
proof-of-burn one-way pegs, two-way pegs, and smart
contract derivatives. We give our conclusions and directions for future work in
Chapter~\ref{chapter:conclusion}.
