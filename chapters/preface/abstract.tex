\begin{center}%
  {\bfseries Abstract}%
\end{center}%

We put forth the first decentralized communication mechanism
between proof-of-work and proof-of-stake blockchains, or combinations thereof.
To construct it, we propose two new cryptographic primitives
which function as cross-chain certificates. For proof-of-stake sources, the ATMS
primitive (Ad-Hoc Threshold Multisignatures) allows attesting to the shifting of
stake from epoch to epoch. For proof-of-work sources, the NIPoPoWs primitive
(Non-Interactive Proofs of Proof-of-Work) allows compressing proof-of-work into
succinct strings that shrink a long blockchain into a succinct polylogarithmic
proof. We provide the first ATMS and NIPoPoWs constructions. For work, we prove
our constructions are secure in both the static and the variable difficulty
setting and we achieve security in the synchronous and bounded delay settings
with concrete adversary bounds in each case. We put forth the first definition
of sidechain security and formally prove our constructions secure. Our proofs
are in the Backbone model for work and in the Ouroboros model for stake.

Our cross-chain certificates allow the transmission of generic information
between blockchains. We describe multiple applications of our sidechains,
including proof-of-burn-based one-way pegs and two-way pegs.
In addition to interoperability, our protocols enable the transmission of
blockchain information for its own internal use. This allows the construction of
\emph{superlight} clients with exponentially smaller communication complexity
than traditional clients. These superlight clients are the first asymptotic
improvement upon SPV. Additionally, our protocols can be used in the work
setting to create \emph{superlight miners}, which need only logarithmic state
to mine and are an exponential improvement over standard proof-of-work
blockchain protocols. We demonstrate the feasibility of our schemes with
experiments, simulations, and implementations, including measurements of
security and performance metrics. We give concrete proposals for deployment
security parameters. We have worked with the industry to implement our schemes
in practice: Our protocols have been implemented and deployed in the real world
proof-of-work blockchains ERGO, nimiq, WebDollar, and Midnight and the
proof-of-stake blockchain Cardano.\\

\ifuniversity
\noindent
\textbf{Subject area:} Blockchains

\noindent
\textbf{Keywords:} cryptography, blockchains, interoperability, proof-of-work
\fi

\ifuniversity
\cleardoublepage
\begin{center}%
  {\bfseries Περίληψη}%
\end{center}%

Προτείνουμε τον πρώτο αποκεντρωμένο μηχανισμό διαλειτουργικότητας ανάμεσα σε
αλυσίδες βασισμένες στην απόδειξη εργασίας, στην απόδειξη μεριδίου, ή
συνδυασμούς τους. Για την κατασκευή του, εισάγουμε δύο νέα κρυπτογραφικά
primitives που λειτουργούν ως πιστοποιητικά επικοινωνίας αλυσίδων. Για πηγές
απόδειξης μεριδίου, οι Αυτοτελείς Κατωφλιακές
Πολυ-υπογραφές (ATMS) επιτρέπουν στην απόδειξη ότι το μερίδιο άλλαξε από εποχή
σε εποχή. Για πηγές απόδειξης εργασίας, οι Μη-Διαδραστικές Αποδείξεις Απόδειξης
Εργασίας (NIPoPoWs)
επιτρέπουν την συμπίεση της απόδειξης εργασίας σε σύντομα αλφαριθμητικά που
μειώνουν το μέγεθος μίας αλυσίδας σε μία πολυλογαριθμικού μεγέθους απόδειξη.
Δίνουμε τις πρώτες κατασκευές ATMS και NIPoPoWs. Για την απόδειξη εργασίας,
αποδεικνύουμε ότι οι κατασκευές μας είναι ασφαλείς στο μοντέλο στατικής και
δυναμικής δυσκολίας και πετυχαίνουμε ασφάλεια στο σύγχρονο μοντέλο αλλά και
στο μοντέλο φραγμένων καθυστερήσεων με συγκεκριμένα φράγματα αντιπάλου σε κάθε
περίπτωση. Δίνουμε τον πρώτο ορισμό ασφάλειας πλευρικών αλυσίδων (sidechain) και
αποδεικνύουμε αυστηρά ότι οι κατασκευές μας είναι ασφαλείς. Οι αποδείξεις μας
είναι στο μοντέλο του Bitcoin Backbone για την εργασία και στο Ouroboros για τα
μερίδια.

Τα πιστοποιητικά επικοινωνίας αλυσίδων που προτείνουμε επιτρέπουν την μεταφορά
γενικών πληροφοριών ανάμεσα σε αλυσίδες. Περιγράφουμε πολλαπλές εφαρμογές των
πλευρικών αλυσίδων μας, μεταξύ άλλων μονής κατεύθυνσης μεταφορές χρήματος βασισμένων
στο κάψιμο χρήματος καθώς και διπλής κατεύθυνσης μεταφορές χρήματος.
Εκτός από την διαλειτουργικότητα, τα πρωτόκολλά μας επιτρέπουν την μεταφορά
πληφοροριών που αφορούν μία αλυσίδα για ιδία χρήση.
Αυτό επιτρέπει την κατασκευή υπερελαφρών πορτοφολιών με εκθετικά μικρότερη
επικοινωνιακή πολυπλοκότητα από τα παραδοσιακά πορτοφόλια.
Αυτά τα \emph{υπερελαφριά πορτοφόλια} είναι η πρώτη ασυμπτωτική βελτίωση σε σχέση με
το σύστημα Απλής Πιστοποίησης Πληρωμών (SPV).
Επιπλέον, τα πρωτόκολλά μας μπορούν να χρησιμοποιηθούν στο πλαίσιο της απόδειξης
εργασίας για να δημιουργήσουν \emph{υπερελαφρείς εξορύκτες} οι οποίοι
χρειάζονται μόνο λογαριθμικό χώρο για να εξορύξουν και αποτελούν εκθετική
βελτίωση σε σχέση με κλασικά πρωτόκολλα εξόρυξης απόδειξης εργασίας.
Δείχνουμε την πρακτικότητα των σχημάτων μας με πειράματα, προσομοιώσεις, και
υλοποιήσεις, συμπεριλαμβανωμένων μετρήσεων ασφάλειας και απόδοσης.
Δίνουμε συγκεκριμένες τιμές για τις παραμέτρους ασφαλείας που μπορούν να
χρησιμοποιηθούν στην πράξη. Δουλέψαμε με το χώρο της βιομηχανίας για να
υλοποιηθούν τα σχήματά μας στην πράξη: Τα πρωτόκολλά μας έχουν υλοποιηθεί σε
αληθινές συνθήκες και διαχειρίζονται πραγματικά κεφάλαια στις αλυσίδες απόδειξης
εργασίας ERGO, nimiq, WebDollar και Midnight, καθώς και στην αλυσίδα απόδειξης
μεριδίου Cardano.\\

\noindent
\textbf{Θεματική Περιοχή:} Blockchains

\noindent
\textbf{Λέξεις Κλειδιά:} κρυπτογραφία, blockchains, διαλειτουργικότητα, απόδειξη
εργασίας
\fi
