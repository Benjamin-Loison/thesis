\section{Constructing Ad-Hoc Threshold Multisignatures}
\label{sec:auth}

\todo{Clean up this section}

In this section we give several ways to instantiate the ATMS primitive.  We order
them by increasing succinctness but also increasing complexity.

\subsection{Plain ATMS}

%The most direct way to build an ATMS is as follows.
Given a EUF-CMA-secure signature scheme, combining signatures and keys can be
implemented  by plain concatenation. Subsequently, combined verification
requires all signatures to be verified individually. This illustrates that the
ATMS primitive is easy to realize if no concern is given to succinctness.
%
The size of these aggregate signatures and aggregate keys is \emph{quadratic} in
the security parameter $\kappa$: for the aggregate key $2k$ individual keys of size
$\kappa$ bits each are concatenated (with $k=\Theta(\kappa)$), while the aggregate signature consists of
at least $k+1$ individual signatures of size $\kappa$ bits.

\subsection{Multisignature-based ATMS}

The previous construction can be improved by
%works, but requires a linear number of signatures in
%$k$.
employing an appropriate multisignature
scheme.
In the construction below, we consider the multisignature scheme
$\Bsig$
from~\cite{boldyreva2003threshold}.

We make use of a homomorphic property of
this scheme:
any $d$ individual signatures $\sigma_1,\ldots,\sigma_d$ created using
secret keys belonging to (not necessarily unique) public keys $vk_1,\ldots,vk_d$
can be combined into a multisignature $\sigma=\prod_{i=1}^{d}\sigma_i$ that can
then be verified using an aggregated public key $avk=\prod_{i=1}^{d}vk_i$.

  %$$\Pi = (
  %  \GenParam,
  %  \Gen,
  %  \Sig,
  %  \Ver,
  %  \KeyCombine,
  %  \Comprises,
  %  \SigCombine,
  %  \VerCombined
  %)$$

Our multisignature-based $t$-ATMS construction works as follows: the procedures $\GenParam$,
$\Gen$, $\Sig$ and $\Ver$ work exactly as in $\Bsig$.
Given a  set $S$, denote by $\commit{S}$ a Merkle-tree commitment to the set $S$ created
 in some arbitrary, fixed, deterministic way.
%
Procedure $\KeyCombine$, given a sequence
of public keys $\VK=\{vk_i\}_{i=1}^{n}$ returns
$avk=(\prod_{i=1}^{n}vk_i,\commit{\VK})$.
%, where $\commit{\VK}$ is a Merkle commitment to the set $\VK$.
%
Since $\KeyCombine$ is deterministic, $\Comprises(\VK,avk)$ simply recomputes it
to verify $avk$.
%
$\SigCombine$ takes the message $m$, $d$ pairs of signatures with their respective public keys
$\{\sigma_i,vk_i\}_{i=1}^{d}$ and $n-d$ additional public keys
$\{\vkh_i\}_{i=1}^{n-d}$ and produces an aggregate signature
\begin{equation}
  \label{eq:sigma}
\sigma =
\left(
  \prod_{i=1}^{d}\sigma_i,
  \{\vkh_i\}_{i=1}^{n-d},
  \{\pi_{\vkh_i}\}_{i=1}^{n-d}
\right)
\end{equation}
where $\pi_{\vkh_i}$ denotes the (unique) inclusion proof of $\vkh_i$ in the
Merkle commitment
\[
  \commit{\{vk_i\}_{i=1}^{d}\cup\{\vkh_i\}_{i=1}^{n-d}}\,.
\]
%
Finally, the procedure $\VerCombined$ takes
a message $m$,
an aggregate key $avk$,
and an aggregate signature $\sigma$ parsed as in~(\ref{eq:sigma}),
and does the following:
(a) verifies that each of the public keys $\vkh_i$ indeed belongs to a different
leaf in the commitment $\commit{\VK}$ in $avk$ using membership proofs
$\pi_{\vkh_i}$;
(b) computes $avk'$ by dividing the first part of $avk$ by
$\prod_{i=1}^{n-d}\vkh_i$;
(c) returns \emph{true} if and only if $d\geq t$ and the first part of $\sigma$
verifies as a $\Bsig$-signature under $avk'$.

Note that  the scheme $\Bsig$ requires $vk_i$ to be  accompanied by a (non-interactive)
proof-of-possession (POP) \cite{pop} of the respective secret key.  This POP can be
appended to the public key and verified when the key is communicated in the
protocol.  For conciseness, we omit these proofs-of-knowledge from the
description (but we include them in the size calculation below).

\paragraph{Asymptotic Complexity.}
This provides an improvement in our use case over the plain scheme:
In the optimistic case where each of the
$2k$ committee members create their local signatures,
both the aggregate key $avk$ and the aggregate signature $\sigma$ are
\emph{linear} in the security parameter, which is optimal.
If %a subset of %(at least $k+1$) keys sign off, the individual
$r<k$ of the keys do \emph{not} provide their local signatures,  the construction falls back
to being \emph{quadratic} in the worst case if $r = k - 1$.
However, for the
practically relevant case where
$r \ll k$ and almost all slot leaders produced a signature, this construction is
clearly preferable.

\paragraph{Concrete space requirements.}
Concrete signature sizes in this scheme for practical parameters
could be as follows.
We set $k = 2160$ (as is done in the Cardano implementations of~\cite{ouroboros})
and for the signature of~\cite{boldyreva2003threshold} we have
 in bits: $|vk_i| = 272$, $|\sigma_i| = 528$
 (N. Di Prima, V. Hanquez, personal communication, 16 Mar 2018), with
 $|vk_i + POP| = |vk_i| + |\sigma_i| = 800$ bits.
 Assuming
256-bit
%%SHA256
hash function is used for the Merkle tree construction,
%%we have that $|H(\cdot)| = 256$.
%%Therefore,
the size of the data which needs to be included
in $\Mainchain$ in the optimistic case during an epoch transition is
$|avk| + |\sigma| + |\langle\pending\rangle| =
|vk_i + POP| + 2|H(\cdot)| + |\sigma_i| = 800 + 512 + 528 = 1840$ bits per epoch. In
a case where $10\%$ of participants fail to sign, the size will be $|avk| +
|\sigma| = |vk_i + POP| + 2|H(\cdot)| + |\sigma_i| + 0.1  \cdot  2  \cdot  k(|vk_i + POP| + \log(k)|H(\cdot))
= 800 + 512 + 528 + 432  \cdot  (500 + 12  \cdot  256) = 1544944$, or about $190$ KB per
epoch (which is approximately 5 days).

%Public keys $\mathcal{VK}' \subseteq \{vk_i:
%i \in [n]\}$ are aggregated into a key using $\KeyCombine$ which simply
%multiplies them together to produce $avk = \prod_{vk_i \in \mathcal{VK}'} vk_i$.
%The aggregated key also includes the hash of a Merkle Tree root $\langle
%vk_i\rangle$ committing to the whole list of public keys:
%$\KeyCombine(\mathcal{VK}') = avk \concat \langle vk_i\rangle$. Signatures are
%also combined together by multiplying them: letting $\mathcal{S} \subseteq
%\{s_i: i \in [n]\}$, define $\SigCombine(\mathcal{S}) = \prod_{k_i \in
%\mathcal{S}} s_i$. The $\Comprises$ algorithm, since $\KeyCombine$ is deterministic, simply
%recomputes it to verify that the same aggregate key is produced.
%Finally, $\Ver$ and $\VerCombined$ are identical. For details on why this
%construction works, consult~\cite{boldyreva2003threshold}.

%One can tolerate any $r$ number of failures by making use of the Merkle Tree
%included as part of the aggregated keys. Concretely, if $r$ keys fail to sign,
%then the product of the keys that \emph{did} sign can be evaluated by starting
%with $avk$ and by dividing by those keys that did not sign. These can be
%listed individually together with a proof-of-inclusion into the Merkle tree root
%$\langle vk_i\rangle$. This yields to an overhead of $\Theta(r \log(k) \kappa)$
%where $r$ is the number of keys that failed to sign, $\log(k)$ is the depth of
%the Merkle tree, and $\kappa$ is the size of each key.

\subsection{ATMS From Proofs of Knowledge}

While the aggregate signatures construction seems sufficient for practice,
%satisfactory in that it compresses multiple signatures, leading to better constant factors in practice,
it still requires a $\sccert$ transaction that is in the worst case quadratic in
the security parameter.
%and so, asymptotically speaking, it is not an improvement.
The approach below, based on proofs of knowledge, improves on that.

We define $avk \gets \KeyCombine(\mathcal{VK})$ to be the root of a Merkle tree
that has $\mathcal{VK}$ at its leaves. Let $\Sig, \Ver$ come from any secure signature
scheme.
%We will construct a succinct ATMS based on proofs of knowledge on top of
%it.
In our ATMS, the local signature is equal to $s_i = \Sig(sk_i, m)$, where
$sk_i$ is the secret key that corresponds to the $vk_i$ verification key.
Letting $S' = \{s_i\}$ be the signatures generated by a sequence $\mathcal{VK}'$
containing keys in $\mathcal{VK}$, the $\SigCombine(\mathcal{VK}, S', m)$
algorithm reconstructs the Merkle tree from $\mathcal{VK}$ and determines the
membership proof $\pi_i$ for each $vk_i \in \mathcal{VK}'$. Regarding the
non-interactive argument of knowledge, the statement of interest is $(avk, m)$
with witness $\{\pi_i, (s_i, vk_i)\}_{i\in S'}$ such that for all $i$ we have that
$\Ver(vk_i, m, s_i) = 1$ and $\pi_i$ is a valid Merkle tree proof pointing to a
unique leaf for every $i$. $\pi_i$ demonstrates that $vk_i$ is in $avk$. We also
require $|S'| \geq t$. It is possible to construct succinct proofs for this
statement using SNARKs~\cite{snarks} or even without any trusted setup
using e.g., STARKs~\cite{ben2017scalable} or
Bulletproofs~\cite{bulletproofs} in the Random Oracle
model~\cite{ro}. In both cases the actual size of the resulting
signature will be at most logarithmic in $k$, while in the case of STARKs the
verifier will also have time complexity logarithmic in $k$.
