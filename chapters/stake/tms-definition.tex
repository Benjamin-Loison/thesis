\section{Ad-Hoc Threshold Multisignatures}
\label{sec:ats}

We  introduce a new  primitive, \textit{ad-hoc threshold
multisignatures (ATMS)}, which borrow  properties from multisignatures and
threshold signatures and are ad-hoc in the sense that signers need to
be selected on the fly from an existing key set.
In Section~\ref{sec:const} we describe how ATMS
are useful for periodically updating the ``anchor of trust'' that the  mainchain parties
 have w.r.t. the sidechain they are not
following.

ATMS %are used by a set of $n$ parties and
are parametrized by a threshold $t$.
On top of the usual digital signatures functionality, ATMS also provide a way to:
%\begin{itemize}
  %\item[-]
  (1)
  aggregate the public keys of a subset of these parties into a single aggregate public
  key $avk$;
%\item[-]
    (2)
  check that a given $avk$ was created using the right sequence of individual public keys;
  %\item[-]
  and (3)
  aggregate $t'\geq t$ individual signatures from $t'$ of the parties into a single
    aggregate signature that can then be verified using $avk$, which is
    impossible if less than $t$ individual signatures are used.
%\end{itemize}

The definition of an ATMS is given below.

\begin{definition}
  A \textit{$t$-ATMS} is a tuple of algorithms
  $\Pi = (
    \GenParam,\allowbreak
    \Gen,\allowbreak
    \Sig,\allowbreak
    \Ver,\allowbreak
    \KeyCombine,\allowbreak
    \Comprises,\allowbreak
    \SigCombine,\allowbreak
    \VerCombined
  )$
  where:
  \begin{description}
  \item $\GenParam(1^\kappa)$
     is the parameter generation algorithm that takes the security parameter
     $1^\kappa$ and returns system parameters~$\params$.
   \item $\Gen(\params)$
     is the key-generation algorithm that takes $\params$ and produces a
     public/private key pair $(vk_i, sk_i)$ for the party invoking it.
   \item $\Sig(sk_i,m)$
    is the signature algorithm as in an ordinary signature scheme: it takes a private
    key and a message and produces a (so-called \emph{local}) signature $\sigma$.
   \item $\Ver(m, pk_i, \sigma)$
    is the verification algorithm that takes a public key, a message and a
    signature and returns \emph{true} or \emph{false}.
   \item $\KeyCombine(\VK)$
    is the key aggregation algorithm that takes a sequence of public keys $\VK$
    and aggregates them into an \emph{aggregate public key} $avk$.
  \item $\Comprises(\mathcal{VK},avk)$
    is the aggregation-checking algorithm that takes a public key sequence
    $\mathcal{VK}$ and an aggregate public key $avk$ and returns \emph{true} or
    \emph{false}, determining whether $\mathcal{VK}$ were used to produce $avk$.
  \item $\SigCombine\left(m,\VK,\left<(vk_1,\sigma_1),\cdots, (vk_d, \sigma_d)\right>\right)$
    is the signature-aggregation algorithm that takes a message $m$, a sequence of
    public keys $\mathcal{VK}$ and a sequence of $d$ pairs $\left<(vk_1,\sigma_1), \cdots, (vk_d, \sigma_d)\right>$
    where each $\sigma_i$ is a local signature on $m$ verifiable by $vk_i$ and
    each $vk_i$ is in a distinct position within $\mathcal{VK}$, $\SigCombine$
    combines these into a multisignature $\sigma$ that can later be verified with
    respect to the aggregate public key $avk$ produced from $\mathcal{VK}$ (as
    long as $d \geq t$, see below).
  \item $\VerCombined(m,avk,\sigma)$
    is the aggregate-signature verification algorithm that takes
    a message $m$, an aggregate public key $avk$, and a multisignature $\sigma$,
    and returns \emph{true} or \emph{false}.
  \end{description}
\end{definition}

\begin{definition}[ATMS correctness]
  \label{def:ATMS-correctness}
  Let $\Pi$ be a $t$-ATMS scheme
  initialized with $\params \gets \GenParam(1^\kappa)$, let
  $(vk_1, sk_1), \cdots, (vk_n, sk_n)$
  be a sequence of keys generated via $\Gen(\params)$, let $\VK$ be a sequence
  containing (not necessarily unique) keys from the above and $avk$ be generated
  by invoking $avk \gets \KeyCombine(\VK)$. Let $m$ be any message and let
  $\left<(vk_1, \sigma_1), \cdots, (vk_d, \sigma_d)\right>$ be any sequence of key/signature pairs provided
  that $d \geq t$ and every $vk_i$ appears in a unique position in the sequence
  $\VK$, where $\sigma_i$ is generated as $\sigma_i = \Sig(sk_i, m)$.
  Let
  $\sigma
  \gets
  \SigCombine\left(
    m,
    \VK,
    \left<(vk_1, \sigma_1), \cdots, (vk_d, \sigma_d)\right>
  \right)
  %\;
  $.
  The scheme $\Pi$ is
  \emph{correct} if for every such message and sequence the following hold:
  \begin{enumerate}
    \item $\Ver(m, vk_i, \sigma_i)$ is \emph{true} for all $i$;
    \item $\Comprises(\mathcal{VK}, avk)$ is \emph{true};
    \item $\VerCombined(m, avk, \sigma)$ is \emph{true}.
  \end{enumerate}
\end{definition}

We define the security of an ATMS
in the definition below,
via a cryptographic game given in Algorithm~\ref{alg.tms-game}.

\import{./}{./chapters/stake/algorithms/alg.tms-figure.tex}

\begin{definition}[Security]
  \label{def:ATMS-security}
  %A \textit{$t$-threshold multisignature scheme}
  A \textit{$t$-ATMS} scheme
  $\Pi = (
    \GenParam,\allowbreak
    \Gen,\allowbreak
    \Sig,\allowbreak
    \Ver,\allowbreak
    \KeyCombine,\allowbreak
    \Comprises,\allowbreak
    \SigCombine,\allowbreak
    \VerCombined
  )$
  is \emph{secure} if for any PPT adversary $\mathcal{A}$ and any polynomial $p$
  there exists some negligible function $\textsf{negl}$ such that
  $
  %\[
    \Pr[ \textsf{ATMS}_{\Pi, \mathcal{A}}(\kappa, p(\kappa)) = 1]
    < \textsf{negl}(\kappa)
    \; .
  %\]
  $
\end{definition}

The quantity $q$ in the ATMS game counts how many keys the adversary is in
control of among her chosen keys $\keyseq$ which will be used for
aggregate-signature verification. The sequence $\keyseq$ can contain both
adversarially-generated keys as well as some of the keys $\mathcal{VK}$
honestly generated by the challenger. The variable $q$ counts the number of
adversarially controlled keys in $\keyseq$. This includes those keys in
$\keyseq$ for which the adversary has obtained a signature for the message in
question (through the use of the oracle $\mathcal{O}^\mathsf{sig}(\cdot)$) or which the
adversary has corrupted completely (through the use of the oracle
$\mathcal{O}^\mathsf{cor}(\cdot)$), as well as those keys which have been generated by
the adversary herself and therefore are not in $\mathcal{VK}$.

It is straightforward to see that if
$\Pi$
is a secure ATMS, then
the tuple $( \GenParam,\allowbreak \Gen,\allowbreak \Sig,\allowbreak \Ver )$ is a \textsf{EUF-CMA}-secure
signature scheme.

Looking ahead, note that since the $\KeyCombine$ algorithm is only invoked with
the public keys of the participants, it can be invoked by anyone, not just the
parties who hold the respective secret keys, as long as the public portion of
their keys is published. Furthermore, notice that the above games allow the adversary
to generate more public/private key pairs of their own and combine them at will.

Concrete instantiations of the ATMS primitive are
presented in the next sections.
