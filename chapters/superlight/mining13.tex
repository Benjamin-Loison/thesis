\section{Logspace mining against 1/3}\label{sec.mining13}

We propose a blockchain mining scheme in which \emph{superlight} miners maintain
only polylogarithmic consensus state. Contrary to full node miners in existing
blockchain systems, our miners do not need to store the whole blockchain.
Additionally, contrary to SPV miners in existing blockchain systems, our miners
do not even need to store all the block \emph{headers}, which grow linearly in
time. Our miners only need to maintain a state which grows
poly\emph{logarithmically} in time, and, hence, require exponentially less space
than existing miners.

While full nodes are compatible with our proposal, we propose the elimination of
full nodes and full miners altogether. Our scheme achieves the same security as
the existing full mining protocol provided the adversary is limited to $1/3$
mining power. In later sections, we extend our scheme to work in the case of
adversaries with up to $1/2$ mining power.

Our core insight is that a large portion of blocks in the blockchain can be
omitted from storage and communication without harm to security. Additionally,
once these blocks are omitted, they will never again be needed. As such, they
can be garbage collected and pruned from the blockchain, and hence do not need
to be stored by any miner, in memory or in hard drives, and they never need to
be sent on the network. The remaining blocks that need to be kept in the state
form a set of blocks $\Pi$ and are exponentially fewer than the original chain
$\chain$. Of course it will hold that $\Pi \subseteq \chain$. Given that the
pointers described in Section~\ref{sec.interlink} are introduced to every block,
we will ensure that the remaining blocks $\Pi$ that are stored in the state
still form a chain, meaning that they can be placed in chronological order and
every block $\Pi[i]$ will include a pointer to its immediately preceding block
$\Pi[i - 1]$, for $i \gets 1$ to $|\Pi| - 1$. The state $\Pi$ is hence a
\emph{compressed} version of the full chain $\chain$. The state $\Pi$ contains
the genesis block $\mathcal{G}$, the last $k$ blocks of $\chain$, and some
(polylogarithmic in count) intermediate blocks that are necessary for the
security of the protocol.

\noindent
\textbf{Application-layer state.}
Our protocol is concerned with storage space regarding \emph{consensus-layer}
state. However, blockchain protocols typically require some storage of
\emph{application-layer} state as well. We do not attempt to optimize these data
structures, as they differ from application to application. Two examples of the
application-layer state are (1) the current UTXO set in Bitcoin, and
(2) the association between accounts and balances as well as the state of all
smart contracts in Ethereum. Our protocol requires that such application-layer
state is included in every block in the manner already done in Ethereum (note
that Bitcoin does not currently place the UTXO in every block and would thus
need further changes to adopt our proposal). We will design a protocol which
will omit large sections of the blockchain from the storage of miners and from
begin communicated on the network, but it is necessary that application state can be
reconstructed given that a miner holds a block which they deem trustworthy. The
problem of compressing the application-layer state is orthogonal to the
consensus-layer compression which we perform here.

The protocol works as follows. First, if we want to upgrade an existing
protocol, then existing full miners transition to logspace miners by compressing
their chains. For chains that wish to deploy our protocol from genesis, the
chain containing only genesis is already a compressed state, and so no
compression needs to be applied. As the protocol is executed, each miner
maintains in its local state a compressed chain $\Pi$, whose tip $\Pi[-1]$
matches the tip a full node would see, i.e., $\chain[-1]$. The miner then mines
on top of $\Pi[-1]$. If successful, the miner produces a new state $\Pi'$, which
he then broadcasts to the rest of the miners. When a miner receives a new state
$\Pi'$ from the network, he compares it against its local state $\Pi$. If the
received state corresponds to more proof-of-work than the compressed chain kept
in the miner's local state, then the miner adopts the new state $\Pi'$. Note
that this requires the miner to be able to \emph{compare} states to determine
which one corresponds to the most proof-of-work.

The application-layer inclusion of transactions or other application data works
as usual. The miner has a known application-layer state for the tip of his
compressed chain $\Pi[-1]$ and any incoming network transactions are applied on
top of it so that they can be organized into a mempool and placed into newly
mined blocks.

\import{./}{chapters/superlight/algorithms/alg.compress.tex}

\paragraph{Chain compression.}
We first describe the algorithm with which an existing full node can transition
to become a logspace node. This algorithm takes a chain of a full node $\chain$
and returns the compressed chain state $\Pi$ to be adopted by a logspace
miner.

Consider a chain $\chain$ and let $\ell$ be the greatest positive integer such
that $|\chain[{:}-k]\upchain^\ell| \geq 2m$.
Consider the $\ell+1$ superchains
$\mathcal{D}[0],\mathcal{D}[1],\dots,\mathcal{D}[\ell]$, satisfying the following.
\begin{itemize}
	\item
		Set
		$\mathcal{D}[\ell] = \chain[{:}-k]\upchain^\ell$
	\item
		For each $0\leq\mu<\ell$, set
		\[
		\mathcal{D}[\mu] = \chain[{:}-k]\upchain^\mu[-2m{:}] \cup
		\chain[{:}-k]\upchain^\mu\{\chain[{:}-k]\upchain^{\mu+1}[-m]{:}\}
		\]
\end{itemize}

Here, each of the elements of $\mathcal{D}$ is a superchain of a certain level
$\mu$.
Observe that $|\mathcal{D}[\ell]| \geq 2m$ and $\mathcal{D}[\mu+1][-m:] \subseteq
\mathcal{D}[\mu]$. If $|\chain[{:}-k]| \geq 2m$, then set $\pi = \bigcup_{\mu=0}^\ell
\mathcal{D}[\mu]$ and note that this forms a chain. Otherwise, set $\pi =
\chain[{:}-k]$. We highlight that $\mathcal{G} \in \pi$. Finally, we set $\chi =
\chain[-k{:}]$ and include it \emph{verbatim} in the state, forming the final chain
$\pi\chi$ to be broadcast. The algorithm for this construction is illustrated in
Algorithm~\ref{alg.compress}.

\import{./}{chapters/superlight/algorithms/alg.nipopow-verifier.tex}

\paragraph{State comparison.}
When the miner receives a state $\Pi'$ of another miner, it first verifies that it forms a chain,
i.e. for all $i \gets 1$ to $|\Pi'| - 1$ we have that $\Pi'[i]$ includes a pointer to
$\Pi'[i - 1]$ and each of $\Pi'[i]$ is a valid block header with valid
proof-of-work. It then
checks that $\Pi'[0] = \mathcal{G}$. Next, it parses this into $\pi'\chi'$ by
taking $\pi' = \Pi'[{:}-k]$ and $\chi' = \Pi'[-k{:}]$. Subsequently, it
parses $\pi'$ into $\mathcal{D}'$ by inspecting the level of
each block in $\pi'$. More concretely, it sets
$\ell' = \max\{\mu: |\pi'\upchain^\mu| \geq 2m\}$ and then
sets $\mathcal{D}'[\ell'] = \pi'\upchain^{\ell'}$, then for each $\mu \gets \ell' - 1$ down
to $0$ it sets
$\mathcal{D}'[\mu] = \pi'\upchain^\mu[-2m{:}]
          \cup \pi'\upchain^\mu\{\pi'\upchain^{\mu+1}[-m]{:}\}$.

If such parsing is impossible or the blocks cannot be
linked or the proof
does not begin with the Genesis block $\mathcal{G}$, the
receiving miner rejects the proof.
Given the local state $\Pi$ parsed into $(\mathcal{D}[0],\dots,\mathcal{D}[\ell])$ and a state
$\Pi'$ parsed into $(\mathcal{D}'[0],\dots,\mathcal{D}'[\ell'])$ incoming from the network, the receiving miner finds the
least $\mu$ such that $\mathcal{D}[\mu]\cap\mathcal{D}'[\mu]\ne\emptyset$.
\begin{itemize}
	\item
		If such $\mu$ exists, let $b = (\mathcal{D}[\mu] \cap \mathcal{D}'[\mu])[-1]$. Then
		$\Pi'$ becomes adopted if $|\mathcal{D}[\mu]\{b{:}\}| \leq |\mathcal{D}'[\mu]\{b:\}|$.
		Otherwise $\Pi$ remains the local state.
	\item
		If no such $\mu$ exists, $\Pi'$ is adopted if $|\ell| < |\ell'|$. Otherwise,
		$\Pi$ remains the local state.
\end{itemize}

\paragraph{State evolution.}
A logspace miner who has adopted a logarithmic state $\Pi$ mines on top of it
by extending block $\Pi[-1]$ and including the blocks $\Pi\upchain^\mu\!\![-1]$ for
each $\mu \in \mathbb{N}$ in their interlink set. We observe that the interlink set of a
full miner is identical to that of a logspace miner. If the logspace miner discovers
a new block $B$, it evolves its state by computing
$\Pi' = \textsf{Compress}_{m,k}(\Pi B)$ and broadcasts this to the network. The
rest of the miners then compare this against their local state and adopt it if
it is deemed to contain more proof-of-work. We observe that this state evolution
matches the state that a full node would arrive at if they were to compress
their newly discovered chain. More concretely, consider a logspace state $\Pi$
with an underlying chain $\chain$ and a block $B$ mined on top of $\Pi$. Then
it will hold that
$\textsf{Compress}_{m,k}(\chain B) = \textsf{Compress}_{m,k}(\Pi B)$. This
observation allows the new state to be computed \emph{online} from the previous
state without holding the whole chain.

% \dionyziz{Remark in the analysis section that we do not withstand temporary
% dishonest majority}
