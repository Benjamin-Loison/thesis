% TABLE <<<
% \nikos{Do we keep the table?}
% \begin{table}
% \begin{center}
% \begin{tabular}{|p{\columnwidth}|}
% \hline
% $\delta$: advantage of honest parties%, $(t/ (n-t) \leq  1-\delta)$
%
% $f$: probability at least one honest party succeeds in finding a POW in a round
%
% $\eps$: quality of concentration of random variables in typical executions,
% cf. Definition~\ref{def:typical}
%
% $k$: number of blocks for the common prefix property
%
% $\ell$: number of blocks for the chain quality property
%
% $\mu$: chain quality parameter
%
% $s$:  number of rounds for the chain growth property
%
% % $\tau$: chain growth parameter
% %
% $L$: the total run-time of the system \\
%
% \hline
% \end{tabular}
% \end{center}
% \caption{\label{tab:requirements}
% The parameters in our analysis. Positive integers $n,t,L,s,\ell,T,k,\kappa$ where $\log T$ is linearly related to $\kappa$; positive reals
% $f,\eps,\delta,\mu,\lambda$, where $f,\eps,\delta,\mu\in(0,1)$.}
% \end{table}%>>>
%
% \subsection{The Backbone Model}
% For our analysis,
% we work in the Backbone model~\cite{backbone} and adopt an
% environment where the network is synchronous and the protocol is executed in
% distinct \emph{rounds}. We give a short overview of the model. Let $\kappa$
% denote the security parameter, and $n$ denote the total number of parties,
% $t$ of which are adversarial.
% Block generation takes
% place, by honest and adversarial parties alike, by invoking a shared Random
% Oracle $H(\cdot)$ whose range is $\{0,1\}^\kappa$. We use $q$ to denote the
% number of random oracle queries available to each party per round. The honest parties search
% the space by performing $q$ queries for $\mathsf{ctr} \gets 1$ to $q$. The
% adversary controlling $t$ parties gets $qt$ total queries per round.
%
% We later relax this
% assumption in Section~\ref{sec.variable} where we extend our protocol to work
% with mining target adjustments in accordance with~\cite{varbackbone}.

% We also assume that honest parties are
% sufficiently greater in number than the adversarial parties. In particular, we
% require:
%
% \begin{quote}
% 	\noindent{\bf Honest Majority Assumption.~~}
% 	A number of $t$ out of $n$ parties are corrupted such that
% 	$t\leq(\frac13-\delta)n$, where $3f+3\eps<\delta\leq\frac13$.
% \end{quote}
%
% It has been proven~\cite{backbone,varbackbone} that executions
% follow the properties of \emph{Chain Growth}, \emph{Common Prefix}, and
% \emph{Chain Quality}.
%
%>>>

% \noindent
% \textbf{The variable difficulty setting.} The above model is extended to the
% variable difficulty setting as follows~\cite{varbackbone}. The number of
% parties $n$ and the number of adversarial parties $t$
% is chosen by the environment and is allowed to change throughout
% execution, making $\{n\}_{r\in\mathbb{N}}$ and $\{t\}_{r\in\mathbb{N}}$ two
% sequences of numbers which can change in every round. As long as the
% number of parties does not diverge dramatically within a short period of time
% (in the wording of~\cite{varbackbone}, these are sequences which are said to be
% \emph{$(\gamma, s)$-respecting}), the same protocol can be used to maintain the
% chain. The key difference is that the chain is partitioned into temporal sections of fixed
% length called \emph{epochs} during which the target remains constant. Within an
% epoch, the analysis follows the same principles as the constant difficulty
% setting. Across epochs, the difficulty is recalculated by setting the target $T$
% used in the PoW equation to be representative of the computing power devoted to
% mining during the previous epoch. More specifically, the system's $T_0$
% parameter, the target of the genesis epoch, is set to correspond to the number
% of parties $n_0$ participating during the genesis block. This sets up a block generation
% rate of $f_0 = n_0 q \frac{T_0}{2^\kappa}$ blocks per round (to see why, recall
% that each of the $n_0$ parties can make $q$ random oracle queries per round and
% each query has a probability of success equal to $\frac{T_0}{2^\kappa}$) which
% the recalculation algorithm attempts to keep constant in future epochs as the
% number of parties changes. Therefore, to recalculate
% the target for epoch $j+1$, the algorithm measures the block generation rate of
% epoch $j$ which is $f_j = \frac{m}{r_{jm} - r_{(j-1)m}}$ (where $m$ denotes the
% number of blocks per epoch and $r_{jm}$ denotes the round during which block
% with height $jm$ was generated) and adjusts the target based on how inaccurate
% the previous target was, setting $T'_{j+1} = \frac{f_0}{f_j} T_j$. This modifies
% the target by the factor $\frac{f_0}{f_j}$ so that block generation rate is
% changed to be equal to $f_0$ in expectation. Finally, the actual epoch target
% $T_{j+1}$ is calculated using the probational target $T'_{j+1}$ by capping it so
% that it cannot change by more than a fixed factor of $\tau$, i.e., $T_{j+1} =
% \min(\max(T'_{j+1}, \tau T_j), \frac{T_j}{\tau})$.
