\subsection{Computational Ledgers}

\todo{Consider merging this into background chapter}

We give some extensions to the validity languages described in
Chapter~\ref{chapter:background}, which will allow us to construct clients that
maintain shorter state than the full historical sequence of transactions.

The blockchain protocol is parametrized by a \emph{blockchain application}
tuple $(\mathcal{L}, \mathcal{S}, S_0, \delta)$ which consists of the following:

\todo{Use a different notation for $\delta$ as this is reserved for the honest
majority gap}

\begin{enumerate}
  \item $\mathcal{L}$: A (potentially infinite) \emph{transaction language} containing all valid
  transactions.
  \item $\mathcal{S}$: A (potentially infinite) \emph{set of valid states} that the system can be in.
  \item $S_0 \in \mathcal{S}$: A \emph{genesis state} that the system begins
  with.
  \item $\delta: \mathcal{S} \times \mathcal{L} \longrightarrow \mathcal{S} \cup \{\bot\}$:
  An efficiently computable \emph{transition function} that takes a transaction
  $\tx$, a previous state $S \in \mathcal{S}$ and returns a next state
  $S' \in \mathcal{S}$, or $\bot$ if the transaction $\tx$ cannot be applied on
  top of state $S$.
\end{enumerate}

The transition function $\delta$ can be extended to apply multiple transactions
in a transition function
$\delta^*: \mathcal{S} \times \mathcal{L}^* \longrightarrow \mathcal{S} \cup \{\bot\}$
which is defined recursively: $\delta^*(S, \epsilon) = S$ and
$\delta^*(S, \tx\,\overline{\tx})) = \begin{cases}
  \bot\text{, if } S = \bot\\
  \delta^*(S, \overline{\tx})\text{, otherwise}\end{cases}$.

Each block $B$ in a blockchain contains application data which consists of the
previous state $B.S \in \mathcal{S}$ and a sequence of transactions
$B.\overline{\tx}$ with $\forall \tx \in B.\overline{\tx} \in \mathcal{L}$ that this
block confirms. It must hold that the transitions described by a block are
valid i.e., that $\delta^*(B.S, B.\overline{\tx}) \neq \bot$ and that each next
block's previous state is the previous block's next state, i.e.,
$\forall i > 0: \delta^*(\chain[i - 1].S, \chain[i - 1].\overline{\tx}) = \chain[i].S$.
The genesis block $\mathcal{G}$ contains a commitment
to the genesis state $\mathcal{G}.S = S_0$.

Whenever a full node receives a blockchain $\chain$ they check its validity by
ensuring that the above properties hold. Then, the reported \emph{stable ledger
state} $\LState_{p}[r]$ for each honest party $p$ at round $r$ is the state $S$
that is committed to the tip of the stable portion of the blockchain.

\import{./}{chapters/superlight/algorithms/alg.persistence.tex}

To define computational persistence, we set up the game illustrated in
Algorithm~\ref{alg.validity-game}.
Let
$r_1 \leq r_2 \in \mathbb{N}$ be rounds and $p_1, p_2$ be honest parties.
Consider the ledger states $\LState_{p_1}[r_1]$ and $\LState_{p_2}[r_2]$
adopted by $p_1$ and $p_2$ at rounds $r_1 \leq r_2$.
In this game, an adversary
$(\mathcal{A}_1, \mathcal{A}_2)$ participates in an execution in which he
attempts to cause $p_1$ and $p_2$ to arrive at ledger states
$\LState_{p_1}, \LState_{p_2}$ at rounds $r_1, r_2$ such that the ledger of
$p_2$ could not have possibly been the
extension of the ledger of $p_1$. We capture the success of the adversary
through a simulator who remains unable to produce a transaction sequence
$\overline{tx}$ which conciliates ledgers $\LState_{p_1}[r_1]$ and
$\LState_{p_2}[r_2]$, i.e.,
$\delta^*(\LState_{p_1}[r_1], \overline{tx}) = \LState_{p_2}[r_2]$. We remark
here that the mere existence of $\overline{tx}$ is insufficient; it must be
efficiently computable by a simulator. An example in which a transaction
sequence exists but is not efficiently computable is the case where the
adversary causes the honest parties to transition from a state $S_1$ in which
an honest party holds certain coins secured by a public signature scheme, to
a state $S_2$ in which the coins have been transferred to the adversary's
control. In this case, the simulator will remain unable to produce the signature
needed to relinquish control of the honest party's coins, and the adversary will
be deemed successful.

\begin{definition}[Computational persistence]
  A protocol $\Pi$ has \emph{computational persistence} if there is a negligible
  function $\negl$ such that for all probabilistic
  polynomial time adversaries $(\mathcal{A}_1, \mathcal{A}_2)$ and all environments $\mathcal{Z}$
  there exists a probabilistic
  polynomial time simulator $\mathcal{A}^*$ such that

  \[
  \Pr[\validityattack_{\mathcal{A}_1,\mathcal{A}_2,\mathcal{Z},\mathcal{A}^*,\Pi}(\kappa)]
  \leq \negl
  \,.
  \]
\end{definition}

\import{./}{chapters/superlight/algorithms/alg.liveness.tex}

For the case of \emph{computational liveness}, we introduce the similar game
illustrated in Algorithm~\ref{alg.liveness-game}. In this game, parameterized by
the \emph{liveness parameter} $u$, the environment issues a sequence of
transactions $\overline{tx}$ each of which is given as input to all honest
parties continuously for $u$ rounds prior to some round $r$. The adversary
attempts to cause an honest party $p$ to arrive at a ledger state $\LState_p[r]$
at round $r$ on which one or more of the transactions in $\overline{tx}$ have
not been taken into account. Again, we express the adversary's success with the
inability of a probabilistic polynomial-time simulator to produce a transaction
sequence $\overline{tx}'$ with $\overline{tx}' \subseteq \overline{tx}$ which
transitions the genesis state $S_0$ to the adopted ledger state $\LState_p[r]$
of party $p$.

\begin{definition}[Computational liveness]
  A protocol $\Pi$ has \emph{computational liveness} if there is a negligible
  function $\negl$ such that for all probabilistic
  polynomial time adversaries $(\mathcal{A}_1, \mathcal{A}_2)$ and all environments $\mathcal{Z}$
  there exists a probabilistic
  polynomial time simulator $\mathcal{A}^*$ such that

  \[
  \Pr[\livenessattack_{\mathcal{A}_1,\mathcal{A}_2,\mathcal{Z},\mathcal{A}^*,\Pi}(\kappa)]
  \leq \negl
  \,.
  \]
\end{definition}
