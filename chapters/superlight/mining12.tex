\section{Logspace mining against 1/2}\label{sec.mining12}

Based on the trapdoor oracle implementation of the previous section,
we modify the mining protocol so that the proof-of-work equation becomes
\[H(G(ctr) || mtr || interlink) \leq T\]
where $G$ is a pre-image resistant hash
function. The difference with the plain protocol is that the nonce $ctr$ is
hashed prior to being put into the proof-of-work equation. The property of a
block being a $\mu$-superblock is then defined by the equation
\[H(H(G(ctr) || mtr || interlink) || ctr) \leq \frac{2^\kappa}{2^\mu}\]
Note
that, in order to determine the $Q$-block status of a block, the inner
evaluation of $H$ must have already been completed.

Mining then works as follows. The honest miners try to solve the new
proof-of-work equation. If they succeed, they broadcast their solution as usual.
They get paid their rewards into a coinbase transaction. However, that coinbase
transaction remains locked for now. The nodes receiving a blockchain check that
the new proof-of-work equation has been solved correctly. After $k$ blocks
(where $k$ is the Common Prefix parameter) have elapsed and the adopted
blockchain is $\chain$, the miner who solved the proof-of-work of block
$\chain[-k]$ discloses the value of $ctr$ with a reference to the respective
block. This information is broadcast and included in a transaction which is
confirmed in blocks following $\chain$. This condition ``unfreezes'' the
coinbase payout of the miner.

\subsection{Security}

We show how to strengthen the Common-Prefix Lemma
(Lemma~\ref{lemma:qblockcommonprefix}), taking advantage of the fact
that the adversary decides to suppress a block without knowing if it
satisfies the block property or not.

\begin{lemma}[$Q$-block Common-Prefix Lemma]\label{lemma:qblockcommonprefix12}
	Assume $t<(\frac12-\delta)n$ with $\delta>3\eps+3f$ and a $\mathcal{Q}$-typical
	execution.
	Consider a round at which a chain $\chain$ is adopted by an honest party and
	suppose there exist another chain $\chain'$ such that
	$\chain'\setminus(\chain'\cap\chain)$ has at least $12\lambda\xi_Qf/\eps$ $Q$-blocks.
	Then, with high probability, $\chain$ has more $Q$-blocks than $\chain'$.
\end{lemma}
\begin{proof}
	Define $r^*,S,W,W'$ as in the proof of
	Lemma~\ref{lemma:qblockcommonprefix}. As in that proof,
	the number of $Q$-blocks on $\chain'\setminus\chain^*$ is
	at most
	\[
		(1+\eps)\xi_Qp|W'|+3\lambda\xi_Qf
	.\]
	To upper bound the number of $Q$-blocks the adversary can suppress we
	are going to apply Lemma~\ref{lemma:negbin}.
	Consider every block computed in $S$ that was suppressed by the
	adversary. Each such block has an associated adversarial block (see
	Lemma~\ref{lemma:balanced} and the paragraph following its proof).
	Let $J$ contain each honest query $j$ in $S$ that attempted to create
	a suppressed block.
	Let $F_j=1$ for each $j\in J$, unless the suppressed block was already
	more than $k$ blocks deep at the time its associated block was
	created.
	Let $M_j=1$ if the suppressed block was a valid $Q$-block.
	By Lemma~\ref{lemma:negbin}, the adversary suppressed at most
	$(1+\eps)\xi_Qp|W|$. Thus
	the number of $Q$-blocks on $\chain\setminus\chain^*$ is at least
	\[
		Y_Q(S)-(1+\eps)\xi_Qp|W|
	.\]
	Subtracting from this the upper bound above we obtain
	\begin{multline*}
		Y_Q(S)-(1+\eps)\xi_Qpqt|S|-3\lambda\xi_Qf\\
		>Y_Q(S)-\frac{(1+\eps)(1-\delta)}{1-f}\xi_Qf|S|-3\lambda\xi_Qf\\
		>Y_Q(S)-\frac{(1+\eps)(1-\delta)}{1-f}\xi_Qf|S|-\eps\xi_Qf|S|\\
		>Y_Q(S)-Z_Q(S)>0
	.\end{multline*}
	The first and the last inequality use Lemma~\ref{lemma:typical}. For
	the middle inequality, observe that if in a typical execution
	$|S|\le3\lambda/\eps$, then less than $12\lambda\xi_Qf/\eps$ $Q$-blocks
	have been computed in total.
\end{proof}
