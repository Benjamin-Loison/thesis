\noindent
\textbf{Related work.}
The most relevant prior work is Proofs of Proofs of Work
\cite{popow,nipopows} with which we share the interlink
data structure that facilitates interlinking between blocks
with special properties as well as using special property blocks to
create representations of blockchain. This concept goes back
to earlier work on sidechains \cite{sidechains,pow-sidechains,pos-sidechains}.
and relevant postings on the blockchain forum
\cite{Miller:2012bn}. Nevertheless, none of these works connect
these concepts to the topic of compressing the actual miner state
nor do they offer any  way to update such state which is one of our key
contributions. Moreover, our variable difficulty protocol also can be used to resolve
the open question of how to adapt NiPoPoWs~\cite{nipopows} to variable difficulty. 
The only other work we are aware of that
attempts a state compression of a similar nature as the one we perform
is Coda~\cite{meckler2018coda}  which suggests to use SNARKs and their recursive composition
\cite{DBLP:conf/innovations/BitanskyCCT12,DBLP:conf/stoc/BitanskyCCT13} as a way to
compress to polylogarithmic size  and update a given blockchain.
A significant advantage and distinguishing feature of our approach is however the fact that it does
not rely on a common reference string as SNARK-based protocols
require, and so does not require a trusted setup. Moreover, no additional assumptions are introduced
beyond those necessary for the security of the Bitcoin blockchain
in the random oracle model.
Note that the security of proof-of-work~\cite{C:DwoNao92}
blockchains~\cite{bitcoin} has been studied for full nodes in the
constant~\cite{EC:GarKiaLeo15} and variable-difficulty~\cite{C:GarKiaLeo17}
settings.
Using superblock-based constructions to create superlight wallets has
been studied in~\cite{popow,nipopows} and their practical feasibility has been
analyzed in~\cite{gtklocker,compactsuperblocks}, although their system is not
guaranteed to be succinct against all possible adversaries.
Likewise, other constructions allow for superlight
clients based on randomized sampling~\cite{flyclient}, but similarly
rely on the existence of full nodes that hold
the entire blockchain.
