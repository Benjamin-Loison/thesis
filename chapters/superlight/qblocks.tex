\section{Analysis \& $Q$-suppression attacks}

% $Q$-blockS <<<
We define a $Q$-block as a block satisfying a predicate $Q$ on its hash.
Note that this evaluation does not depend on any particular execution.

\begin{definition}[block property, $Q$-block]
	A \emph{block property} is a predicate $Q$ defined on a hash output
	$h \in \{0, 1\}^\kappa$.
	Given a block property $Q$, a valid block with hash $h$ is called
	a $Q$-block if $Q(h)$ is true.
\end{definition}

The block properties we are interested in will be evaluated in actual
executions as $Q(H(\left<ctr, s, x\right>))$ for particular blocks. As
such, we will be interested in properties which are polynomially
computable given $h$ as the input.

%>>>

% RANDOM VARIABLES <<<
\paragraph{Definitions of random variables.}

We will call a query of a party \emph{successful} if it submits a triple
$(ctr,s,x)$ such that $H(ctr,s,x)\leq T$.
Consider a block property $Q$. Let $\xi_Q=\Pr[Q(h)|h\leq T]$, when $h$ is
uniformly distributed over the range of the hash function.
For each round $i$, query $j\in[q]$, and $k\in[t]$ (the $k$-th party
controlled by the adversary), we define Boolean random
variables $X_Q(i),Y_Q(i)$ and $Z_Q(i,j,k)$ as follows.
If at round $i$ an honest party obtains a $Q$-block, then $X_Q(i)=1$,
otherwise $X_Q(i)=0$.
If at round $i$ exactly one honest party obtains a $Q$-block, then
$Y_Q(i)=1$, otherwise $Y_Q(i)=0$.
Regarding the adversary, if at round $i$, the $j$-th query of the $k$-th
corrupted party obtains a $Q$-block, then $Z_Q(i,j,k)=1$, otherwise
$Z_Q(i,j,k)=0$.
Define also $Z_Q(i)=\sum_{k=1}^t\sum_{j=1}^qZ_Q(i,j,k)$.
For a set of rounds $S$, let $X_Q(S)=\sum_{r\in S}X_Q(r)$ and similarly
define $Y_Q(S)$ and $Z_Q(S)$.
We drop the subscript from all variables $X,Y,Z$, when the $Q$-block is
simply the property of being a valid block.
Further, if $X(i)=1$, we call $i$ a \emph{successful round} and if
$Y(i)=1$, a \emph{uniquely successful round}.
%>>>

% BOUNDS ON EXPECTATIONS <<<
As in \cite{backbone}, the probability $f$ that at least one honest party
computes a solution at given round is an important parameter.
Writing $p=T/2^\kappa$ for the probability of success of a single query, we
have
\[
	(1-f){pq(n-t)}\leq f=\Ex[X(i)]=1-(1-p)^{q(n-t)}\leq pq(n-t)
.\]
The following bounds relate the expectations of the random variables defined
above to $f$, for all $i$ and block properties $Q$.
\begin{gather*}
	\xi_Q f\leq\Ex[X_Q(i)]\le\frac{\xi_Q f}{1-f},\quad
		\xi_Q f(1-f)<\Ex[Y_Q(i)],\\
	\Ex[Z_Q(i)]\le\frac{\xi_Q f}{1-f}\cdot\frac t{n-t}
.\end{gather*}
The derivations of these inequalities follow \cite{backbone-new}.
%>>>

\paragraph{Typical executions.} %<<<
We now define our typical set of executions. This follows \cite{backbone-new}, but
extended to include block properties.
Informally, this set consists of
those executions with polynomially many rounds and with the property that all
the random variables of interest over sufficiently many (at least
$\lambda=\Omega(\kappa)$) consecutive rounds do not deviate too much from
their expectation.
To this end, recall the following terms from \cite{backbone-new}. An
\emph{insertion} occurs when, given a chain $\chain$ with two consecutive
blocks $B$ and $B'$, a block $B^*$ created after $B'$ is such that $B,B^*,B'$
form three consecutive blocks of a valid chain. A \emph{copy} occurs if the
same block exists in two different positions. A \emph{prediction} occurs when
a block extends one which was computed at a later round.

\begin{definition}[Typical execution]\label{def:typical}
	For a real $\eps\in(f,\frac14)$, integer $\lambda$, and
	a collection of polynomially many block properties $\mathcal{Q}$, we say an
	execution is $\mathcal{Q}$-\emph{typical} (or simply typical), if the following
	hold.
	\begin{itemize}
		\item
			For any $Q\in\mathcal{Q}$ and any set $S$ of at least $\lambda/\xi_Q$
			consecutive rounds we have
			\begin{align}
				(1-\eps)\Ex[X_Q(S)]<~&X_Q(S)<(1+\eps)\Ex[X_Q(S)],
				\\(1-\eps)\Ex[Y_Q(S)]<~&Y_Q(S),
				\\&Z_Q(S)<\Ex[Z_Q(S)]+\eps\Ex[Y_Q(S)].
			\end{align}
		%\item
		%	For any $\D,i\in[L]$ and $Q\in\mathcal{Q}$, if
		%	$\sum_{j\ge i}A_\D(j)Z_Q(j)\ge k$, then
		%	\begin{equation}
		%		\sum_{j\ge i}A_\D(j)Z_Q(j)\le(1+\eps)\xi_Q\sum_{j\ge i}A_\D(j)
		%	.\end{equation}
		\item
			No insertions, no copies, and no predictions occurred.
	\end{itemize}
\end{definition}%>>>

\begin{theorem}\label{theorem:alltypical}%<<<
	If $t<(1-\delta)(n-t)$ with $\delta>3\eps+3f$,
	an execution is typical with probability $1-e^{-\Omega(\eps^2f\lambda)}$.
\end{theorem}
\begin{proof}
	The proof uses standard Chernoff bounds, along the lines of \cite{backbone-new}.
	We just note that the variables $X_Q(i)$ (and similarly $Y_Q(i)$ and
	$Z_Q(i,j,k))$ are independent Bernoulli trials for each $Q$ and successful
	with probability $\Theta(\xi_Qf)$. In addition, a union bound is applied
	over all $Q$.
\end{proof}%>>>

\begin{lemma}\label{lemma:typical}%<<<
	Assume $t<(1-\delta)(n-t)$ with $\delta>3\eps+3f$ and a $\mathcal{Q}$-typical
	execution.
	Then, the following hold for any $Q\in\mathcal{Q}$ and any set $S$ of at least
	$\lambda/\xi_Q$ consecutive rounds.
	\begin{itemize}
	\item[(a)]
		$(1-\eps)\xi_Qf|S|<X_Q(S)<(1+\eps)\cdot\frac{\xi_Qf}{1-f}\cdot|S|$.
	\item[(b)]
			$(1-\frac\delta3)\xi_Qf|S|<(1-\eps)\xi_Qf(1-f)|S|<Y_Q(S)$.
	\item[(c)]
		$Z_Q(S)<(\frac t{n-t}\cdot\frac1{1-f}+\eps)\cdot\xi_Qf|S|
			\le(1-\frac{2\delta}3)\xi_Qf|S|$.
	%\item[(d)]
	%	$Z_Q(S)
	%		<(1+\frac\delta2)\cdot\frac t{n-t}\cdot X_Q(S)+\frac{\eps f}{1-f}\cdot|S|
	%		%<(1+\frac\delta2)\cdot\frac t{n-t}\cdot X(S)+\eps f|S|
	%		<(1-\frac\delta3)X(S)$.
	\item[(e)] $Z_Q(S)<Y_Q(S)$.
	\end{itemize}
\end{lemma}
\begin{proof}
	This follows with straightforward calculations from the properties of
	a typical execution, the bounds (given in the introduction of this section)
	on the expectations of the involved random variables, and the assumed bounds
	on $t/n,\delta,\eps$ and $f$. (See also \cite{backbone-new}.)
\end{proof}
%>>>

We now establish an upper bound in the number of
$Q$-blocks an adversary can suppress, regardless of what attack method she
follows.

% PAIRING LEMMA <<<
Uniquely successful rounds have the following important property, which was
implicit in the proof of the Common-Prefix Lemma in \cite{backbone} and was made
explicit in \cite{backbone-new}.
\begin{lemma}[Pairing]\label{lemma:pairing}
	For any $i$ and any pair of distinct blocks $\chain[i]$ and
	$\chain'[i]$, if $\chain[i]$ was computed by an honest party in a uniquely
	successful round, then $\chain'[i]$ was computed by the adversary.
\end{lemma}
\begin{proof}
	Let $r$ be the uniquely successful round that $\chain[i]$ was computed.
	No honest party would extend $\chain'[i-1]$ at a round later than $r$, since
	every honest party would have a chain of length at least $i$.
	Similarly, if an honest party computed $\chain'[i]$ at some round
	earlier than $r$, then no honest party would have extended $\chain[i-1]$ at
	round $r$.
	Finally, $\chain'[i]$ cannot have been computed by an honest party at round
	$r$, since $r$ was a uniquely successful round.
\end{proof}
%>>>

% INTERVAL-COVER LEMMA <<<
\begin{lemma}[Suppression]\label{lemma:balanced}
	If $r$ is a uniquely successful round and the corresponding block
	does not belong to the chain of an honest party at a later round,
	then there is a set of consecutive rounds $S$ such that
	$r\in S$ and $Y(S)\leq Z(S)$.
\end{lemma}
\begin{proof}
	Let $\chain$ be the chain of the honest party that was successful at round
	$r$ and $u$ the depth of the corresponding block.
	Let $r'$ be the first round after $r$ in which an honest party has a chain
	$\chain'$ which does not contain block $\chain[u]$. Let $\chain'[u']$ the
	last block of $\chain'$ at round $r'$.
	Let $\chain[u^*]=\chain'[u^*]$ be the last honest block on the common prefix
	of $\chain$ and $\chain'$, and let $r^*$ be its timestamp.
	We claim that the set $S=\{i:r^*<i<r'\}$ satisfies the requirements of the
	statement.
	Clearly, $r\in S$. Let us verify that $Y(S)\leq Z(S)$.
	Indeed, if $\chain^*[v]$ is any block computed during a uniquely successful
	round $i\in S$, it must hold $u^*<v\leq u'$. The first inequality is because
	the party computing $\chain^*[v]$ knows of $\chain[u^*]$ (it was announced
	at round $r^*$ and received by round $i>r^*$) and would not mine on
	a shorter chain.
	The second inequality holds because $v>u'$ contradicts an honest party
	having a chain of length $u'$ at round $r'>i$ (since $\chain^*[v]$ was
	received by round $r'$).
	The inequality then follows by Lemma~\ref{lemma:pairing}, since it is always
	possible to find a block distinct from $\chain^*[v]$ on $\chain$ or
	$\chain'$ (we may use $\chain'$, unless $\chain^*[v]$ is on $\chain'$, in
	which case---due to the minimality of $r'$---we have $v<u$ and we can use
	$\chain$).
\end{proof}

An observation that follows from the above lemma and its proof is that if
the adversary manages to suppress a $Q$-block from the chain of an honest party
and this $Q$-block was computed in a uniquely successful round,
then we can associate with it an adversarial block.
In particular, if $r$ is a uniquely successful round and the corresponding
block does not belong to the chain of an honest party at a later round, then
there is an \emph{associated} adversarial block of the same height that was
adopted by an honest party.
% >>>

% MAIN LEMMA <<<
We now state and prove our Unsuppressibility Lemma. Informally, the lemma says that if the
number of blocks the adversary obtained in a set of consecutive rounds is $z$
and the number of the uniquely successful blocks the honest parties obtained
in the same set of rounds is $y$, then there exist $y-2z$ blocks that will
always belong to the chain of every honest party.
It follows that if the power of the adversary is bounded below 1/3 of the
total power, then with overwhelming probability there will be a nonzero number of such
blocks.

An important note with respect to the Unsuppressibility Lemma is the following.
Fix all the randomness the random oracle requires for a given execution.
This determines the successful queries of every party and therefore determines
the parameters $y$ and $z$ above. The observation is that even if these
random coins are revealed to the adversary at the beginning of the execution,
one can determine about $y-2z$ blocks which once computed---and no matter the
adversary's strategy---will always belong to the chain of every honest party.

\begin{lemma}[Unsuppressibility]\label{lem:unsuppressibility}
	In a typical execution,
	every set of consecutive rounds $U$ has a subset\/ $S$ of uniquely
	successful rounds, such that
	\begin{itemize}
		\item
			$|S|\ge Y(U)-2Z(U)-2\lambda f(\frac t{n-t}\cdot\frac1{1-f}+\eps)$ and
		\item
			after the last round in $S$ the blocks corresponding to\/ $S$ belong to
			the chain of every honest party.
	\end{itemize}
\end{lemma}
\begin{proof}
	Let $U'$ be the set of consecutive rounds that contains $U$ and also the
	$\lambda$ rounds that come before and after $U$.
	By Lemma~\ref{lemma:balanced}, we may take $S$ to contain all those uniquely
	successful rounds $r\in U$ such that for any set of consecutive rounds
	$S' \subseteq U'$ containing $r$, $Y(S')>Z(S')$. Note that, in a typical execution,
	no such $S'$ may contain elements outside $U'$.
	Letting $y=Y(U)$ and $z=Z(U)$, we
	need to show $y-|S|\le2z+2(1-\frac{2\delta}{3})\lambda f$.

	Let us focus on the uniquely successful rounds not in $S$.
	%; call this set of rounds $\bar S$.
	Consider a collection $\mathcal{T}$ of sets of consecutive rounds with the
	following properties.
	\begin{itemize}
		\item
			For all $T\in\mathcal{T}$, $Y(T)\leq Z(T)$.
		\item
			For each $r\in U\setminus S$, there is a $T\in\mathcal{T}$ that contains $r$.
		\item
			$|\mathcal{T}|$ is minimum among all collections with the above properties.
	\end{itemize}
	We now observe that the minimality condition on $\mathcal{T}$ implies that no round
	$r$ with $Z_r>0$ belongs to more than two sets of $\mathcal{T}$. If that was the
	case, then there would be three sets $T_1,T_2,T_3$ in $\mathcal{T}$ with
	$T_1\cap T_2\cap T_3\neq\emptyset$. But then, we could keep the two sets
	with the leftmost and rightmost endpoints, contradicting the minimality of
	$\mathcal{T}$. Furthermore, no round in $U'\setminus U$ belongs to more than one set
	of $\mathcal{T}$. Thus,
	\[
		y-|S|=\sum_{i\in U\setminus S}Y_i
			\le\sum_{T\in\mathcal{T}}Y(T)
			\le\sum_{T\in\mathcal{T}}Z(T)
			\le2z+Z(U'\setminus U)
			%\le2z+\frac{2\lambda f}{1-f}\Bigl(\frac t{n-t}+\eps\Bigr)
			%\le2z+2(1-\frc{2\delta}3)\lambda f
	.\]
	The third inequality holds because every round in which the adversary
	was successful is counted at most twice inside $U$ and at most once outside
	$U$ (by the discussion above the inequalities). Finally,
	using $|U'\setminus U|\le2\lambda$ and Lemma~\ref{lemma:typical}(c) we
	obtain the stated bound.
\end{proof}

%\nikos{Sketch strategy and argument of optimality.}

%\nikos{Do we want a formal statement about optimality?}

The proof of this lemma seems to be quite generous to the adversary on two
accounts. First, it reveals to the adversary all coin flips in the beginning
of the execution. Second, it gives the adversary two choices for each one of
his blocks, and assumes that he will be able to choose among these as he sees
fit. Nevertheless, we conjecture that the bound $y-2z$ cannot be substantially
increased in the case the property is rare.

%\nikos{Requiring $t<(\frac12-\delta)(n-t)$ and $2f+2\eps\leq\delta$ gives that
%for any set $S$ of consecutive rounds in a typical execution
%$
%	Y(S)-2Z(S)>\delta(1-\eps)(1-f)f|S|>\frac\delta2\cdot f|S|
%.$}
%>>>

% COROLLARIES<<<
%\nikos{The next two don't contain the error term. Need to be fixed if we want
%them.}
%
%\begin{corollary}
%	Let\/ $S$ be the set of rounds of an execution. If\/ $Y(S)>3Z(S)$, then
%	there are more than\/ $\frac13\cdot Y(S)$ honest blocks in the chain of
%	every honest party.
%\end{corollary}
%\begin{proof}
%	By Lemma~\ref{lemma:balanced}, the chain of every honest party has at least
%	$Y(S)-2Z(S)>Y(S)-\frac23\cdot Z(S)\ge\frac13\cdot Y(S)$ honest blocks.
%\end{proof}
%\begin{corollary}
%	Assuming\/ $t<(\frac14-\delta)n$ and a typical execution, any honest chain
%	has more honest $Q$-blocks than the total number of adversarial ones.
%\end{corollary}
%\begin{proof}
%	Let $Y$ denote the total number of uniquely successful rounds in the
%	execution and $Z$ the total number of adversarial blocks computed.
%	In a typical execution, $Y>3Z$. By the corollary, there are more than $Z$
%	honest blocks in the chain of every honest party.
%\end{proof}
%>>>

We can now prove that an adversary with less 1/3 of the total mining power
cannot create a chain with more $Q$-blocks than an honest chain. Such a task
would require on the same time to suppress many $Q$-blocks from the honest
chain and to obtain many of them for the adversarial chain.

\begin{lemma}[$Q$-block Common-Prefix]\label{lemma:qblockcommonprefix}
	Assume $t<(\frac13-\delta)n$ with $\delta>3\eps+3f$ and a $\mathcal{Q}$-typical
	execution.
	Consider a round at which a chain $\chain$ is adopted by an honest party and
	suppose there exist another chain $\chain'$ such that
	$\chain'\setminus(\chain'\cap\chain)$ has at least $22\lambda\xi_Qf$ $Q$-blocks
	Then, with overwhelming probability, $\chain$ has more $Q$-blocks than $\chain'$.
\end{lemma}
\begin{proof}
	Assume an execution in which the assumptions of the lemma hold.
	Let $r^*$ be the round on which the last honest block on
	$\chain^*=\chain\cap\chain'$ was computed (if no such block exists let
	$r^*=0$) and define the set of rounds $S=\{i:r^*<i\leq r\}$.
	%Let also $B$ denote the first block of $\chain'\setminus\chain^*$.
	%Let $r'>r^*$ be the round on which the $k$th block on
	%$\chain'\setminus(\chain'\cap\chain)$ was computed and define the set of
	%rounds $S=\{i:r'\leq i\leq r\}$.
	We will study the execution during the rounds in $S$.
	To that end, let $W'$ denote the set of adversarial queries
	on $\chain'\setminus\chain^*$ at some round at least $\lambda$ greater from
	$r^*$. Denote by $W$ the rest of the adversarial queries in $S$.

	We first observe that no query in $W'$ could have suppressed
	a $Q$-block on $\chain$.
	As in the proof of Lemma~\ref{lemma:balanced}, in such a case there would
	exist a set of consecutive rounds $|S^*|\ge\lambda$ such that $Y(S^*)\le
	Z(S^*)$.
	This contradicts the last item of Lemma~\ref{lemma:typical}.

	From this observation and the Main Lemma, there are at least
	$Y(S)-2Z(W)-2\lambda f(\frac t{n-t}\cdot\frac1{1-f}+\eps)$
	blocks that the adversary cannot
	suppress.
	Each of these is a $Q$-block independently with probability $\xi_Q$.
	Under our assumptions,
	$2(\frac t{n-t}\cdot\frac1{1-f}+\eps)<\frac{1-\delta}{1-\eps}$.
	We conclude that, with overwhelming probability, there are at least
	\[
		(1-\eps)\xi_Q\cdot\bigl[Y(S)-2Z(W)\bigr]-(1-\eps)\lambda\xi_Qf
	\]
	$Q$-blocks on $\chain\setminus\chain^*$.

	On the other hand, the number of $Q$-blocks on $\chain'\setminus\chain^*$ is
	at most the $Q$-blocks from the $W'$ queries plus the $Q$-blocks
	from the initial $\lambda$ rounds.
	The latter can be shown to be at most $3\lambda\xi_Qf$.
	For the former, using Lemma~\ref{lemma:negbin} (with
	$F_j=1$ when $j\in W'$ and $M_j=1$ when it resulted in a $Q$-block) and
	Lemma~\ref{lemma:typical}, in a typical execution, are at most
	$(1+\eps)\xi_QZ(W')$. Thus, there at most
	\[
		(1+\eps)\xi_Qp|W'|+3\lambda\xi_Qf
	.\]
	$Q$-blocks on $\chain'\setminus\chain^*$.
	Since these are at least $22\lambda\xi_Qf$, it can be shown that
	the difference between the last two displayed expressions is at least
	$(1-\eps)\xi_Q\cdot[Y(S)-2Z(S)]$. This is positive in a typical execution
	in which the power of the adversary is bounded below $\frac13-\delta$ the
	total power.
\end{proof}

We are now ready to prove the construction of Section~\ref{sec.mining13} secure
and succinct.

\begin{theorem}
	For every $\frac{1}{3}$-bounded PPT adversary $\mathcal{A}$,
	in a typical execution,
	given two valid proofs $\Pi$ and $\Pi'$, the
	verifier accepts the one that corresponds to the longer chain.
	% \dionyziz{The \emph{longer chain} is not well-defined at this point. What does
	% it mean to \emph{correspond}? The theorem must be made more precise.}
\end{theorem}
\begin{proof}
	Assume $\chain$ belongs to an honest prover and is longer than $\chain'$. We
	need to show that $\Pi$ will be the proof accepted by the verifier.

	Let us consider first the case that a $\mu$ as above exists.
	When $\mu=0$, the verifier determines the longer chain and always
	(correctly) accepts the corresponding proof.
	%
	Let us now focus on the case $0<\mu\leq\ell$.
	Note that, since $\mathcal{D}[\mu-1]\cap\mathcal{D}'[\mu-1]=\emptyset$, not both
	superchains can have less than $m$ blocks after their common block. In such
	a case, the adversary has failed to include all the last $m$ blocks of
	$\mathcal{D}'[\mu]$ in $\mathcal{D}'[\mu-1]$ and his proof is invalid.
	If only the adversary has less than $m$ blocks after the common block, then
	clearly $\Pi$ is accepted. Otherwise, the $Q$-block Common-Prefix Lemma
	implies that (again) $\Pi$ is accepted.

	Next, consider the case that no such $\mu$ exists.
	Clearly, $\ell\ne\ell'$ (otherwise $\mathcal{D}[\ell]\cap\mathcal{D}'[\ell']$ would
	contain the genesis block) and we need to argue that $\ell>\ell'$.
	Assume---towards a contradiction---that $\ell<\ell'$ and consider the
	statement of the $Q$-block Common-Prefix Lemma instantiated with blocks of
	level $\ell+1$ as the $Q$-blocks.
	Together with $\ell<\ell'$, it implies that $\chain'$ has less than $m$
	$Q$-blocks after the common block with $\chain$ (since $\chain$ has less
	$Q$-blocks than $\chain'$ in total, it must also have less on its fork).
	But then, both $\chain$ and $\chain'$ have less than $m$ $Q$-blocks after
	their common block. Assuming $\mathcal{D}[\ell]\cap\mathcal{D}'[\ell]=\emptyset$, the
	validity of the adversary's proof is contradicted.
\end{proof}

\begin{theorem}[Succinctness]
	In a typical execution with $t < (\frac{1}{3} - \delta)n$ with
	$3\epsilon + 3f < \delta < \frac{1}{3}$ and letting $m = \lambda$,
	an honest miner's state is in $O(m^2\log(r))$ at round $r$.
\end{theorem}
\begin{proof}
	As $t < (\frac{1}{3} - \delta)n$ with
	$3\epsilon + 3f < \delta < \frac{1}{3}$, therefore
	$c = \Ex[Y] - 2\Ex[Z] - 2\lambda f(\frac{t}{n-t}\frac{1}{1-f} + \epsilon)$ will be
	a positive constant and for sets of consecutive rounds $U$ with
	$|U| \geq \lambda$,
	we will have
	$Y(U) - 2Z(U) - 2\lambda f(\frac{t}{n-t}\frac{1}{1-f} + \epsilon) > (1 - \epsilon)c|U|$.

	Consider a state $\Pi$ generated by an honest prover and suppose for
	contradiction that $|\Pi| \in \omega(m\log(r))$, where $r$ indicates the
	current round number. From the security of the
	scheme, this state will correspond to some underlying chain $\chain$ such that
	$\Pi$ is the compression of $\chain$. Consider the variables
	$(\mathcal{D}, \chi) = \textsf{Distill}_{m,k}(\chain)$. As $|\chi| = k$ is
	constant, therefore $|\bigcup_{d \in \mathcal{D}} d| \in \omega(m\log(r))$.
  Let $\ell = |\mathcal{D}|$.	It holds that $\ell \in O(\log(|\chain|))$.
  Consequently, $\sum_{d \in \mathcal{D}} |d| \in \omega(m\log(r))$. Therefore
  there must exist some $\mu$ such that $|\mathcal{D}[\mu]| \in \Omega(\lambda^2)$.
	Consider the maximum such $\mu$.

	We distinguish two cases.

	Case 1: $\mu = \ell$. Then consider $\mathcal{D}[\ell]$.
	Let $u_0$ denote the round during which $\mathcal{D}[\ell][0]$ was generated
	and $u_1$ denote the round during which $\mathcal{D}[\ell][-1]$ was generated
	and consider the set $U$ of consecutive rounds from $u_0$ to $u_1$. As
	$\mathcal{D}[\ell]$ forms a chain, we have that
	$|U| \geq |\mathcal{D}[\ell]| > \lambda$. Applying the Main Lemma, we obtain
	that at least $|S| \geq c|U| = c|\mathcal{D}[\ell]| \in \Omega(\lambda)$
	rounds of $U$ must have been uniquely successful and belong to the chain of
	every honest party. Therefore $|\mathcal{D}[\ell]\upchain^{\ell+1}| \geq (1 - \epsilon)\frac{|S|}{2}$. By the definition of $\ell$ this is impossible.

	Case 2: $0 \leq \mu < \ell$. By maximality of $\mu$, we have
	$|\mathcal{D}[\mu + 1]| \in O(\lambda)$, but
	$|\mathcal{D}[\mu]| \in \Omega(\lambda^2)$.
	By the definition of
	$\mathcal{D}[\mu] = \chain[{:}-k]\upchain^\mu[-2m{:}] \cup
	\chain[{:}-k]\upchain^\mu\{\chain[{:}-k]\upchain^{\mu+1}[-m]{:}\}$, clearly
	$|\chain[{:}-k]\upchain^\mu[-2m{:}]| = 2m$ so necessarily
	$\chain[{:}-k]\upchain^\mu\{\mathcal{D}[\mu+1][-m]{:}\} \in \Omega(\lambda^2)$. Therefore there exist blocks $A$ and $B$
	in $\mathcal{D}[\mu+1]$ and $\mathcal{D}[\mu]$ such that
	$|\mathcal{D}[\mu + 1]\{A{:}Z\}| = 1$, but
	$|\mathcal{D}[\mu]\{A{:}Z\} \in \omega(\lambda)$. Similarly to case
	1, consider the rounds $u_0$ and $u_1$ during which blocks $A$ and $Z$ were
	generated respectively and the set of consecutive rounds $U$ from $u_0$ to
	$u_1$ with $|U| \in \omega(\lambda)$. Using the Main Lemma, there
	must exist a set of uniquely successful rounds $|S| \geq c|U|$ whose blocks
	have been adopted by all honest parties and of which at
	least $(1 - \epsilon)\frac{|S|}{2} \geq 0$ will be of level $\mu + 1$.
	Therefore there must exist a block between $A$ and $Z$ in
	$\mathcal{D}[\mu + 1]$.

	Both cases are contradictions.
\end{proof}

The previous theorem allows us to make miners reject incoming state that is too
large (more than polylogarithmic) without processing them fully.

\noindent
\textbf{Remark.} One important difference between our scheme and the existing
blockchain protocols is that full nodes are able to verify the whole state
evolution of the system from Genesis. This allows them to recover in case of
temporary dishonest majority~\cite{dishonest-majority}, while our system is
not able to do so. However, this treatment is beyond the Bitcoin backbone model.

We note here that our analysis critically relies on the honest majority
assumption. The reason why \emph{validity} holds for our verifiers is that, once
they receive a chain $\chain$ which is the longest, they inductively know that
$\chain[-k]$ must contain valid state $S$. Then, since they have all the last
$k$ blocks, they can validate the transition function $\delta$ on the
transactions included in the last $k$ blocks before further mining on top of
them.
