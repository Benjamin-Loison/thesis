\section{Introduction}

The typical blockchain protocol, as exemplified by the Bitcoin~\cite{bitcoin}
blockchain, has parties called \emph{miners} issue blocks by solving a cryptographic
puzzle known as \emph{proof of work} (PoW)~\cite{C:DwoNao92}. This block extends
the longest (or, more precisely, the most \emph{difficult}) chain of blocks that
is known to the party at each time. Blocks are exchanged between parties and
accepted as long as they are fresh and are extending the longest chain a party
is aware of. This protocol logic has been formally analysed in a number of
previous works,
e.g.,~\cite{EC:GarKiaLeo15,C:GarKiaLeo17,DBLP:conf/eurocrypt/PassSS17,DBLP:conf/crypto/BadertscherMTZ17}
and shown to be capable of withstanding a suitably bounded, in terms of
computational power, adversary.


Distributed ledgers are typically intended to operate over a long period
and this creates a set of performance considerations,
both in terms of time and space, that impacts both
continuously operating nodes as well as the onboarding process for new nodes.
At the time of writing the size of the Bitcoin blockchain is
over 200GB\footnote{See \url{https://www.blockchain.com/charts/blocks-size} for
current statistics} imposing a considerable storage and onborading requirement.
The standard way of improving this is the utilisation of SPV (``simple payment
verification'') clients that maintain only the headers of blocks
while suppressing any other information of the ledger that is not relevant to
the client. While this can be a substantial space saver it still requires a
state that grows {\em linearly} with the runtime of the system. The fundamental
open question we tackle in this work is whether it is possible to reduce this
further, and specifically to a state that grows only {\em logaritmically} in the
system's run-time.

{\bf  Our contributions.} We put forth a novel method to compress and update
a PoW blockchain. In particular our method enables the compression of
the state of a full-node into one that merely includes
the current application state (such as the UTXO set) as well as a suitable sequence of logarithmically many
blocks that
is representative of a blockchain. As in the case
of uncompressed blockchain mining, our clients can also have an SPV
variant that enables even further savings in terms of space.
No node needs to maintain a linearly-growing blockchain in our protocol, as
miners can even mine on top of these compressed chains. As mining produces new
blocks, the state of the miners evolves and can be shared across the network.
Our construction is based on the previously known notion of \emph{superblocks}~\cite{popow,nipopows,compactsuperblocks},
blocks that have achieved unusually high difficulty. Such blocks are rare and
are representative of the underlying proof-of-work that was performed in the
surrounding blocks in the full chain. Our construction prunes all blocks that
are not superblocks of sufficient difficulty, while only keeping superblocks in
the state.
In more details we provide the following.

\begin{enumerate}[wide, labelwidth=!, labelindent=0pt, label=(\roman*)]
  \item \textbf{Log-space mining.}
A suite of algorithms for compressing and updating a compressed form
of a blockchain. Our compression method relies on selecting a suitable
subset of blocks from a full length blockchain to act as representative
of its length. The compressed chain still
contains the genesis block $\mathcal{G}$ as well as a constant number ($k$) of
recently generated blocks. However, blocks between these two ends are included
only sparsely. The selection of blocks for the representative subset
depends on a certain set of stochastic block properties
and favors those blocks that satisfy the properties that are more rare.
The regular stochastic occurrence of such blocks during
the mining process ensures their availability for the compression
process. A suitable data structure interlinking such blocks is used
to ensure their temporal relationship. Our \emph{state update} algorithm
shows how it is possible to compare two competing compressed blockchains
and choose the one that represents the blockchain with the most difficulty.
Despite not keeping the whole chain in
their state, our superlight miners can help a new miner safely synchronize to
the current state, even if it wakes up having just the Genesis block in its
state. We note that our protocol requires exponentially less consensus-layer
state even compared to SPV protocols.
We present two variants of our algorithms. The first one
uses public stochastic block properties and is secure
against adversaries commanding less than $1/3$ of the total available
computational power. Our second variant, called blinded mining,
uses specially crafted
{\em private} stochastic block properties and provides security
against adversaries of computational power below the $1/2$ level.
In either case we show that the size of the state of the compressed
blockchain remains logarithmic.

  \item\textbf{A toolset for security analysis for compressed blockchains.}
In order to analyse the behavior of the blockchain protocol
when operating over compressed states we develop a probabilistic
methodology that studies the ability of a PoW adversary
to suppress blocks with a given stochastic property.
This methodology can be of independent interest and for this reason we present
it in a generic way. In particular we define the concept of a $Q$-block,
where $Q$ is a stochastic block property (an example of such property
is whether the parity of the hash of the block is odd)
and we study the ability of an adversary that commands a certain
portion of the total computational power to suppress the frequency of such a
property in the blockchain possessed by an honest party.
For the case of public stochastic block properties we show that
suppression of $Q$-blocks is limited and, importantly,
the subsequence of a blockchain containing $Q$-blocks preserves
the ``common prefix'' property, which is the critical property
for consistency as long as the adversary is bounded below
the $1/3$ level in terms of computational power.
In order to go above that and up to below the $1/2$ level,
we introduce the concept of a \emph{trapdoor random oracle},
which models a hash function with the property that each input
has, beyond a hash output, a special private string that is
randomly distributed. Such value is recoverable given a special
trapdoor value which is unpredictable for each specific input.
Trapdoor random oracles can be instantiated using a standard
random oracle and thus represent no fundamental extension to the
set of assumptions that are typically used in the analysis
of blockchain protocols. We show how they can be exploited
to define a set of stochastic block properties that are
hidden, something that gives rise to our aforementioned
concept of blinded mining, in which the miner places
some strategic high-entropy piece of information into every mined block, which
is only revealed at a later time.
Hiding
the $Q$-block property of a specific block
for a short window of time
enables honest parties
to defend more effectively against suppression attacks and allows
us to present a blockchain compression and update algorithm that
tolerates an adversary below the $1/2$ threshold.

\item\textbf{A soft fork deployment methodology.}
As with any modification
to the core blockchain protocol, an important consideration is whether
it is possible to augment  a currently running instance of the
blockchain with the suggested extension. Perhaps somewhat surprisingly,
given its seemingly sharp divergence from standard blockchain
protocols, log-space mining is feasible to be rolled in
as a ``soft fork'' in blockchains supporting smart contracts such as Ethereum.
We give a practical construction for this in our
Supplementary Material.
\end{enumerate}

\import{./}{chapters/superlight/related}
