\section{Defining Security of Pegged Ledgers}
\label{sec:definition}

\todo{Clean up this section}

In this section we give the first formal definition of security desiderata for
a system of pegged ledgers (popularly often called sidechains).
We start by conveying its intuition and then proceed to the formal treatment.

We consider a setting where a set of parties run a protocol maintaining $n$
ledgers $\Ledger_1,\Ledger_2, \ldots,\Ledger_n$, each of the ledgers potentially
carrying many different assets. (This protocol might of course be a combination
of subprotocols for each of the ledgers.)
%At any given time $t$, in the
%view of any given party $\party$, a ledger $\Ledger_i$ will is at state
%$\LState_i^{\party}[t]$.
For each $i\in[n]$, we denote by~$\Assum_i$ the security assumption
required by $\Ledger_i$: For example, $\Assum_i$ may denote that
there has never been a majority of hashing power (or stake in a particular
asset, on this ledger or elsewhere) under the control of the adversary; that a
particular entity (in case of a centralized ledger) was not corrupted; and so
on.  We assume that all $\Assum_i$ are \emph{monotone} in the sense that once
violated, they cannot become true again. Formally, $\Assum_i$ is a
sequence of events $\Assum_i[t]$ for each time slot $t$ that satisfy
$\lnot\Assum_i[t] \Rightarrow \lnot\Assum_i[t+1]$ for all $t$.
%monotone
%predicate (which can only go from \textit{true} to \textit{false}) evaluated on
%the whole execution of the respective ledger protocol.
%We denote by $\Secure$ the set $\setd{i}{\Assum_i\text{ is satisfied}}$ of indices
%of currently uncompromised ledgers.

There is an a priori unlimited number of (types of) assets, each asset
representing e.g. a different cryptocurrency.
For simplicity we assume that assets of the same type
are fungible, but our treatment easily covers also non-fungible assets.
We will allow specific rules of behavior for each asset (called \emph{validity
languages}), and each asset behaves
according to these rules on each of the ledgers where it is present.

We will fix an operator $\merge(\cdot)$ that
 merges a set of ledger states
$\mathcal{L}=\{\LState_1,\LState_2,\ldots,\LState_n\}$ into a single ledger
state  denoted by $\merge(\mathcal{L})$. We will discuss concrete instantiations
of $\merge(\cdot)$ later, for now simply assume that some canonical way of
merging all ledger states into one is given.

Informally, at any point $t$ during the execution, our security definition only
provides guarantees to the  subset $\Secure$ of ledgers that have their security
assumptions $\Assum_i[t]$ satisfied (and hence are all considered uncorrupted). We
require that:
  \begin{itemize}
    \item[-] each ledger in $\Secure$ individually maintains both persistence and liveness;
    \item[-] for each asset~$\Asset$, when looking at the sequence of all
      $\Asset$-trans\-actions $\sigma$ that occurred on the ledgers in $\Secure$
      (sequentialized via the $\merge$ operator), there must exist a
      hypothetical sequence of $\Asset$-transactions $\tau$ that could have happened on the
      compromised ledgers, such that the merge of $\sigma$ and $\tau$ would be
      valid according to the validity language of $\Asset$.
  \end{itemize}
We now proceed to formalize the above intuition.

\begin{definition}[Assets, syntactically valid transactions]
  For an asset $\Asset$, we denote by~$\Trans_{\Asset}$ the \emph{valid
  transaction set of $\Asset$}, i.e., the set of all syntactically
  valid transactions involving $\Asset$.
  %We assume that $\Trans_{\Asset}$ is finite.
  For a ledger $\Ledger$ we denote by $\Trans_\Ledger$ the set of transactions
  that can be included into~$\Ledger$. For notational convenience, we define
  $\Trans_{\Asset,\Ledger}\defeq\Trans_{\Asset}\cap\Trans_\Ledger$.
  Let $\Assets(\Ledger)$  denote the set of all assets that are supported by
  $\Ledger$.
  Formally,
  $\Assets(\Ledger)
  \defeq
  \setd
    {\Asset}
    {\Trans_{\Asset,\Ledger}\neq\emptyset}$.
\end{definition}


%We will write $L \subseteq \Trans_\Ledger$ to indicate that $L$ is a sequence of
%transactions such that for all $\tx$ in the sequence $L$ we have that
%$\tx \in \Trans_\Ledger$.
% PG: if L is a sequence, the correct notation would be $L\in\Trans_\Ledger^*$,
% but I think this is clear and does not have to be spelled out.

We assume that each transaction pertains to a particular asset
and belongs to a particular ledger, i.e., for distinct
$\Asset_1\neq\Asset_2$ and
$\Ledger_1\neq\Ledger_2$, we have that
$\Trans_{\Asset_1}\cap\Trans_{\Asset_2}=\emptyset$ and
$\Trans_{\Ledger_1}\cap\Trans_{\Ledger_2}=\emptyset$.
However, our treatment can be easily generalized to alleviate this restriction.

We now generically characterize the \emph{validity} of a sequence of
transactions involving a particular asset. This is captured individually for each
asset via a notion of an asset's \emph{validity language},
which is simply  a set of words over the alphabet of this asset's transactions.
The asset's validity language is meant to capture how the asset is mandated to
behave in the system.
Let $\eps$ denote the empty sequence and $\concat$ represent
concatenation.

\begin{definition}[Asset validity language]
  For an asset $\Asset$,
  the \emph{asset validity language of $\Asset$}
  is any language
  $\ValLang_\Asset\subseteq\Trans_{\Asset}^*$
  that satisfies the following properties:
  \begin{enumerate}
    \item \textbf{Base.}
      $\eps\in\ValLang_\Asset$.
    \item \textbf{Monotonicity.}
      For any $w,w'\in\Trans_{\Asset}^*$ we have
      $
      w\not\in\ValLang_\Asset
      \Rightarrow
      w\concat w'\not\in\ValLang_\Asset
      $.
    \item \textbf{Uniqueness of transactions.}
      Words from $\ValLang_{\Asset}$ never contain the same transaction twice:
      for any $\tx\in\Trans_{\Asset}$ and any $w_1,w_2,w_3\in\Trans_{\Asset}^*$ we
      have $w_1\concat\tx\concat w_2\concat\tx\concat w_3\not\in\ValLang_{\Asset}$.
  \end{enumerate}
\end{definition}

The first condition in the definition above is trivial, the second one mandates
the natural property that if a sequence of transactions is invalid, it cannot
become valid again by adding further transactions. Finally, the third condition
reflects a natural ``uniqueness'' property of transactions in existing
implementations. While not necessary for our treatment, it allows for some
simplifications.

The following definition allows us to focus on a particular asset or ledger
within a sequence of transactions.

\begin{definition}[Ledger state projection]
  Given a ledger state $\LState$, we call a
  \emph{projection of~$\LState$ with respect to a set $\set{X}$}
  (and denote by $\proj_{\set{X}}(\LState)$) the ledger state that is obtained from
  $\LState$ by removing all transactions not in $\set{X}$. To simplify
  notation, we will use $\proj_\Asset$ and $\proj_{\I}$ as a shorthand for
  $\proj_{\Trans_\Asset}$ and $\proj_{\bigcup_{i\in\I} \Trans_{\Ledger_i}}$,
  denoting the projection of the transactions of a ledger state with respect to
  particular asset $\Asset$ or a particular set of individual ledger
  indices.
  Naturally, for a language $\ValLang$ we define the
  \emph{projected language}
  $\proj_\set{X}(\ValLang):=\setd{\proj_\set{X}(w)}{w\in\ValLang}$, which
  contains all the sequences of transactions from the original language,
  each of them projected with respect to $\set{X}$.
\end{definition}

The concept of {\em effect transactions} below captures ledger interoperability
at the syntactic level.

\begin{definition}[Effect Transactions]
\label{def:effect}
For two ledgers $\Ledger$ and $\Ledger'$, the \emph{effect mapping} is a
mapping of the form
$
  \eff{\Ledger}{\Ledger'}
\colon
\Trans_{\Ledger}
\to
\left(
  \Trans_{\Ledger'}
  \cup
  \{\noeffect\}
\right)
%\; .
$.
  A transaction $ \tx'=\eff{\Ledger}{\Ledger'}(\tx) \neq \noeffect$ is called the \emph{effect
transaction} of the transaction $\tx$.
\end{definition}

Intuitively, for any transaction $\tx\in\Trans_{\Ledger}$, the corresponding transaction
$\eff{\Ledger}{\Ledger'}(\tx) \in \Trans_{\Ledger'}\cup\{\noeffect\}$ identifies
the necessary effect  on ledger $\Ledger'$ of the  event of the inclusion of the transaction $\tx$ into the ledger
$\Ledger$. With foresight, in an implementation of a  system of ledgers where a ``pegging''   exists, the transaction
$\eff{\Ledger}{\Ledger'}(\tx)$ has to be eventually valid and includable in
$\Ledger'$ in response to the inclusion of $\tx$ in $\Ledger$.
Additionally, throughout the paper we assume that an effect transaction is
always clearly identifiable as such, and its corresponding ``sending''
transaction can be derived from it; our instantiation does have this property.


We use a special symbol $\noeffect$ to indicate that the transaction $\tx$ does
not necessitate any action on $\Ledger'$ (this will be the case for most
transactions).
We will now be interested mostly in transactions that \emph{do}
require an action on the other ledger.

\begin{definition}[Cross-Ledger Transfers]
For two ledgers $\Ledger$ and $\Ledger'$ and an effect mapping
$\eff{\Ledger}{\Ledger'}(\cdot)$, we refer to a transaction in
$\Trans_\Ledger$ that requires some effect on $\Ledger'$ as
a \emph{$(\Ledger,\Ledger')$-cross-ledger transfer
transaction} (or \emph{cross-ledger transfer} for short).
The set of all cross-ledger transfers is denoted by
  $\Trans_{\Ledger,\Ledger'}^\cl\subseteq\Trans_\Ledger$,
formally
$
  \Trans_{\Ledger,\Ledger'}^\cl
\defeq
\setd
  {\tx \in \Trans_\Ledger}
  {\eff{\Ledger}{\Ledger'}(\tx)\neq\noeffect}
%\; .
$.
\end{definition}

%\medskip
Given ledger states $\LState_1, \LState_2, \ldots,\LState_n$, we need to
consider a joint ordered view of the transactions in all these ledgers. This is provided
by the  $\merge$ operator. Intuitively, $\merge$  allows us to
create a combined view of multiple ledgers, putting all of the transactions
across multiple ledgers into a linear ordering.
%The $\merge$ function is formally defined later in this section.
%We are now ready to formally describe the requirements of the $\merge$ function.
We expect that even if certain ledgers are missing from its input,
$\merge$ is still able to produce a global ordering for the remaining ledgers.
With foresight, this ability of the $\merge$ operator will enable us to reason
about the
%firewall property
% PG: undefined yet
situation
when some ledgers fail: In that case, the respective
inputs to the $\merge$ function will be missing. The $\merge$ function
definition below depends on the $\effect$ mappings, we keep this dependence
implicit for simpler notation.

\begin{definition}[Merging ledger states]
  \label{def:merge}
  The $\merge(\cdot)$ function is any mapping taking a
  subset of ledger states $\mathcal{L} \subseteq
  \{\LState_1,\LState_2,\ldots,\LState_n\}$
  and producing a ledger state $\merge(\mathcal{L})$
  such that:
  \begin{enumerate}
    \item \textbf{Partitioning.}
      The ledger states in $\mathcal{L}$ are disjoint subsequences of
      $\merge(\mathcal{L})$ that cover the whole sequence $\merge(\mathcal{L})$.
    \item \textbf{Topological soundness.}
      %For any set of $n$ ledger states $\mathcal{L} = \{\LState_1,
      %\LState_2,\ldots\}$ and the effect mappings
      For any $i\neq j$ such that $\LState_i,\LState_j\in\mathcal{L}$
      and any two transactions $\tx \in
      \LState_i$ and $\tx' \in \LState_j$,
      if $\tx' = \eff{\Ledger_i}{\Ledger_j}(\tx)$
      then $\tx$ precedes $\tx'$ in $\merge(\mathcal{L})$.
  \end{enumerate}
\end{definition}

%From the above definition, it follows that, if for some ledger state
%$\LState_i$, $\tx_1$ precedes $\tx_2$ in $\LState_i$, then $\tx_1$ precedes
%$\tx_2$ in $\merge(\mathcal{L})$.

%In our treatment of pegged ledgers we will consider protocols that have certain
%common characteristics. We call any such protocol a {\em system-of-ledgers
%protocol}. Such a protocol $\Pi$ for a system of ledgers
%$\{\Ledger_i\}_{i\in[n]}$ is initialized among $n$ parties with some initial
%stake, and allows the environment (which is
%adversarial) to ask the issuance of transactions that can move state among parties. The
%standard assumptions on stake distribution apply, i.e., \emph{honest majority} and
%a \emph{finite bound} on stake redistribution. For a detailed treatment of these assumptions
%and their necessity, see~\cite{ouroboros}.

%In our construction, in the interest of space and readability, we will skip some
%of the formalities in the exact messages exchanged between the environment and
%the honest parties and imply them in our protocol description. This will be in
%understanding that the protocol can always be expressed with exact messages to
%the lowest level of Interactive Turing Machines as done in,
%e.g.,~\cite{backbone}.

We will require that our validity languages are \emph{correct} in the following
sense.

\begin{definition}[Correctness of $\ValLang_{\Asset}$]
  \label{def:correctness}
  A validity language $\ValLang_{\Asset}$ is \emph{correct}
  with respect to
      a mapping $\merge\left(\cdot\right)$,
if
  for any ledger states
  $\mathcal{L}\defeq(\LState_1, \ldots, \LState_n)$
  such that
  $
    \proj_{\Asset}\left(
      \merge\left(
       \mathcal{L}
      \right)
    \right)
    \in \ValLang_{\Asset}
  $,
indices $i \neq j$,
and any cross-ledger transfer %transaction
$\tx\in\LState_i\cap\Trans_{\Ledger_i,\Ledger_j}^\cl$
such that  %its effect transaction
  $\eff{\Ledger_i}{\Ledger_j}(\tx)=\tx'\neq\noeffect$
  is not in $\LState_j$,
we have
$$
  \proj_{\Asset}\left(
    \merge\left(
      \LState_1, \ldots, \LState_i, \ldots,
      %\LState_{j-1},
      \LState_j\concat\tx',
      %\LState_{j+1},
      \ldots, \LState_n
      \right)
  \right)
\in \ValLang_{\Asset}
\; .
$$
\end{definition}

The above definition makes sure that if a cross-ledger transfer of an asset
$\Asset$  is included into
some ledger $\Ledger_i$ and mandates an effect transaction on $\Ledger_j$, then
the inclusion of this effect transaction will be consistent with
$\ValLang_\Asset$. Note that this does not yet guarantee that the effect
transaction will indeed be included into $\Ledger_j$, this will be provided by
the liveness of $\Ledger_j$ required below.

%The above definition captures the fact that the protocol~$\Pi$
%advances the ledgers so that an effect transaction will always be permitted in the view
%of a single party that has adopted the corresponding ledgers.
%The security of the pegging construction is captured by the following
%definition.

We are now ready to give our main security definition.
In what follows, we call a \emph{system-of-ledgers protocol} any protocol run by a
(possibly dynamically changing) set of parties that maintains an evolving state
of $n$ ledgers $\{\Ledger_i\}_{i\in[n]}$.

\begin{definition}[Pegging security]
  \label{def:security}
  A system-of-ledgers protocol~$\Pi$ for $\{\Ledger_i\}_{i\in[n]}$ is
  \emph{pegging-secure} with liveness parameter $u\in\N$
  with respect to:
  \begin{itemize}
    \item[-]
      a set of assumptions $\Assum_i$ for ledgers $\{\Ledger_i\}_{i\in[n]}$,
    \item[-]
      a merge mapping $\merge\left(\cdot\right)$,
    \item[-]
      validity languages $\ValLang_\Asset$ for each %asset
      $\Asset\in\bigcup_{i\in[n]}\Assets(\Ledger_i)$,
  \end{itemize}
  if for all PPT adversaries, all slots $t$
  and for
  $
  \Secure_t\defeq\setd{i}{\Assum_i[t] \text{ holds}}
  %\; ,
  $
  we have that except with negligible probability in the security parameter:
  \begin{description}
    \item \textbf{Ledger persistence:}
      For each $i\in\Secure_t$, $\Ledger_i$ satisfies the persistence property. %parametrized by $k\in\N$.
    \item \textbf{Ledger liveness:}
      For each $i\in\Secure_t$, $\Ledger_i$ satisfies the liveness property para\-metrized by $u$.
    \item \textbf{Firewall:}
      For all $\Asset\in\bigcup_{i\in\Secure_t}\Assets(\Ledger_i)$,
      %we have:
      $$
      \proj_{\Asset}\left(
        \merge\left(
          \setd
            {\Ledger_i^{\cup}[t]}
            {i\in\Secure_t}
        \right)
      \right)
      \in
      \proj_{\Secure_t}(\ValLang_{\Asset})
      \; .
      $$
  \end{description}
\end{definition}

%The \textit{firewall property} in the above security definition captures the
%fact that assets remain valid after $\merge(\cdot)$.
%The reason why this
%property really does capture firewalling is as follows.
Intuitively, the firewall property above gives the following guarantee:
If the security
assumption of a particular sidechain has been violated, we demand that the
sequence of transactions $\sigma$ that appears in the still uncompromised
ledgers is a valid projection of some word from the asset validity language onto
these ledgers. This means that there exists a sequence of transactions $\tau$
that \emph{could have happened} on the compromised ledgers, such that it would
``justify'' the current state of the uncompromised ledgers as a valid state.  Of
course, we don't know whether this sequence $\tau$ actually occurred on the
compromised ledger, however, given that this ledger itself no longer provides
any reliable state, this is the best guarantee we can still offer to the
uncompromised ledgers.

Looking ahead, when we define a particular validity language for our concrete,
fungible, constant-supply asset, we will see that this property will translate
into the mainchain maintaining ``limited liability'' towards the sidechain: the
amount of money transferred back from the sidechain can never exceed the amount
of money that was previously moved towards the sidechain, because no plausible
history of sidechain transactions can exist that would justify such a transfer.
