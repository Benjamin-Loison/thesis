\subsection{Introduction}

\todo{Clean up this section}

Blockchain protocols and their most
prominent application so far,  cryptocurrencies like Bitcoin~\cite{bitcoin},
have been gaining increasing popularity and acceptance by
a wider community.   While enjoying wide adoption,
there  are several fundamental open questions remaining to be resolved
that include (i)
         Interoperability:
           How can different blockchains interoperate and exchange
           assets or other data?
(ii)  Scalability:
           How can blockchain protocols scale, especially proportionally to the number of           participating nodes?
(iii)
         Upgradability:
           How can a deployed blockchain protocol codebase evolve to support a new
           functionality, or correct an implementation problem?

The main function of a blockchain protocol  is to organise \textit{application
data} into \textit{blocks} so that a set of nodes that evolves over time can
arrive eventually to consensus about the sequence of events that took place.
The consensus component can be achieved in a number of ways, the most popular
is using proof-of-work~\cite{pow} (cf. \cite{bitcoin,backbone}),
while a promising alternative is to use proof-of-stake
(cf. \cite{algorand,ouroboros,snowwhite,praos}).
Application data typically
consists of \textit{transactions} indicating some transfer of value as in the
case of Bitcoin~\cite{bitcoin}.  The transfer of value can be conditioned on
arbitrary predicates called \textit{smart contracts} such as, for example, in
Ethereum~\cite{buterin,wood}.

The conditions used to validate transactions %for each node of the system
depend on local blockchain events according to the view of each node and they
typically cannot be dependent on other blockchain sessions. Being
able to perform operations across blockchains, for instance from a main blockchain
such as Bitcoin to a ``sidechain'' that has some enhanced functionality, has been
frequently considered  a fundamental technology enabler in the blockchain space.\footnote{See
e.g., \url{https://blockstream.com/technology/} and \cite{sidechains}.  }


Sidechains, introduced in \cite{sidechains},  are a
way for multiple blockchains to communicate with each other and have one react
to events in the other. Sidechains can exist in two forms. In the first case, they are simply a mechanism for two
existing \textit{stand-alone blockchains} to communicate, in which case any of
the two blockchains can be the sidechain of the other and they are treated as
equals. In the second case, the
sidechain can be a ``child'' of an existing blockchain, the \textit{mainchain},
in that its genesis block, the first block of the blockchain, is somehow seeded
from the parent blockchain and the child blockchain is meant to depend on the
parent blockchain, at least during an initial bootstrapping stage.

A sidechain system can choose to enable certain types of interactions between
the participating block\-chains. The most basic interaction
is the transfer of assets from
one blockchain to another. In this application, the nature of the asset
transferred is retained in that it is not transformed into a different class of
asset (this is in contrast to a related but different concept of \emph{atomic
swaps}).
As such, it maintains its value and may also be transferred back.
The ability to move assets back and
forth between two chains is sometimes referred to as a \textit{2-way peg}. Provided
the two chains are both secure as individual blockchains, a secure
sidechain protocol construction allows this security to be carried on to
cross-chain transfers.

A secure sidechain system could be of a great value vis-\`a-vis all three
of the pressing open questions in blockchain systems mentioned above. Specifically:

%\noindent
{\em Interoperability.} There are currently hundreds of
    cryptocurrencies deployed in production. Transferring assets between
    different chains requires transacting with intermediaries (such as exchanges). Furthermore,
    there is no way to securely interface with another blockchain to react to
    events occurring within it. Enabling sidechains allows
    blockchains of different nature to communicate, including interfacing with
    the legacy banking system which can be made available through the use of
    a private ledger.

%\noindent
{\em   Scalability.} While sidechains were not originally proposed for
    scalability purposes, they can be used to off-load the load of a blockchain
    in terms of transactions processed. As long as 2-way pegs are enabled, a
    particular sidechain can offer specialization by, e.g., industry, in order
    to avoid requiring the mainchain to handle all the transactions occurring
    within a particular economic sector. This provides a straightforward way to
    ``shard'' blockchains, cf. \cite{sharding,omniledger,rapidchain}.

%\noindent
{\em Upgradability.} A child sidechain can be created from a parent
    mainchain as a means of exploring a new feature, e.g., in the scripting language, or
    the consensus mechanism
    without requiring a soft, hard, or velvet fork~\cite{nipopows,velvet}. The
    sidechain does not need to maintain its own separate currency, as value can
    be moved between the sidechain and the mainchain at will. If the feature of
    the sidechain proves to be popular, the mainchain can eventually be
    abandoned by moving all assets to the sidechain, which can become the new
    mainchain.

Given the benefits listed above for distributed ledgers, there is a pressing
need to address the question of sidechain security and feasibility, which so far, perhaps surprisingly, has not received any proper formal treatment.

\medskip

\noindent
{\bf Our contributions. }
First, we formalize the notion of sidechains by proposing a rigorous cryptographic
definition, the first one to the best of our knowledge.
The definition is abstract enough to be able to capture the security for
blockchains based on proof-of-work, proof-of-stake, and other consensus mechanisms.

A critical security feature of a  sidechain system that we formalise
is the \textit{firewall property} in
which a catastrophic failure in one of the chains, such as a violation of its
security assumptions, does not make the other chains vulnerable providing a
sense of limited liability.%
\footnote{To follow the analogy with the term of limited liability in corporate
law, a catastrophic sidechain failure is akin to a corporation going bankrupt
and unable to pay its debtors. In a similar fashion, a sidechain in which the
security assumptions are violated may not be able to cover all of its debtors.
%In fact,
%because it can be under attack,
We give no assurances regarding assets residing on a sidechain if its
security assumptions are broken.
 However, in the same way that stakeholders of a corporation are personally
protected in case of corporate bankruptcy, the mainchain is also protected in
case of sidechain security failures. Our security will guarantee that
each incoming transaction from a sidechain will always have a valid
explanation  in the sidechain ledger
independently of whether the underlying
security assumptions are violated or not. A simple embodiment of this rule
is that a sidechain can return to the mainchain at most as many coins as they have been sent to the sidechain over all time.
}
The   firewall property formalises and generalises the concept of a blockchain \textit{firewall} which was described in high level in~\cite{sidechains}. Informally the blockchain firewall suggests
that no more money can ever return from the sidechain than the amount that was moved
into it. Our general firewall property  allows relying on an
arbitrary definition of exactly how assets can correctly be moved back and forth
between the two chains, we capture this by a so-called
\textit{validity language}. In case of failure, the firewall
ensures that transfers from the sidechain into the mainchain are rejected unless
there exists a (not necessarily unique) plausible history of events on the sidechain that could, in case the
sidechain was still secure, cause the particular transfers to take place.
%If
%the correctness validity language describing the valid transfers between the
%mainchain and the sidechain consists of a simple law of conservation, then the
%limited liability  property captures  precisely a blockchain firewall.

Second, we  outline a concrete exemplary
construction for sidechains for proof-of-stake
blockchains. For conciseness our construction is described
with respect to a generic PoS blockchain consistent with the  Ouroboros
protocol~\cite{ouroboros} that underlies the Cardano blockchain, which is currently one of the largest
pure PoS blockchains by market capitalisation,\footnote{See \url{https://coinmarketcap.com}. }
nevertheless we also discuss how to modify our construction to
operate for
Ouroboros Praos~\cite{praos},
Ouroboros Genesis~\cite{genesis},
Snow White \cite{snowwhite}
and
Algorand \cite{algorand}.

We prove our construction secure   using standard cryptographic
assumptions. We show that our construction (i) supports safe cross-chain value
transfers when the security assumptions of both chains are satisfied, namely
that a majority of honest stake exists in both chains, and (ii) in case of a
one-sided failure,  maintains the firewall property, thus
containing the damage
to the chains whose security conditions have been violated.
%

A critical consideration in a sidechain construction is safeguarding
a new sidechain in its initial ``bootstrapping'' stage against a ``goldfinger'' type of attack \cite{mining-economics,hostile}. Our construction
features a mechanism we call {\em merged-staking}
that allows  mainchain stakeholders who have signalled
sidechain awareness to create sidechain blocks even without moving stake
to the sidechain. In this way, sidechain security can be maintained
assuming honest stake majority among the entities that have signaled sidechain
awareness that, especially in the bootstrapping stage, are expected to be
a large superset of the set of stakeholders that maintain assets
in the sidechain.
%Refel the maintainers
%of the sidechain need to monitor the blocks generated by the mainchain, but the
%maintainers of the mainchain need not be aware of the sidechain's blocks.

Our techniques can be used to facilitate various forms of 2-way peggings
between two chains. As an illustrative example we focus on a parent-child
mainchain-sidechain
configuration where sidechain nodes follow also the mainchain (what we call direct observation) while mainchain nodes need to be able to receive cryptographically
certified signals
from the sidechain maintainers,
taking advantage of the proof-of-stake nature of the underlying protocol. This is achieved by having
mainchain nodes maintain sufficient information about the sidechain that allows
them to authenticate a
small subset of  sidechain stakeholders that is sufficient to
reliably represent the view of a stakeholder majority on the sidechain.
%a recent snapshot of the stakeholder distribution on
%the sidechain (committed in the form of a Merkle-Patricia trie)
This piece of information is updated in regular intervals to account
for  stake shifting on the sidechain.
Exploiting this, each withdrawal  transaction from the sidechain to the mainchain
is signed by this  small subset of sidechain stakeholders.
%that is sufficient to reliably represent the view of a stakeholder majority on the sidechain.
To minimise overheads we batch this authentication information and all the withdrawal transactions from
the sidechain in a single message that will be prepared once per ``epoch.'' We will
refer to this signaling  as
 {\em cross-chain certification}.

In greater detail, adopting some terminology  from \cite{ouroboros},
the sidechain certificate  is constructed by obtaining
signatures  from the set of so-called \emph{slot leaders} of the last
$\Theta(k)$ slots of the previous epoch, where $k$ is the security parameter.
Subsequently, these signatures will be combined together with all necessary
information to convince the mainchain nodes (that do not have access to the
sidechain) that the sidechain certificate is valid.

We abstract the notion of
this trust transition into a new cryptographic primitive called
\emph{ad-hoc threshold multisignatures (ATMS)}
that we implement in three distinct ways. The first one
simply concatenates signatures of elected slot leaders. While secure, the disadvantage of
this implementation is that the size of the sidechain certificate is
$\Theta(k)$ signatures.
An improvement can be achieved by
employing multisignatures and Merkle-tree hashing for verification key aggregation;  using this we can drop the sidechain-certificate size to
$\Theta(r)$ signatures where $r$  slot leaders do not participate in its generation; in the optimistic case $r\ll k$ and thus this scheme can be
a significant improvement in practice. Finally, we show that STARKs and bulletproofs~\cite{ben2017scalable,bulletproofs}
can be used to bring down the size of the certificate to be optimally succinct in the random oracle model.  We observe that in the case of an active sidechain (e.g., one that returns assets
at least once per epoch) our construction with succinct sidechain
certificates has optimal storage requirements in the mainchain.

\noindent {\bf  Related work. }
Sidechains were first proposed as a high level concept in~\cite{sidechains}.
%To the best of
%our knowledge, so far  no
%formal definition of sidechain security exists, and existing constructions are not
%accompanied by formal analysis.
Notable proposed implementations of the concept are given in~\cite{drivechains,lerner}.
In these works, no formal proof of security
is provided and their performance is sometimes akin to maintaining the whole
blockchain within the sidechain, limiting any potential scalability gains.
%
There have been several attempts to create various cross-chain transfer
mechanisms including Polkadot~\cite{polkadot},
Cosmos~\cite{tendermint}, Blockstream's Liquid \cite{federated-interoperability} and Interledger \cite{interledger}. These constructions  differ in various aspects from our work including in that
they focus on proof-of-work or private (Byzantine) blockchains, require
federations, are not decentralized and --- in all cases ---
 lack a formal security model and analysis.
Threshold multi-signatures were considered before, e.g., \cite{pass-asynchronous},
without the ad-hoc characteristic we consider here.
A related primitive that has been considered as potentially useful for enabling
proof-of-work (PoW) sidechains (rather than PoS ones) is a (non-interactive) proof of
proof-of-work~\cite{popow,nipopows}; nevertheless,  these works do not give a
formal security definition for sidechains, nor provide a complete sidechain
construction. We reiterate that while we focus on  PoS, our definitions and model
are fully relevant for the PoW setting as well.
