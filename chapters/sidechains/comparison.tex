\subsection{Comparison}

We now compare alternatives for proof-of-burn proposed in previous work. Our burn primitive captures all of these schemes.

\newcommand{\opreturn}{\texttt{OP\_RETURN}}

\noindent
\textbf{$\opreturn$.}
Bitcoin provides a native opcode called  $\opreturn$~\cite{opreturn} which can be used for burning.
Unfortunately,
standard wallets do not provide a user friendly interface for creating $\opreturn$ transactions.
However, it benefits the Bitcoin network by allowing the UTXO to be pruned, at
the cost of not being uncensorable.
Similarly to $\opreturn$, any provably failing Bitcoin script can be used for
burning~\cite{stewart}.

\noindent
\textbf{P2SH $\opreturn$.}
An $\opreturn$ or other provably failing script can also be used as the redeemScript for a Pay to Script Hash (P2SH)~\cite{p2sh} address. It is unspendable since there is no scriptSig that could make the script succeed. Additionally, it is uncensorable if the tag is not revealed. Finally, the scheme is user friendly since any regular wallet can create a burn transaction.

\noindent
\textbf{Nothing-up-my-sleeve.}
An address is manually crafted, so that it is clear it was not generated from a regular keypair. For example, the all-zeros address is considered nothing-up-my-sleeve\footnote{The Bitcoin address \texttt{1111111111111111111114oLvT2} encodes the all-zeros string and has received more than $50{,}000$ transactions dating back to Aug 2010.}. It is hard to obtain a public key hashing to this address, thus funds sent to it are unspendable. Because no metadata can be associated with such a burn, this scheme is not binding.

We compare the aforementioned schemes on whether they satisfy the burn protocol properties we define: Binding, unspendability and uncensorability. Additionally, we compare them based on how easily they translate to multiple cryptocurrencies. For instance, $\opreturn$ and P2SH $\opreturn$ rely on Bitcoin Script semantics and do not directly apply to any non-Bitcoin based cryptocurrencies like Monero~\cite{cryptonote}, thus we say they are not \emph{flexible}. The comparison is illustrated on Table~\ref{table:comparison}.

\begin{table}[h!]
    \newcommand{\y}{$\bullet$}
    \newcommand{\n}{}
    \centering
    \caption{Comparison between proof-of-burn schemes.\label{table:comparison}}

    \begin{tabular}{ |r|c|c|c|>{\columncolor{lightgray}} c| }
     \hline
                                    & $\opreturn$ & P2SH $\opreturn$ & \begin{tabular}{@{}c@{}}Nothing up\\ my sleeve\end{tabular} & \begin{tabular}{@{}c@{}}$a \xor 1$\\ \textbf{(this work)}\end{tabular} \\
     \hline
     Binding                        & \y      & \y       & \n          & \y  \\
     Flexible                       & \n      & \n       & \y          & \y  \\
     Unspendable                    & \y      & \y       & \y          & \y  \\
     Uncensorable                   & \n      & \y       & \y          & \y  \\
     User friendly                  & \n      & \y       & \y          & \y  \\
     \hline
    \end{tabular}
\end{table}
