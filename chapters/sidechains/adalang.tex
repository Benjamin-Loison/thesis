\subsection{A Concrete Asset $\ada$}
\label{sec:inst}

We now present an example of a simple fungible asset with fixed supply, which we
denote $\ada$, and describe its validity language $\adalang$.
%or $\ADA$ or $\ValLang_{\ADA}$.
This will be the asset (and validity language) considered in our construction
and proof. While $\adalang$ is simple and natural, it
allows us to exhibit the main features of our security treatment and illustrate
how  it can be applied to more complex languages such as those capable
of capturing smart contracts; we omit such extensions in this version.
%for now due to lack of space.
Note that our language is account-based, but a UTXO-based validity
language can be considered in a similar manner.

\subsubsection{Instantiating $\ValLang_{\Asset}$}
\label{sec:inst-vallang}

The validity language $\adalang$ for the asset $\ada$ considers
two ledgers: the mainchain ledger $\Ledger_0\defeq\MC$ and the sidechain ledger
$\Ledger_1\defeq\SC$. For this asset, every transaction $\tx\in\adatrans$ has the form
$
\tx=\left(
  \txid,
  \llid,
  (\send, \sendAccount),
  (\rec, \recAccount),
  \amount,
  \sig
\right)
%\; ,
$,
where:
\begin{itemize}
  \item $\txid$ is a transaction identifier that prevents replay attacks.
    We assume that
    $\txid$ contains sufficient information to identify $\llid$ by inspection
    and that this is part of syntactic transaction validation.
  \item $\llid \in \{0, 1\}$ is the ledger index where the transaction belongs.
  \item $\send \in \{0, 1\}$ is the index of the sender ledger $\Ledger_\send$ and
    $\sendAccount$ is an account  on this ledger,
    this is the sender account. For simplicity, we assume that $\sendAccount$ is
    the public key of the account.
  \item $\rec \in \{0, 1\}$ is the index of the recipient ledger $\Ledger_\rec$ and
    $\recAccount$ is an account (again represented by a public key) on this
    ledger, this is the recipient account.
    We allow either $\Ledger_\send=\Ledger_\rec$, which denotes a \emph{local
    transaction}, or $\Ledger_\send\neq\Ledger_\rec$, which denotes a
    \emph{remote transaction} (i.e., a cross-ledger transfer).
\item $\amount$ is the amount to be transferred.
\item $\sig$ is the signature of the sender, i.e. made with the private key
      corresponding to the public key $\sendAccount$ on the plaintext
      $(\txid, (\send, \sendAccount), (\rec, \recAccount), \amount)$.
\end{itemize}
The correctness of $\llid$ is enforced by the ledgers, i.e., for
both $i\in\{0,1\}$ the set $\Trans_{\ada,\Ledger_i}$ only contains transactions
with $\llid=i$.
Note that although we sometimes notationally distinguish between an account and the
public key that is associated with it, for simplicity we will assume that these
are either identical or can always be derived from one another (this assumption is not
essential for our construction).

\import{chapters/sidechains/}{algorithms/alg.validity.tex}

The membership-deciding algorithm for $\adalang$
is presented in Algorithm~\ref{alg.validity}.
It
processes the sequence of transactions $(\tx_1, \tx_2, \dots,
\tx_m)$ given to it as input in their order. Assuming transactions are
syntactically valid, the function verifies for each transaction $\tx_i$ the
freshness of $\txid$, validity of the signature, and availability of sufficient
funds on the sending account. For an intra-ledger transaction (i.e., one that
has $\send=\rec$), these are all the performed checks.

More interestingly, $\adalang$ also allows for cross-ledger transfers. Such
transfers are expressed by a pair of transactions in which $\send \neq \rec$.
The first transaction appears in $\llid = \send$, while the second transaction
appears in $\llid = \rec$. The two transactions are identical except for this
change in $\llid$ (this is the only exception to the $\txid$-freshness
requirement). Every receiving transaction has to be preceded by a matching
sending transaction. Cross-chain transactions have to, similarly to intra-ledger
transactions, conform to laws of balance conservation.

%\paragraph{Remarks about this validity language.}
Note that $\adalang$ does not require that every ``sending''
cross-ledger transaction on the sender ledger is matched by a ``receiving''
transaction on the receiving ledger. Hence, if the asset $\ada$ is sent from
ledger $\Ledger_{\send}$ but has not yet arrived on $\Ledger_{\rec}$ then
validity for this asset is \emph{not} violated. All the validity language
ensures is that appending the $\mathsf{sidechain\_receive}$ transaction to the
$\rec$ will eventually be a valid way to extend the receiving ledger, as long as
the $\mathsf{sidechain\_send}$ transaction has been included in $\send$.

\subsubsection{Instantiating $\eff{\Ledger_i}{\Ledger_j}$}
For the simple asset $\ada$ outlined above, every cross-ledger transfer is a
``sending'' transaction $\tx$ with
$\Ledger_{\llid}=\Ledger_{\send}\neq\Ledger_{\rec}$ appearing in $\Ledger_\send$,
and its effect transaction is a ``receiving'' transaction $\tx'$ with
$\Ledger_{\llid}=\Ledger_{\rec}\neq\Ledger_{\send}$ in~$\Ledger_\rec$ that is
otherwise identical (except for the different $\llid' = 1-\llid$).
%and signature).
Hence, we
define $\eff{\Ledger_\send}{\Ledger_\rec}(\tx)=\tx'$ exactly for all these transactions
and no other.

\subsubsection{Instantiating $\merge(\cdot)$}
It is easy to construct a canonical function $\merge(\cdot)$ once we see its inputs not only as
ledger states (i.e., sequences of transactions) but we also exploit the
additional structure of the blockchains carrying those ledgers.
%We first define two natural properties that the $\merge(\cdot)$ function should
%satisfy.
%
% PG: seems irrelevant here?
%The first point requires sorting transactions within a particular ledger
%$\LState_i$. While formally the ordering within $\LState_i$ is available to the
%$\merge$ function, it is useful to specify that, even if it were not provided, a
%natural $\merge(\cdot)$ would simply organize transactions according first to
%the timestamps (i.e., slot numbers) of the blocks they are contained in and
%second by the order of transactions within a block. The second point makes the
%more material observation that causality must be preserved in a merge.
%
%Subsequently, we define a concrete way to merge ledgers in a way satisfying the
%above conditions.
The \textit{canonical merge} of the set of ledger states
$\mathcal{L}$ is the lexicographically minimum topologically sound merge,
in which transactions of ledger $\LState_i$ are compared favourably to
transactions in $\LState_j$ if $i < j$.
However, note that the construction we provide
below will work for any topologically sound merge function.

\bigskip
\noindent
One can easily observe the following statement.

\begin{proposition}
The validity language $\adalang$
  is correct (according to Definition~\ref{def:correctness})
with respect to the $\merge$ function
defined above.
\end{proposition}

%\newpage
