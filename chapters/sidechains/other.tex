\section{Adaptation to Other Proof-of-Stake Blockchains}
\label{section:other}

\todo{Move this section into Proof-of-Stake chapter}

Our construction can be adapted to work with other provably secure
proof-of-stake blockchains discussed in Section~\ref{sec:pos}:
Ouroboros Praos~\cite{EC:DGKR18},
Ouroboros Genesis~\cite{genesis},
Snow White~\cite{DBLP:journals/iacr/BentovPS16a}, and
Algorand~\cite{algorand}.
Here we assume some familiarity with the
considered protocols and refer the interested reader to the original papers for
details.
%defer a more self-contained discussion of these
%adaptations to the full version of this paper.

\subsection{Ouroboros Praos and Ouroboros Genesis}

These protocols~\cite{EC:DGKR18,genesis} are strongly related and differ from each
other only in the chain-selection
rule they use, which is irrelevant for our discussion here, hence we
consider both of the protocols simultaneously.
Ouroboros Praos was shown secure in the
semi-synchronous model with fully adaptive corruptions (cf.
Section~\ref{sec:model}) and this result extends to Ouroboros Genesis.
Despite sharing the basic structure with Ouroboros,
they differ in several significant points which we now outline.

The slot leaders are elected differently: Namely, each party for each
slot evaluates a verifiable random function (VRF,~\cite{PKC:DodYam05}) using
the secret key associated with their stake, and providing as inputs to the VRF
both the slot index and the epoch randomness. If the VRF output is below a
certain threshold that depends on the party's stake, then the party is an
eligible slot leader  for that slot, with the same consequences as in Ouroboros.
%and is allowed to create, sign, and broadcast a block (containing transactions
%that move stake among stakeholders).
Each leader then includes into the block it creates the VRF output and a
proof of its validity to certify her eligibility to act as slot leader.
%
The probability of becoming a slot leader is roughly proportional to the amount of
stake the party controls, however now it is independent for each slot and each
party, as it is evaluated locally by each stakeholder for herself.  This local
nature of the leader election implies that there will inevitably be some slots
with no, or several, slot leaders.
%
In each epoch $j$, the stake distribution used in Praos and Genesis
for slot leader
election corresponds to the distribution recorded in the ledger up to the last
block of epoch $j-2$.  Additionally, the \emph{epoch randomness $\rnd_j$} for epoch $j$
is derived as a hash of additional VRF-values included into blocks
from the first two thirds of epoch $j-1$ for this purpose by the respective slot
leaders.
%
Finally, the protocols use \emph{key-evolving signatures} for
block signing, and in each slot the honest parties are mandated to update their
private key, contributing to their resilience to adaptive corruptions.

%\paragraph{Security}
Ouroboros Praos was shown~\cite{EC:DGKR18} to achieve
persistence and liveness under weaker assumptions than Ouroboros,
namely:
%\begin{enumerate}
  %\item
    (1) $\Delta$-semi-synchronous communication
    (where $\Delta$ affects the security bounds
    but is unknown to the protocol);
  %\item
    (2) majority of the stake is always controlled by honest parties.
  %\item
    %(3) the stake shift per epoch is limited.
%\end{enumerate}
In particular, Ouroboros Praos is secure in face of fully adaptive corruptions
without any corruption  delay.
Ouroboros Genesis provides the same guarantees as Praos, as well as several
other features that will not be relevant for our present
discusion.

\noindent
\textbf{Construction of Pegged Ledgers.}
The main difference compared to our treatment of Ouroboros
would be in the construction of the sidechain certificate
(cf. Section~\ref{sec:sccert}). The need for a modification is caused by the
private, local leader selection using VRFs in these protocols, which makes it
impossible to identify the set of slot leaders for the suffix of an epoch at the
beginning of this epoch, as done for Ouroboros.

The sidechain certificate included in $\MC$ at the beginning of epoch $j$
would hence contain the following, for parameters $Q$ and $T$ specified below:
\begin{enumerate}
\item
the epoch index;
\item
a Merkle commitment to the list of withdrawals as in the case of Ouroboros;
\item
a Merkle commitment to the $\SC$ stake distribution $\SDb_{j}$;
\item
a list of $Q$ public keys;
\item
$Q$ inclusion proofs (with respect to $\SDb_{j-1}$ contained in the previous
certificate) and $Q$ VRF-proofs certifying that these $Q$ keys belong
to slot leaders of $Q$ out of the last $T$ slots in epoch $j-1$;
\item
$Q$ signatures from the above $Q$ public keys on the above; these can be
replaced by a single aggregate signature to save space on $\MC$.
\end{enumerate}

The parameters $Q$ and $T$ have to be chosen in such a way that with
overwhelming probability, there will be a chain growth of at least $Q$ blocks
during the last $T$ slots of epoch $j-1$,
but the adversary controls $Q$ slots in this period only
with negligible probability (and hence at least one of the signatures will have
to come from an honest slot leader).
The existence of such constants for $T=\Theta(k)$ was shown in~\cite{genesis}.

While the above sidechain certificate is larger (and hence takes more space on
$\MC$) than the one we propose for Ouroboros, a switch to Ouroboros
Praos or Genesis would also bring several advantages. First off, both
constructions would give us security in the semi-synchronous model with fully
adaptive corruptions (as shown in~\cite{EC:DGKR18,genesis}),
%and discussed in Section~\ref{sec:ouroboros}),
and the use of
Ouroboros Genesis would allow newly joining players to bootstrap from the
mainchain genesis block only---without the need for a trusted checkpoint---as discussed
extensively in~\cite{genesis}.

%Additionally, since the epoch randomness generation in these two protocols is
%significantly more efficient than in Ouroboros and only involves a hash function
%invocation, our construction could be adapted so that $\SC$ produces its own
%epoch randomness in this way instead of reusing the mainchain randomness,
%opening doors to a greater separation of the two chains.

\subsection{Snow White}

The high-level structure of Snow White execution is similar to the protocols we have
already discussed: it contains epochs, committees that are sampled for each
epoch based on the stake distribution recorded in the blockchain prior to that
epoch, and randomness used for this sampling produced by hashing special nonce
values included in previous blocks. Hence, our construction can be adapted to
work with Snow White-based blockchains in a straightforward manner.


\subsection{Algorand}

Algorand does not aim for the so-called eventual consensus. Instead it runs a
full Byzantine Agreement protocol for each block before moving to the next
block, hence blocks are immediately finalized. Consider a setting with $\MC$
and $\SC$ both running Algorand. The main difficulty to address when
constructing pegged ledgers is the continuous  authentication of the
sidechain certificate constructed by $\SC$-maintainers for $\MC$ (other aspects,
such as deposits from $\MC$ to $\SC$ work analogously to what we
described above).
%
As Algorand does not have epochs, and creating and processing a sidechain
certificate for each block is overly demanding, a natural choice is to
introduce a parameter $R$ and execute this process only once every $R$ blocks.
Namely, every $R$ blocks, the $\SC$-maintainers produce a certificate that
the $\MC$-maintainers insert into the mainchain. This certificate most
importantly contains:
\begin{enumerate}
\item
a Merkle commitment to the list of withdrawals in the most recent $R$-block
    period;
\item
a Merkle commitment to the full, most recent stake distribution $\SDb_j$ on
    $\SC$;
\item
a sufficient number of signatures from a separate committee
%that would be elected via the
%Algorand sortition mechanism from the previously-committed stake distribution
%$\SDb_{j-1}$,
certifying the above information, together with proofs justifying the membership
of the signature's creators in the committee.
\end{enumerate}
This additional committee is sampled from $\SDb_{j-1}$ (the stake
distribution committed to in the previous sidechain certificate) via Algorand's
private sortition mechanism such that the expected size of the
committee is large enough to ensure honest supermajority
(required for Algorand's security) translates into a strong
honest majority within the committee.
Note that the sortition mechanism also allows for a succinct proof of membership
in the committee.
The members of the committee then
insert their individual signatures (signing the first two items in the
certificate above) into the $\SC$ blockchain during the period of $R$ blocks
preceding the construction of the certificate.
%
All the remaining mechanics of the pegged ledgers are a direct analogy of
our construction above.
