\subsection{Sidechains as Smart Contracts}

\todo{Clean up this section}

Bitcoin~\cite{bitcoin} is the most successful \emph{cryptocurrency} to
date. It introduced \emph{blockchains}, a
cryptographic consensus protocol in which \emph{transactions} are organized
into \emph{blocks} which are put in a mutually agreed sequence despite the
presence of adversaries. Consensus is achieved via
\emph{proof-of-work}~\cite{pow} which is the precondition for block
validity.  Transactions moving value within blockchains have been proven to
be secure and that consensus is eventually achieved, cf.
\cite{backbone,pass-asynchronous,varbackbone}.

Ethereum~\cite{buterin} extends Bitcoin's functionality introducing
Turing-complete \emph{smart contracts} programmed in
languages like Solidity which run on top of the Ethereum Virtual
Machine~\cite{wood}. These contracts execute autonomously. The smart contracts
are confined to access data only within the blockchain itself, such as previous
transactions and blocks. Access to external data requires a trusted
third party or group thereof to vouch for the data validity~\cite{towncrier}.

Sidechains~\cite{sidechains} are a mechanism for cross-chain communication in
blockchains. They allow smart contracts on one blockchain to receive and
react to \textit{events} taking place on another blockchain without the need
of trusted parties. Despite their widely agreed usefulness
there exist no constructions that are decentralised and efficient at the same
time.

\todo{Move to contributions in introduction}
\noindent\textbf{Our contributions. } In this paper, we introduce the first
trustless construction for proof-of-work sidechains. We describe how to build
generic communication between blockchains. As one application, we give the
construction of a \emph{two-way pegged} asset which can be moved from one
blockchain to another while retaining its nature. We provide a high-level
construction in Solidity. Our construction works across a broad range of
blockchains requiring only two underlying properties. First, that the
\emph{source} blockchain is a proof-of-work blockchain supporting
Non-Interactive Proofs of Proof-of-Work (NIPoPoWs), a cryptographic primitive
which allows constructing succinct proofs \emph{about} events which occur in a
proof-of-work blockchain and which was recently introduced in~\cite{nipopows}.
Support for NIPoPoWs can be introduced to practically any
work-based cryptocurrency such as Bitcoin and Ethereum without a hard or soft
fork. Second, that the \emph{target} blockchain is able to validate such proofs
through smart contracts such as, e.g., Ethereum or Ethereum
Classic.
We give a formal proof of security of our construction via
reduction to NIPoPoW security under the assumption that the interoperating
blockchains are secure individually.
To our knowledge, we are the first to
provide such a construction in
full and prove its security.

\todo{Move to related work section}
\noindent\textbf{Related work. }
Sidechains were introduced as a Bitcoin upgrade mechanism by Back et
al.~\cite{sidechains}. They proposed introducing a new \emph{child} blockchain
which implements a new protocol version, with which assets are \emph{2-way
pegged}. The \emph{firewall} property was articulated. No complete construction
of the protocol was given. Their paper hints at the need for ``\emph{efficient
SPV proofs}'' (Appendix B) in future work, which we implemented here. We use the
term \emph{sidechains} in a more general notion than in their work. Our
sidechains allow communication between \emph{stand alone} blockchains and also
convey \emph{any} information, not just transfers of value. In our work, a
blockchain is a sidechain of another chain if it can react to events on that
chain, and so the relationship can be symmetric.

\emph{Polkadot}~\cite{polkadot}, \emph{Tendermint},
\emph{Cosmos}~\cite{tendermint}, \emph{Liquid} and
\emph{Interledger}~\cite{interledger} also build cross-chain transfers. Their
validation relies on a trusted committees, federations or is left unspecified.
\emph{Drivechains} and \emph{rootstock} are sidechain proposals which require
miners of both chains to be aware of both networks. In our scheme, miners remain
agnostic to the existence of other chains and connect only to one network.
\emph{BTCRelay} is a trustless mechanism relaying information one-way from
Bitcoin to Ethereum, in which miners are connected to their network only.
BTCRelay requires the transmission of the entirety of the source blockchain
headers into the target blockchain. Our proposal only requires data logarithmic
in size of the source blockchain. This stems from the \emph{succinctness}
property of the NIPoPoW scheme.
None of the aforementioned constructions include
proofs of security.
Other related work includes Plasma~\cite{plasma},
XCLAIM~\cite{xclaim}, PeaceRelay, COMIT~\cite{comit}, Plasma Cash~\cite{plasmacash},
NOCUST~\cite{nocust} and Dogethereum~\cite{dogethereum}.
