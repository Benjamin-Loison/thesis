\begin{appendices}

\chapter{Mathematical Background}\label{appendix:math}
\section{Identities and Inequalities}

\subsection{The Chernoff Bound}

The following well-known theorem is due to
Rubin~\cite{chernoff1952measure,chernoff2014career}. It will help us derive
negligible bounds for probabilities of bad events.

\begin{theorem}[Chernoff Bounds]
  Let $\{X_i: i \in [n]\}$ be mutually independent Boolean random variables
  such that for all $i \in [n]$ it holds that $\Pr[X_i = 1] = p$. Let
  $X = \sum_{i = 1}^n X_i$ and $\mu = \Ex[X] = pn$. Then, for all
  $\delta \in (0, 1]$ it holds that:

  \[
    \Pr[X \leq (1 - \delta)\mu] \leq \exp(-\frac{\delta^2\mu}{2})
    \text{ and }
    \Pr[X \geq (1 + \delta)\mu] \leq \exp(-\frac{\delta^2\mu}{3})
  \]
\end{theorem}

\chapter{License}
\doclicenseFullText

\end{appendices}
