\section{Epilogue}

Computer science is a data-driven science in which optimization according to
some measurable metric or another always remains the main goal. In our case, we have
optimized the space and communication complexity of blockchain consensus
protocols, and this has given rise to important applications on top. In focusing
on a narrow optimization problem, it is often easy to forget that our work has
moral impact, and one has to keep in mind the moral character of cryptographic
work~\cite{moral}. In addition to the moral dilemmas of secrecy and transparency
faced by our predecessor cryptographers who worked on secure messaging and
digital signatures, as blockchain scientists we are facing broader ethical
questions which stem from the fact that the protocols we design have the
potential for enormous economic and political impact if they are ever to become
mainstream. When I began this thesis four years ago, I was, perhaps na\"ively,
extremely excited about the democratization that blockchain protocols can bring
to the world, from their promise to \emph{bank the unbanked} to the elimination
of the extravagant fees charged by private financial institutions.

Throughout the duration of this work, after studying and understanding the
topics in depth and develping new protocols, some of that initial excitement
faded and turned to partial disillusionment. This came especially through
numerous discussions and research conducted together with my colleague Dimitris
Karakostas and our findings on lack of blockchain
egalitarianism~\cite{egalitarianism} (which does not form part of the present
work). As a new scientist, naturally it is often easy to dismiss legacy systems
such as the existing monetary and banking system by focusing on their
shortcomings instead of their advantages which one often overlooks. Despite more
sober, I am still excited about the future that blockchains and decentralized
protocols can bring if we make good use of them. We shall keep working on them
with ethics in mind. Some big picture questions will keep coming up: Are our new
protocols better than the legacy system, and in which ways? Do they lack in
others? Most importantly, are we building systems which will form a net benefit
for humankind and the less fortunate in our society? Do they preserve or improve
upon egalitarianism and democracy, and in which ways exactly? These are not
exact science questions. While everyone's answers might be different, it is
imperative that we consider the questions and each of us makes their own
judgement. For, in solving our mathematical equations and proving our theorems,
we must not forget the real people that our work will impact.
