\section{Future Work}

As we have seen, the consensus compression primitives have numerous important
applications which can have significant impact in the space. For these to be
deployable in practice with confidence, more research is needed around the
primitives. Around the topic of proof-of-stake sidechains, the central question
that remains is whether it is possible to do so succinctly, i.e., in
$\bigO(polylog(|\chain|))$. For NIPoPoWs, more research is needed to establish
how much the model in which they achieve security can be relaxed.

We identify three major directions for future work, which we summarize below.

\textbf{Velvet NIPoPoWs.} In Chapter~\ref{chapter:work}, we discussed various
deployment strategies for NIPoPoWs and we pointed out that deployment can be
made using a hard fork, a soft fork, or a \emph{velvet fork}, and we described
some algorithmic modifications to our protocols which make velvet forking
possible. However, no proof of security was provided. In fact, velvet fork
security is more complicated than what it seems on the surface. At first
glance it seems that the adversary cannot benefit from including incorrect
pointers: She can only include either altogether \emph{wrong} pointers, or
pointers that point to the correct superblock level but are not the most recent
superblock. The first ones can be filtered out by the verifier who can check the
hash of the superblock pointer. As for the wrong ones, it seems that the
adversary can only submit pointers to older blocks in the same chain, which
would only harm their success rate. However, upon closer inspection one observes
the following interesting attack: The adversary is able to include pointers not
only to older superblocks of the correct level \emph{on the same chain}, but
also on different chains. These blocks, although they have necessary been
created prior to the block in which their pointer is included, may not be
ancestors of the block in question. This allows the adversary to play a game of
\emph{chain sewing} by ``cutting'' and ``pasting'' chunks of the honestly
generated chain into their dishonest fork. This cutting and pasting allows the
velvet adversary to ``borrow'' proof-of-work from an honest chain which does not
belong to its adversarial fork, making it potentially win when compared against
an honest fork. In short, the tree containing all interlink pointers does not
look like a tree at all in the velvet case, but it forms a Directed Acyclic
Graph. Patching the protocol and proving security under these settings is
challenging and is explored in follow-up work~\cite{velvet-nipopows}.

\textbf{NIPoPoWs under dishonest majority.} While our analysis has used the
assumption that honest majority holds for \emph{all} rounds, it is a known
result that full nodes in proof-of-work settings can withstand
temporary dishonest majority~\cite{dishonest}. More specifically, consider an
execution in which honest majority holds \emph{most} of the time, but a spike of
dishonest majority occurs for a limited number of rounds. The short period of
dishonest majority is then followed by a much longer period of honest majority.
While the ledger property of persistence can be lost for the duration of the
dishonest majority spike and for an additional period thereafter, the protocol
is \emph{self-healing} in that, given sufficient time during which honest
majority holds, the protocol can recover and persistence will be restored. This
stems from the fact that the \emph{common prefix} property of chains is
self-healing. A natural question that arises is whether NIPoPoWs are also
self-healing. More specifically, in a temporary dishonest execution, a NIPoPoW
verifier can be convinced to choose a chain that does not correspond to the
chain that a full node honest party would choose. For example, that chain could
be a short chain (such that the full node would reject it) with many superblocks
(such that the NIPoPoW verifier would accept it). However, it seems that such
proofs will become superseded by honestly generated proofs that correspond to
a chain adopted by an honest full node if sufficient time time with honest
majority is allowed to pass. A full proof of this security claim in the
temporary dishonest model would significantly increase our confidence in
superblock-based NIPoPoWs.

\textbf{Bribing-resilient NIPoPoWs.} One criticism of using superblocks to
construct NIPoPoWs has been the susceptibility of these proofs to
\emph{bribing}~\cite{flyclient}. The argument goes like this: While under the
honest majority assumption the distribution of superblocks within chains will be
as expected, the \emph{real} protocol does not have honest majority. Instead,
parties behave \emph{rationally} and would be happy to deviate from the honest
protocol if appropriate incentives are provided. The idea then is for the
adversary to \emph{bribe} miners in exchange for keeping high-level superblocks
withheld. Because high-level blocks are rare, the money needed to bribe these
miners will be small. More specifically, suppressing $\mu$-level superblocks
in expectation has a cost equal to the block reward multiplied by
$|C|2^{-\mu}$ which is the number of blocks that the adversary wishes
suppressed. In fact, the cost can be even less if not \emph{all} $\mu$
superblocks are suppressed. In the case of our \emph{charity with goodness}
construction, such bribes can only harm the succinctness of the proofs, not
their security. However, a harm in succinctness can translate to a harm in
security depending on the application. For example, a cross-chain smart contract
as discussed in Chapter~\ref{chapter:sidechains} has very limited gas
available, and proofs that are $\omega(polylog(|\chain|))$ would easily exhaust
those limits. A failure of the smart contract to receive proofs would not only
cause a denial of service, but can easily have financial cost incured by the
victims. That cost can be sufficient to provide capital to an adversary for
bribing miners. On the other hand, our \emph{charity without goodness} and our
\emph{distill} constructions directly rely on good distribution of superblocks
within the chains even for security, not just for succinctness. As such, a bribe
can directly harm security, and it is not hard to imagine a rational adversary
who makes sufficient money from their attack to be able to use some of it for
bribing purposes. To fix this issue, miner rewards have to be rebalanced
according to superblock levels. As such, a $\mu$-level superblock must receive
a reward equal to $2^\mu$ times the reward of a $0$-level block. This makes
bribing for the purposes of block suppression as costly as bribing for
suppressing the whole chain. Therefore, if the participants believe the chain
itself to be secure and not bribable, the superblocks will also survive for the
same reasons. In fact, it may be possible to allocate such rewards with a soft
fork using a smart contract beneficiary. The exact mechanics needed for reward
allocation to make NIPoPoWs bribing-resilient will be explored in future work.
