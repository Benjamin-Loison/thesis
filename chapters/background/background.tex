\chapter{Background}

This chapter gives an overview of prerequisites upon which we build our
cross-chain protocols.

\section{Notation}
We use standard mathematical notation throughout this work. We define all the
non-standard or unusual notation in this section. We use standard cryptographic
notation, which is also introduced here for reference. The reader unfamiliar
with this notation can consult some reference book in the subject such
as~\cite{katz} for a more complete treatment.

\subsection{Asymptotic probabilistic security}
The shorthand \emph{PPT} is used to denote a probabilistic polynomial-time
Turing machine. In our cryptographic model, we will model adversaries as such.
Our theorems are then by computational reduction, in which we show that bad
events happen only with negligible probability in some security parameter, which
we will denote $\lambda \in \mathbb{N}$. This parameter allows all our
cryptographic primitives to be instantiated with the required level of security;
for example, it provides the number of bits in the output of our hash function.

\begin{definition}[Negligible]
  A function $f: \mathbb{N} \longrightarrow \mathbb{R}^+$ is
  \emph{negligible} if for all $k \in \mathbb{N}$ it holds that
  $f \in \mathcal{O}(\frac{1}{\lambda^k})$.
\end{definition}

We will use the notation $\textsf{negl}$ to denote any negligible function.

\begin{definition}[Overwhelming]
  A function $f: \mathbb{N} \longrightarrow \mathbb{R}^+$ is
  \emph{overwhelming} if it can be written as $f(n) = 1 - \textsf{negl}(n)$ for
  some negligible function $\textsf{negl}$.
\end{definition}

\subsection{Sequences}
We use $[n]$ to denote the set of natural numbers from $0$ up to and including
$n$. We use $\epsilon$ to denote the empty sequence (or empty string). We write
one sequence next to another to denote string concatenation. Likewise, we
concatenate sequences to symbols by juxtaposition.

Our sequences are indexed starting at $0$. Given a sequence $\chain$, we use
$|\chain|$ to denote its length. We use Python notation to denote sequence
addressing. For $i \in [n - 1]$, we denote the $i^\text{th}$ element from the
beginning as $\chain[i]$. The first element of the sequence is thus $\chain[0]$.
For $i \in [n] \setminus \{0\}$, we denote the $i^\text{th}$ element from the
end as $\chain[-i]$. The last element of the sequence is thus $\chain[-1]$. We
call this the \emph{tip} of the sequence. Given $i \in \mathbb{Z},
j \in \mathbb{Z}$ with $i \leq j$, we denote $\chain[i:j]$ the subsequence from
$i$ (inclusive) to $j$ (exclusive), that is the sequence which contains exactly
the elements $\chain[i], \chain[i + 1], \dots, \chain[j - 1]$. If $i > j$, then
by convention we set $\chain[i:j] = \epsilon$. We allow this \emph{range}
notation to be used with negative indices as well, indicating ranges starting or
ending (or both) in indices considered from the end of the sequence, hence
allowing for $\chain[-i:j], \chain[i:-j]$, and $\chain[-i:-j]$. The left end of
a range can omitted if it is $i = 0$. The right end of a range can be omitted if
it is $j = |\chain|$. For example, $\chain[:-k]$ is the sequence $\chain$ with
the last $k$ elements excluded. In this example, if $|\chain| < k$, then
$\chain[:-k] = \epsilon$.

% TODO: \chain\{A:Z\} notation (inclusive, exclusive). But how to include Z?
\section{Probabilities}

\subsection{The Chernoff Bound}

The following well-known theorem is due to
Rubin~\cite{chernoff1952measure,chernoff2014career}. It will help us derive
negligible bounds for probabilities of bad events.

\begin{theorem}[Chernoff Bounds]
  Let $\{X_i: i \in [n]\}$ be mutually independent Boolean random variables
  such that for all $i \in [n]$ it holds that $\Pr[X_i = 1] = p$. Let
  $X = \sum_{i = 1}^n X_i$ and $\mu = \Ex[X] = pn$. Then, for all
  $\delta \in (0, 1]$ it holds that:

  \[
    \Pr[X \leq (1 - \delta)\mu] \leq \exp(-\frac{\delta^2\mu}{2})
    \text{ and }
    \Pr[X \geq (1 + \delta)\mu] \leq \exp(-\frac{\delta^2\mu}{3})
  \]
\end{theorem}

\section{Cryptographic Primitives}
\subsection{Hash Functions}
% collision resistance
% SHA256
\subsection{Signatures}
% elliptic curves, secp256k1
\subsection{Proof-of-Work}

\section{The Transaction Layer}
\subsection{Transactions}
\subsection{The UTXO Model}
\subsection{The Account Model}

\section{Authenticated Data Structures}
\subsection{Merkle Trees}
\subsection{Merkle–Patricia Tries}

\section{Blockchains}
\subsection{The Consensus Problem}
\subsection{Blocks}
\subsection{Chains of Blocks}
\subsection{Blockchain Addressing}
\subsection{SPV}

\section{Cryptocurrencies}
\subsection{Bitcoin}
\subsection{Ethereum}

\section{Smart Contracts}
\subsection{Bitcoin Script}
\subsection{Solidity}

\section{Model}
\subsection{The Random Oracle}
% Random Oracles as machines and as random functions
\subsection{Blockchain Backbone}
% Honest Majority Assumption
% Chain Growth
% Common Prefix
% Chain Quality
% The Constant Difficulty assumption and its relaxation
\subsection{The Common Reference String}
% Mention that CRS is not needed (due to Bootstrapping the Blockchain - Directly paper)
\subsection{Ouroboros}
% Epochs

% SPV?
