\chapter{Background}\label{chapter:background}

The cryptographic treatment has three core characteristics~\cite{katz}.

\begin{enumerate}
  \item \textbf{Formal definitions} play a central role. They specify the
        desirable properties of our protocols. As we will see, these can often
        be quite tricky to develop. One such example is the definition of
        \emph{pegging security} in Chapter~\ref{chapter:sidechains}.
  \item \textbf{Clearly articulated assumptions} allow us to understand the
        limitations of our protocols. Our protocols never work
        unconditionally, and we must restrict our model to obtain security. One
        such example is the computing power of the adversary. In the case of
        Chapter~\ref{chapter:work}, we can withstand a $1/3$ adversary, but the
        extended model of Chapter~\ref{chapter:variable} can only withstand a
        $1/4$ adversary.
  \item \textbf{Rigorous proofs of security} give us the \emph{guarantee} that
        our protocols are secure, as long as our assumptions hold. Instead of
        employing \emph{ad hoc} arguments, the proofs are mathematical theorems
        employing computational reductions, and they assert that the protocols
        are secure \emph{for all} adversaries.
\end{enumerate}

This chapter gives an overview of prerequisites upon which we build our
protocols. Blockchain science is a new field. As such, many of the elements we
employ here are folklore knowledge in the community, and some of them have never
been written down precisely before. Thus, this chapter may be of independent
interest as reference. Two remarkable examples are the security proof for
Merkle Trees, which to our knowledge has not been written down before, as well
as an explicit description of the static difficulty, variable difficulty,
synchronous, and $\Delta$-bounded delay environments in the form of pseudocode,
which has previously only appeared in imprecise textual descriptions.

\section{Notation}
We use standard mathematical notation throughout this work. We define all the
non-standard or unusual notation in this section. We use standard cryptographic
notation, which is also introduced here for reference. The reader unfamiliar
with this notation can consult some reference book in the subject such
as~\cite{katz} for a more complete treatment.

Given a distribution $\mathcal{M}$, we denote by $m \gets \mathcal{M}$ the experiment by which the random variable $m$ is chosen according to the distribution $\mathcal{M}$. Given a finite set $M$, we use $\uniform(M)$ to denote the uniform distribution which assigns probability $1 / 2^{|M|}$ to each element $m \in M$. We will use $m \getsrandomly M$ to denote the experiment in which $m$ is sampled from $\uniform(M)$.

As a shorthand for probabilities and to avoid excessive subscripting, we will write the experimental set up (such as the sampling of random variables) within the $\Pr[\cdot]$ prior to the predicate of interest and separated by $;\,$. For example, $\Pr[x \gets \mathcal{D}_1, y \gets \mathcal{D}_2; x + y = 1]$ denotes the experiment of independently sampling two random variables, $x$ and $y$, from the distributions $\mathcal{D}_1$ and $\mathcal{D}_2$ respectively, summing their values, and observing whether their sum is equal to $1$.

\subsection{Asymptotic probabilistic security}
Following the extended Church--Turing definition, we consider the class of
problems in $\textsc{P}$ to be \emph{easy}, and we call \emph{hard} those which
are not easy~\cite{sipser}. We will talk about \emph{honest parties}, Turing
Machines~\cite{turing} which run our code, and the \emph{adversary}, which is an arbitrary
Turing Machine that can run any code. It is assumed that all honest parties and the adversary have polynomial available time. Both the honest parties and the adversary have access to true randomness and are thus probabilistic Turing Machines.
The shorthand \emph{PPT} is used to denote a probabilistic polynomial-time
Turing machine.

Our theorems are by computational reduction, in which we show that bad
events happen only with negligible probability in some security parameter, which
we will denote $\lambda \in \mathbb{N}$. This parameter allows all our
cryptographic primitives to be instantiated with the required level of security;
for example, it provides the number of bits in the output of our hash function.

\begin{definition}[Negligible]\index{Negligible}
  A function $f: \mathbb{N} \longrightarrow \mathbb{R}^+$ is
  \emph{negligible} if for all $k \in \mathbb{N}$ it holds that
  $f \in \mathcal{O}(\frac{1}{\lambda^k})$.
\end{definition}

We will use the notation $\negl$ to denote any negligible function.

\begin{definition}[Overwhelming]\index{Overwhelming}
  A function $f: \mathbb{N} \longrightarrow \mathbb{R}^+$ is
  \emph{overwhelming} if it can be written as $f(n) = 1 - \negl(n)$ for
  some negligible function $\negl$.
\end{definition}

We will define the \emph{security} of various cryptographic protocols by making use of challenger-adversary games in which the challenger is a known Turing Machine defined by us, but the adversary is an \emph{arbitrary} Turing Machine. Herein lies the beauty of cryptography as a science; it allows us to create protocols which we prove secure against \emph{any} adversary, even those we cannot conceive. A protocol will be considered \emph{secure} if no PPT adversary can win the respective game, except with negligible probability.

\subsection{Sequences}
We use $[n]$ to denote the set of natural numbers from $0$ up to and including
$n$. We also use $[\mathcal{M}]$ to denote the support of a distribution
$\mathcal{M}$; the distinction between the two notations will be clear from
context.
We use $\epsilon$ to denote the empty sequence (or empty string). We write
one sequence next to another to denote string concatenation. Likewise, we
concatenate sequences to symbols by juxtaposition.

Our sequences are indexed starting at $0$. Given a sequence $\chain$, we use
$|\chain|$ to denote its length. We use Python notation to denote sequence
addressing. For $i \in [n - 1]$, we denote the $i^\text{th}$ element from the
beginning as $\chain[i]$. The first element of the sequence is thus $\chain[0]$.
For $i \in [n] \setminus \{0\}$, we denote the $i^\text{th}$ element from the
end as $\chain[-i]$. The last element of the sequence is thus $\chain[-1]$. We
call this the \emph{tip} of the sequence. Given $i \in \mathbb{Z},
j \in \mathbb{Z}$ with $i \leq j$, we denote $\chain[i{:}j]$ the subsequence from
$i$ (inclusive) to $j$ (exclusive), that is the sequence which contains exactly
the elements $\chain[i], \chain[i + 1], \dots, \chain[j - 1]$. If $i > j$, then
by convention we set $\chain[i{:}j] = \epsilon$. We allow this \emph{range}
notation to be used with negative indices as well, indicating ranges starting or
ending (or both) in indices considered from the end of the sequence, hence
allowing for $\chain[-i{:}j], \chain[i{:}-j]$, and $\chain[-i{:}-j]$. The left end of
a range can omitted if it is $i = 0$. The right end of a range can be omitted if
it is $j = |\chain|$. For example, $\chain[:-k]$ is the sequence $\chain$ with
the last $k$ elements excluded. In this example, if $|\chain| < k$, then
$\chain[:-k] = \epsilon$.

If an element $A$ is a member of a sequence $\chain$ we will use the notation $A
\in \chain$ to denote this, i.e. that there exist words $w, v$ such that $\chain
= wAv$. It will be clear from the context whether we are speaking about sequence
or sets. Given $A, Z \in \chain$ such that $A$ and $Z$ exist only once in
$\chain$, we denote by $\chain\{A{:}Z\}$ the subsequence of $\chain$ starting from
$A$ (inclusive) and ending in $Z$ (exclusive). If $A = \chain[0]$, it can be
omitted. Omitting $Z$ denotes the sequence starting with $A$ and containing all
subsequent elements until the end of the sequence.

% TODO: \chain\{A:Z\} notation (inclusive, exclusive). But how to include Z?


\section{Cryptographic Primitives}

We now overview the cryptographic primitives we will make use of. In particular,
cryptographically secure hash functions, public-key signatures, and
proof-of-work. This section is a review. For a full
treatment, refer to any introductory textbook in the subject~\cite{katz,handbook,foundations1,foundations2}.

\subsection{Hash Functions}\index{Hash}
\glsxtrnewsymbol[description={a cryptographically secure hash function}]{H}{$H$}\glsadd{H}
A hash function $H^s: \mathcal{M} \longrightarrow \{0, 1\}^\lambda$ is a function
parameterized by the security parameter $\lambda$ which takes any string from the distribution of input strings $\mathcal{M}$ and
outputs a string of constant size $\lambda$. To capture the fact that the hash
function behaves like a randomly chosen function, the hash
function is instantiated using a key-generating function
$\textsf{Gen}(1^\lambda)$ which generates a hash key $s$. The hash function itself is then $H^s$, a different function for each value of the key $s$. As hash functions are the building blocks and workhoses of cryptography, other protocols are designed on top of them that make use of them. We will do so in this work. In practice, the key $s$ is assumed to have been generated by the designers of the higher level protocol that makes use of the hash function and is typically fixed and publicly known. The hash protocol is the tuple $\Pi = (\textsf{Gen}, H)$.

Practical hash functions allow us to map any message $x$ of arbitrary length
$x \in \{0, 1\}^*$ to a fixed-length bitstring $\{0, 1\}^\lambda$. Hash
functions are easy to compute, but hard to invert. In applications, it is
assumed that a hash uniquely represents its preimage (it is \emph{binding}) and
that the preimage cannot be discovered from the image given sufficient entropy
(it is \emph{hiding}). This makes them ideal for constructing \emph{commitment
schemes}\index{Commitment Scheme}.

These intuitive ideas are captured by the difficulty of finding
collisions in hash functions. This is formalized in the next definition.

\begin{figure}[t]
\begin{algorithm}[H]
    \caption{\label{alg.hash-collision} The collision-finding
             experiment $\textsf{hash-collision}_{\Pi,\mathcal{A}}$.}
    \begin{algorithmic}[1]
        \Function{\sf hash-collision$_{\Pi,\mathcal{A}}$}{$\lambda$}
            \Let{s}{\textsf{Gen}(1^\lambda)}
            \Let{x_1, x_2}{\mathcal{A}(s)}
            \If{$H^s(x_1) = H^s(x_2) \land x_1 \neq x_2$}
                \State\Return{true}
            \EndIf
            \State\Return{false}
        \EndFunction
        \vskip8pt
    \end{algorithmic}
\end{algorithm}
\end{figure}


\begin{definition}[Collision resistance]\index{Collision Resistance}
  A hash function $H: \{0, 1\}^* \longrightarrow \{0, 1\}^\lambda$ is called
  \emph{collision resistant} if for all PPT adversaries $\mathcal{A}$ there is a
  negligible function $\negl$ such that

  \[
  \Pr[\textsf{hash-collision}_{\Pi,\mathcal{A}} = 1] \leq \negl(\lambda)\,.
  \]
\end{definition}

A weaker notion of security mandates that no adversary can reverse the function (pre-image resistance) or that no adversary can find a second value giving the same output as a given random value. The two cryptographic games and definitions are illustrated in Algorithms~\ref{alg.hash-preimage} and~\ref{alg.hash-second-preimage}.

\begin{figure}[t]
\begin{algorithm}[H]
    \caption{\label{alg.hash-preimage} The preimage-finding
             experiment $\textsf{hash-preimage}_{\Pi,\mathcal{A}}$.}
    \begin{algorithmic}[1]
        \Function{\sf hash-preimage$_{\Pi,\mathcal{A}}$}{$\lambda$}
            \Let{s}{\textsf{Gen}(1^\lambda)}
            \Let{x}{\mathcal{M}}
            \Let{x'}{\mathcal{A}(s, H^s(x))}
            \If{$H^s(x) = H^s(x')$}
                \State\Return{true}
            \EndIf
            \State\Return{false}
        \EndFunction
        \vskip8pt
    \end{algorithmic}
\end{algorithm}
\end{figure}

\begin{figure}[t]
\begin{algorithm}[H]
    \caption{\label{alg.hash-second-preimage} The second-preimage-finding
             experiment $\textsf{hash-second-preimage}_{\Pi,\mathcal{A}}$.}
    \begin{algorithmic}[1]
        \Function{\sf hash-second-preimage$_{\Pi,\mathcal{A}}$}{$\lambda$}
            \Let{s}{\textsf{Gen}(1^\lambda)}
            \Let{x}{\mathcal{M}}
            \Let{x'}{\mathcal{A}(s, x)}
            \If{$H^s(x) = H^s(x') \land x \neq x'$}
                \State\Return{true}
            \EndIf
            \State\Return{false}
        \EndFunction
        \vskip8pt
    \end{algorithmic}
\end{algorithm}
\end{figure}


\begin{definition}[Pre-image resistance]
  A hash function $H: \{0, 1\}^* \longrightarrow \{0, 1\}^\lambda$ is called
  \emph{pre-image resistant} if for all PPT adversaries $\mathcal{A}$ there is a
  negligible function $\negl$ such that

  \[
  \Pr[\textsf{hash-preimage}_{\Pi,\mathcal{A}} = 1] \leq \negl(\lambda)\,.
  \]
\end{definition}

\begin{definition}[Second pre-image resistance]
  A hash function $H: \{0, 1\}^* \longrightarrow \{0, 1\}^\lambda$ is called
  \emph{second pre-image resistant} if for all PPT adversaries $\mathcal{A}$
  there is a negligible function $\negl$ such that

  \[
  \Pr[\textsf{hash-second-preimage}_{\Pi,\mathcal{A}} = 1] \leq \negl(\lambda)\,.
  \]
\end{definition}

A hash function that is collision-resistant is also pre-image resistant; additionally, if it is pre-image resistant, then it must also be second pre-image resistant, as long as it provides sufficient compression~\cite{rogaway2004cryptographic}.

Protocols deployed in practice make use of fixed hash functions; that is, hash
functions with a fixed security parameter and a fixed key. In Bitcoin, the hash
function $\SHA$~\cite{sha256} is used for both commitments and proof-of-work.
Its domain and range are
$\SHA: \{0, 1\}^* \longrightarrow \{0, 1\}^{256}$. In Ethereum, the hash
function $\keccak$~\cite{bertoni2008indifferentiability}, a variant of
$\texttt{SHA3}$, is used for commitments. Its domain and range are $\keccak: \{0, 1\}^*
\longrightarrow \{0, 1\}^{256}$. The function used for proof-of-work is a
variant of this.

\subsection{Signatures}
A \emph{digital signature} allows parties to authenticate the origin of a
message as well as its integrity~\cite{katz}. If Alice signs a message $m$, she generates a
signature $\sigma$ which is uniquely associated with that message. That
signature can then only be used to verify that particular message. In a secure signature scheme, an adversary
cannot \emph{forge} signatures that correctly verify for messages that have not
been signed by the honest party.

Signing and verification are two separate tasks which are asymmetric. Only the
authorized party can sign a message, but anyone can verify the signature. This
is achieved by having each party generate their own \emph{public-private
key pair} $(pk, sk)$ in which $pk$ is the public key and $sk$ is the secret key.
Signatures are then generated using the secret (or signing) key $sk$ and
verified using the public (or verification) key $sk$. A key pair is generated
using the polynomial-time key generation algorithm $(pk, sk) \gets
\Gen(1^\lambda)$. A signature is generated by invoking the polynomial-time
signing algorithm $\sigma = \Sig(sk, m)$. Verification is done by checking
whether the verification algorithm $\Ver(pk, m, \sigma)$ returns $\true$ or
$\false$. The signature scheme $\Pi$ then is defined as the tuple
$\Pi = (\Gen, \Sig, \Ver)$.

Signature schemes must be \emph{correct}.

\begin{definition}[Signature correctness]
  A signature scheme is \emph{correct} if there is a
  negligible function $\negl$ such that for all messages $m \in \{0, 1\}^*$ it
  holds that

  \[
    \Pr[(pk, sk) \gets \Gen(1^\lambda); \Ver(pk, m, \Sig(sk, m)) = \false] < \negl(\lambda)\,.
  \]
\end{definition}

A \emph{secure} signature scheme requires that no adversary is able to forge
signatures. This is captured in the game-based definition of Algorithm~\ref{alg.forgery}. In this game, the challenger first generates a public/private key pair by invoking $\Gen(1^\lambda)$. Subsequently, the challenger asks the adversary $\mathcal{A}$ to attempt to find a signature forgery. The adversary is allowed to ask the challenger to have any messages signed by invoking the ${\Sig}O$ oracle with messages of her choice. The adversary is allowed to make multiple adaptive queries to the oracle. When the adversary is ready, she presents a message $m$, which she must not have requested from the oracle ${\Sig}O$ and a signature $\sigma$. The adversary is successful if the signature verifies.

\begin{figure}[t]
\begin{algorithm}[H]
    \caption{\label{alg.forgery} The forgery
             experiment $\textsf{sig-forge}_{\Pi,\mathcal{A}}$.}
    \begin{algorithmic}[1]
        \Function{\sf sig-forge$_{\Pi,\mathcal{A}}$}{$\lambda$}
            \Let{(pk, sk)}{\textsf{Gen}(1^\lambda)}
            \Let{(m, \sigma)}{\mathcal{A}^{Sig_{sk}}(pk)}
            \If{$Ver(pk, m, \sigma)$}
                \State\Return{true}
            \EndIf
            \State\Return{false}
        \EndFunction
        \vskip8pt
    \end{algorithmic}
\end{algorithm}
\end{figure}


\begin{definition}[Security]
  A signature scheme $\Pi = (\Gen, \Sig, \Ver)$ is \emph{secure} if for all PPT adversaries $\mathcal{A}$ there is a negligible function $\negl$ such that

  \[
  \Pr[\textsc{sig-forge}_{\Pi,\mathcal{A}}(\lambda)] < \negl(\lambda)\,.
  \]
\end{definition}

There are multiple ways to construct a secure signature scheme. Our signature schemes of interest make use of the \emph{discrete logarithm} problem in a group. In such a construction, a cyclic group $\mathbb{G}$ with order close to $2^\lambda$ and a generator $G \in \mathbb{G}$ are fixed initially. A public key corresponds to an element $A \in G$ of the group, while the corresponding private key $a \in \mathbb{Z}_{|\mathbb{G}|}$ is the order of $A$ with respect to $G$, that is $A = aG$. The keys are generated by first choosing a private key $a$ uniformly at random and then computing its corresponding public key. The public key can be computed quickly from the private key using multiplication by doubling~\cite{shoup}, but it is believed that the inverse problem is hard.

The problem of finding $a$ from $A$ is made formal in Algorithm~\ref{alg.dlog}. Here, we assume that an efficient algorithm $\mathcal{G}$ can be used to pick a suitable group of the appropriate order and output its description. Furthermore, we assume the group operator is efficiently computable. The challenger generates a group and chooses one of its elements at random. The adversary is then asked to find the \emph{discrete logarithm} of that element.

\begin{figure}[t]
\begin{algorithm}[H]
    \caption{\label{alg.dlog} The \textsc{DLOG} problem.}
    \begin{algorithmic}[1]
        \Function{\sf DLOG$_{\mathcal{G},\mathcal{A}}$}{$\lambda$}
            \Let{(\mathbb{G}, G, |\mathbb{G}|)}{\mathcal{G}(1^\lambda)}
            \State$a \getsrandomly \mathbb{Z}_{|\mathbb{G}|}$
            \Let{A}{aG}
            \Let{a^*}{\mathcal{A}(\mathbb{G}, G, |\mathbb{G}|, A)}
            \State\Return{$a^*G = A$}
        \EndFunction
        \vskip8pt
    \end{algorithmic}
\end{algorithm}
\end{figure}


\begin{definition}[Discrete Logarithm Problem]
  The \emph{discrete logarithm problem} is hard in the family of groups $\{\mathcal{G}(1^\lambda)\}_{\lambda \in \mathbb{N}}$ if for all PPT adversaries $\mathcal{A}$ there is a negligible function $\negl$ such that

  \[
    \Pr[\textsf{DLOG}_{\mathcal{G},\mathcal{A}}(\lambda) = 1] < \negl(\lambda)\,.
  \]
\end{definition}

The particular instantiation of signature schemes in the context of
cryptocurrencies makes use of elliptic curves~\cite{ec} in which the discrete logarithm problem is believed to be hard. More specifically, Bitcoin
and Ethereum both use the \texttt{secp256k1} curve~\cite{secp256k1}.

\begin{remark}
We remark that, perhaps contrary to popular belief, blockchain protocols
do not depend at all on encryption primitives. Therefore, we choose not to
treat encryption at all in the present work.
\end{remark}


\todo{Copy lecture notes from our \emph{Introduction to Blockchains} course for the next sections...}

\section{Authenticated Data Structures}
\subsection{Merkle Trees}
Consider a set $S = \{s_1, s_2, \cdots, s_n\}$ of strings
$s_i \in \{0, 1\}^*$. At some initial time, this set is compressed into a
\emph{root} string $s$ which is short ($|s| = \lambda$). This compressed string
is produced honestly and is given to a party called the \emph{verifier}. Given
this short trusted root string, the verifier receives claims from untrusted
\emph{provers} which claim that a certain piece of data $s_i$ existed in $S$.
The verifier's job is to decide whether such claims are truthful or fraudulent.

This protocol is an \emph{authenticated data structure}. It consists of
four algorithms $\mathcal{G}$, $\textsc{compress}$, $\textsc{prove}$,
$\textsc{verify}$. At the beginning of the execution, $\mathcal{G}(1^\lambda)$
is invoked to initialize the protocol parameters. These parameters can be shared
among multiple invocations of the protocol. As these parameters are fixed by the
protocol in its concrete implementations, we will make them implicit from now
on. A set $S$ is compressed by invoking $\textsc{compress}(S)$
which produces the root $s$. When an honest prover wishes to prove that some
$s_i$ exists in $S$, they produce an \emph{inclusion proof} $\pi =
\textsc{prove}(S, s_i)$. When the verifier receives an element $s_i$ together
with a proof of inclusion $\pi$, they check its veracity by invoking
$\textsc{verify}(\pi, s_i, s)$, which returns $\true$ or $\false$.

Authenticated data structure protocols must be correct. This means that honest
executions should always work.

\begin{definition}[Correctness]
  Consider an authenticated data structure protocol
  $\Pi = (\textsc{compress}, \textsc{prove}, \textsc{verify})$.
  We say that $\Pi$ is
  \emph{correct} if

  \[\forall S:
    \forall s_i \in S:
    \textsc{verify}(\textsc{prove}(S, s_i), s_i, \textsc{compress}(S))\,.\]
\end{definition}

Such protocols are useful when $s$ and $\pi$ are short.

\begin{definition}[Succinctness]
  Consider an authenticated data structure protocol
  $\Pi = (\textsc{compress}, \textsc{prove}, \textsc{verify})$.
  We say that $\Pi$ is
  \emph{succinct} if
  for all $S$ it holds that

  \[|\textsc{compress}(S)| \in \bigO(polylog(|S|))
    \land
    \forall s_i \in S:
    |\textsc{prove}(S, s_i)| \in \bigO(polylog(|S|))
    \,.\]
\end{definition}

In the protocols we will explore, we will have $|s| = \lambda \in \bigO(1)$ and
$\pi \in \bigO(\log(|S|))$. Furthermore, $|S|$ will be polynomial in $\lambda$.

An authenticated data structure protocol is \emph{secure} if no adversary can
convince a verifier about the inclusion of an element which is not in the
set. This is made formal in the game illustrated in
Algorithm~\ref{alg.authenticated}.

\begin{figure}[t]
\begin{algorithm}[H]
    \caption{\label{alg.authenticated} The authenticated data structure challenger.}
    \begin{algorithmic}[1]
        \Function{\sf AUTH$_{\Pi,\mathcal{A}}$}{$\lambda$}
            \Let{(S, s_i, \pi)}{\mathcal{A}(1^\lambda)}
            \State\Return{$s_i \not\in S \land \textsc{verify}(\textsc{compress}(S), s_i, \pi)$}
        \EndFunction
        \vskip8pt
    \end{algorithmic}
\end{algorithm}
\end{figure}


\begin{definition}[Security]
  An authenticated data structure protocol $\Pi = (\textsc{compress},\allowbreak \textsc{prove},\allowbreak \textsc{verify})$ is \emph{secure} if for all PPT adversaries $\mathcal{A}$
  there is a negligible function $\negl$ such that

  \[
    \Pr[\textsf{AUTH}_{\Pi,\mathcal{A}}(\lambda)] < negl(\lambda)\,.
  \]
\end{definition}

A construction that solves this problem which is used extensively in blockchain
protocols is the \emph{Merkle Tree}~\cite{merkle}. This construction is
illustrated in Algorithm~\ref{alg.merkle}. It is parameterized by a collision
resistant hash funciton $H$. The construction presented works for $|S|$ equal to
a power of $2$ and assumes the domain of values in $S$ is disjoint from the
domain of $H$.

\begin{algorithm}[H]
    \caption{\label{alg.merkle} The Merkle Tree construction for $|S| = 2^k$ for
                                some $k$.}
    \begin{algorithmic}[1]
        \Function{\sf heapify$^H$}{$S$}
            \Let{Z[|S|{:}]}{\textsf{map}(H, S)}
            \For{$i \gets |S| - 1 \text{ down to } 1$}
                \Let{Z[i]}{H(Z[2i] \concat Z[2i + 1])}
            \EndFor
            \State\Return{$Z$}
        \EndFunction
        \Function{\sf compress$^H$}{$S$}
            \State\Return{$\textsf{heapify}^H(S)$[1]}
        \EndFunction
        \Function{\sf prove$^H$}{$S, i$}
            \Let{Z}{\textsf{heapify}^H(S)}
            \Let{i}{|S| + i}
            \Let{\pi}{[\,]}
            \While{$i > 1$}
                \Let{b}{i \mod 2}
                \Let{\pi}{\pi \concat (b, Z[i \xor 1])}
                \Let{i}{\lfloor i / 2\rfloor}
            \EndWhile
            \State\Return$\pi$
        \EndFunction
        \Function{\sf verify$^H$}{$\pi, s_i, s$}
            \Let{s_i}{H(s_i)}
            \For{$(b, h) \in \pi$}
                \If{$b$}
                    \Let{s_i}{H(s_i \concat h)}
                \Else
                    \Let{s_i}{H(h \concat s_i)}
                \EndIf
            \EndFor
            \State\Return{$s_i = s$}
        \EndFunction
        \vskip8pt
    \end{algorithmic}
\end{algorithm}


It treats $S$ as a sequence and organizes it into a complete binary tree $Z$
using the $\textsf{heapify}$ routine. The routine places the hashes of the
elements of $S$ on the leafs of $Z$ stored at locations $Z[|S|{:}]$. The value
of each internal node is the hash of the concatenation of the values of its
children. The $\textsf{compress}$ function returns the value of the root which
resides at $Z[1]$. To create a proof $\pi$, the $\textsf{prove}$ routine takes
an index of an element $i$ and finds its position in the binary tree, namely the
leaf stored at $Z[|S| + i]$. It then traverses the path from that leaf up to the
root, maintaining the index of the current node in the variable $i$. In every
iteration, it includes a bit indicating whether the current node is a left child
($b = 0$) or a right child ($b = 1$). For each node, it includes the value $Z[i
\xor 1]$ of the node's sibling. To verify a proof, the verifier successively
hashes the element whose inclusion is proven with the hashes $h$ of the siblings
provided in the proof $\pi$ on the correct side indicated by $b$. In the end, it
checks whether it has arrived at the trusted root $s$ and this determines the
result of the verification.

Correctness is achieved because the hash function is deterministic and the
\textsc{verify} applies hashes in the same manner as \textsc{heapify} does.
Succinctness follows because $|s| = \lambda$. Furthermore, the tree contains
$2|S| - 1$ elements and the height of the tree is $\Theta(\log(|S|))$, making
$|\pi| \in \Theta(\log(|S|))$. Lastly, security follows by a direct
computational reduction from the collision resistance of $H$.


\section{Model}
\subsection{The Random Oracle}\index{Random Oracle}

Real protocols are instantiated using real hash functions. However, as concrete
mathematical objects, these are not possible to analyze cryptographically, as
they do not have a security parameter. For example, \texttt{SHA256} is a
function with a fixed size of $256$ bits. In these terms, one cannot talk about
negligible probability of failure. Keyed hash functions in which the function
depends on the security parameter are possible to analyze in this manner, but
their use is limited in practice. Additionally, many of the guarantees provided
by keyed hash functions discussed above are insufficient for more elaborate
protocols. In particular, collision and preimage resistance are not enough for
our needs. Nevertheless, our intuition is that practical concrete hash functions
behave nicely, and we wish to include this notion in our model.

To bypass these limitations, we model our hash functions in the Random Oracle
model~\cite{ro}. In this model, the hash function behaves like an ideal random
function. This is helpful, because we can argue all of its outputs will be
unbiased and independent. This abstraction also signifies that we are interested
in working in the realm of \emph{protocol design}, and the intricacies of
practical hash function design and implementation, which stands in the realm of
efficient symmetric cryptography, are beyond the scope of our work.

In the Random Oracle model, we assume the existence of a global oracle machine
$H$ to which every party, adversarial or honest, has access to. The machine
models the hash function and hence allows parties to ask for its evaluation at
any input. The Random Oracle machine receives any input in $\{0, 1\}^*$ and
returns an output in $\{0, 1\}^\lambda$.

\begin{figure}[ht]
\begin{algorithm}[H]
    \caption{\label{alg.ro} The Random Oracle model parameterized by security
             parameter $\kappa$.}
    \begin{algorithmic}[1]
        \Let{T}{\emptyset}
        \Function{\sf H$_\kappa$}{$x$}
            \If{$x \not\in T$}
                \State$T[x] \getsrandomly \{0, 1\}^\kappa$
            \EndIf
            \State\Return$T[x]$
        \EndFunction
        \vskip8pt
    \end{algorithmic}
\end{algorithm}
\end{figure}


The output is chosen as illustrated in Algorithm~\ref{alg.ro}. In detail, the
Random Oracle is parameterized by the security parameter $\lambda$. Upon
receiving some input $x$, if the input has not been encountered before, then it
produces a fresh uniformly random $\lambda$-bit string which it stores in a
dictionary $T$ and returns. If on the other hand it receives an input it has
seen before, it answers consistently with its previous answer, giving the same
answer for the same query by consulting the dictionary $T$. Hence, this
functionality is stateful.

It is imperative that the instance of the machine that all parties communicate
with is the same. Hence, if a party makes a particular query $x$ to the oracle
and then another party asks the same query, the answer will be the same. If the
random oracle is modelled as a stateful functionality, communication between the
parties and the Random Oracle machine can be modelled as Interactive Turing
Machines communicating.

Alterantively but equivalently, the Random Oracle can be defined as a shared
oracle which answers queries according to a function selected uniformly at
random at the beginning of the execution. As a random function can neither be
sampled in polynomial time nor represented in polynomial space, this latter
formulation means that, when the oracle is treated that way, it cannot be
modelled as an Interactive Turing Machine, but must remain an oracle. In this
formulation, at the beginning of the execution, the random oracle $H$ is sampled
uniformly at random from the function space $(\{0, 1\}^\lambda)^{\{0, 1\}^*}$,
and the adversary and honest parties are invoked with oracle access to this same
$H$. As all parties will be polynomial in our treatment, this sampling can also
be understood to be done uniformly at random from the function space $(\{0,
1\}^\lambda)^{\{0, 1\}^{p(\kappa)}}$ where $p(\kappa)$ denotes the polynomial
bounding the total execution of all parties together.

An important feature of the Random Oracle model is that the adversary cannot
compute $H$ locally, but must invokve the oracle to do so. This means that, in
our mathematical treatment, we are allowed to speak of the queries the adversary
has made, count them, look at their responses, and so on. This ability, termed
\emph{random oracle observability} will be critical in our analyses. On the
other hand, we will not make use of the ability of a computational reduction to
modify the outputs of the random oracle at will, termed \emph{random oracle
programmability}, which is at the heart of many security proofs in cryptography.
Our results are therefore stronger, as we only assume a \emph{non-programmable}
Random Oracle~\cite{nielsen2002separating,fischlin2010random}, which is a weaker
assumption than usually made.

\subsection{The Environment}\label{sec:env}

We will begin with a simple model and make it successively more nuanced until it
is sufficiently sophisticated to satisfactorily capture the real world. First,
we describe the simple environment in which the network is \emph{synchronous}
and the execution is with \emph{static difficulty}. We then relax the synchrony
assumption by introducing a $\Delta$-bounded delay network. Subsequently, we
relax the static assumption by introducing executions of
\emph{variable difficulty} in which populations can be adversarially evolved.
Last, we give the adversary even more power by allowing her to adaptively
corrupt parties of her choice. We now introduce these models in order.

In our setting, we will study executions of protocols in which some parties
are \emph{honest} while others are \emph{adversarial}. All the honest parties
typically run the same code termed \emph{the honest protocol} (which is the
protocol we will design), while the adversary can run any code she wishes, but
is bounded by polynomial time bounds. Both the honest parties and the adversary
are probabilistic Turing Machines. As we are working in distributed settings,
our protocols will be long-lived and involve multiple parties running
simultaneously and communicating over the network while maintaining local state.
Among the parties in our execution, we will denote by $n$\index{$n$} the total number of
parties and with $t$\index{$t$} the number of parties that are adversarial. To strengthen
our adversary, we assume all the adversarial parties \emph{collude} and are
controlled by a single adversary. The situation where multiple adversaries are
not colluding is also captured by our stronger model (this can be captured by a
single adversary which simulates the multiple non-colluding adversaries).

To model the distributed setting, we must speak of executions concretely.
Towards this purpose, we conjure an \emph{environment} $\mathcal{Z}$
\index{Environment}\index{$\mathcal{Z}$} which is an
Interactive Turing Machine~\cite{interactive-tm} (ITMs) \index{Interactive
Turing Machine} and is responsible for
orchestrating the whole execution. An Interactive Turing Machine is a Turing
Machine which models interactive computation by employing additional input and
output tapes that can be written to by external machines. The machine can decide
to pause computation by entering a special state and its computation is resumed
by writing to its input tape.

The environment spawns $n - t$ honest parties running the honest protocol $\Pi$ as
$n$ different Interactive Turing Machines. The environment also spawns one
adversarial Interactive Turing Machine $\mathcal{A}$. The honest parties and the
adversary can pass messages to the environment and receive messages from it by
writing and/or reading from their interactive tapes. The environment takes as
input the security parameter $1^\lambda$ and functions as an operating system
scheduler to activate the honest parties and the adversary according to some
schedule. The environment halts after polynomial time. The Interactive Turing
Machine model is equivalent to having the environment faithfully simulate the
execution of the honest parties and the adversary and correctly maintaining
their state across pausing and resumption.

We study an execution by observing its transcript (the messages exchanged by the
parties) as well as the internal state of the parties throughout the execution.
This transcript, which we will denote $\view_{\Pi,\mathcal{A}}^{n,t}$, is a
random variable which is a function of the coins of the probabilistic
Interactive Turing Machines that form the execution, namely the environment
itself, the adversary, the honest parties, and the Random Oracle. We remark that
this treatment is similar to the setting of Universal Composability~\cite{uc}.
Despite the similarities on the surface, we do not fully employ it and neither
are our protocols composable, nor are our security proofs simulation-based. On
the contrary, we use a direct property-based approach in our proofs instead of
employing universally composable functionalities.

A skeleton for the environment is illustrated in Algorithm~\ref{alg.environment}.

\begin{figure}[t]
\begin{algorithm}[H]
    \caption{\label{alg.environment} The environment and network model running
             for a polynomial number of rounds $p(\lambda)$.}
    \begin{algorithmic}[1]
        \Function{$\mathcal{Z}^{n,t}_{\Pi,\mathcal{A}}$}{$1^\lambda$}
            \For{$i \gets 1 \text{ to } n - t$}
                \Comment{Boot honest ITMs}
                \Let{P_i}{\Pi.\textsf{init}(1^\lambda)}
            \EndFor
            \Let{A}{\mathcal{A}.\textsf{init}(1^\lambda, n, t)}
            \Comment{Boot adversarial ITM}
            \Let{\overline{C}}{[\,]}
            \For{$i \gets 1 \text{ to } n - t$}
                \Let{\overline{C}[i]}{[\,]}
            \EndFor
            \For{$r \gets 1 \text{ to } p(\lambda)$}
                \Let{C}{\emptyset}
                \For{$i \gets 1 \text{ to } n - t$}
                    \Let{C}{C \cup \{P_i.\textsf{execute}(\overline{C}[i])\}}
                    \Comment{Execute honest party $i$ for round $r$}
                \EndFor
                \Let{\overline{C}}{A.\textsf{execute}(C)}
                \Comment{Execute rushing adversary for round $r$}
                \For{$c \in C$}\Comment{Ensure all parties will receive message $c$}
                    \For{$i \gets 1 \text{ to } n - t$}
                        \label{alg.environment.connectivity}
                        \State{$\textsf{assert}(c \in \overline{C}[i])$}
                    \EndFor
                \EndFor
            \EndFor
        \EndFunction
        \vskip8pt
    \end{algorithmic}
\end{algorithm}
\end{figure}


At the beginning of the execution, all the ITMs are booted by invoking their
constructors. The environment spins up $n - t$ honest machines that run
the protocol $\Pi$ and one adversarial machine that runs the protocol
$\mathcal{A}$ and represents the $t$ adversarial parties. These machines are
stateful, and so we denote the respective ITM (which can be paused and resumed)
by $P_i$ for the honest parties (running protocol $\Pi$) and by $A$ for the
adversary (running protocol $\mathcal{A}$). The machines are given time
polynomial in $\lambda$ by invoking their constructors with the parameter
$1^\lambda$. During construction, the adversary
learns of the number of honest parties $n - t$ and adversarial parties $t$.
Importantly, the honest parties do not have this privilege. We
call this setting \emph{permissionless setting}\index{Permissionless} (also
known as the \emph{anonymous byzantine} or \emph{open setting}), because honest
parties are not informed of each others' identities nor their count. Contrary to
our treatment throughout this thesis, there also exists a
\emph{permissioned}~\cite{consensus-sok} setting in the blockchain literature.
In that setting, the $n$ nodes are given authenticated channels between each
other and the quantity $n$ is known to all parties. We will not make this
assumption here. The fact that we are working in the permissionless setting
gives rise to the \emph{decentralized}\index{Decentralized} title of this
thesis.

Time is quantized into discrete \emph{rounds}~\cite{backbone}\index{Round}\index{Slot} (or \emph{slots}~\cite{ouroboros}) numbered $r = 1, 2, 3, \cdots$\index{r}.
The
environment contains a main loop which executes one iteration per round $r$ for
a total polynomial number of rounds $p(\lambda)$. During
every round, it first activates every honest party $P_i$ by invoking its
\textsf{execute} method. Subsequently, at the end of
the round, it actives the adversary $A$. The fact that the adversary is activated at
the end of every round is an advantage for the adversary. We call such an
adversary a \emph{rushing adversary}\index{Rushing Adversary}, because it can use its computational power
for the round after it has observed what the honest parties have done during the
same round. Both the honest party and the adversary can know the index of the
current round by counting how many times they have been activated so far in
their persistent state. Because both the honest parties and the
adversary are PPT machines, they will run for polynomial time every time they
are activated. Additionally, we assume $n$ is polynomial, thus ensuring the
total execution time is polynomial.

\subsection{The Network}

When an honest party is activated, it is given messages from the network to
\emph{read}, which are written to a special location within its input tape by
the environment. Here, we denote the network messages received by party $i$ as
$\overline{C}[i]$ and pass them as input to the \textsf{execute} method. The
party can then \emph{write} messages to the network during the round, which we
denote by the \textsf{execute} method returning a value. We say that such
messages are \emph{diffused}\index{Diffuse} to the network and we will use
the \textsf{Diffuse} notation within the implementation of honest protocols to
signify that a message needs to be diffused to the rest of the parties. At the end of the
round, the adversary can see all the messages $C$ that have been diffused by the
honest parties during the same round. The adversary can then decide what will
appear in the network tape of every honest party at the beginning of the next
round by outputting an array $\overline{C}$ that contains a list of messages
$\overline{C}[i]$ for every party $i$. The adversary can reorder the messages and
insert as many of her own as she wishes. That is, it is possible that
$\overline{C}[i]$ will contain more messages than $\overline{C}$ and that the
messages in $\overline{C}[i]$ will appear in a different order than in
$\overline{C}$. As such, communication is not authenticated.

However, she must ensure that all messages diffused by any honest party
during the previous round appear in the network tape of every other honest party
at the beginning of the next round. This is ensured by the assertion in Line~\ref{alg.environment.connectivity}. This means that no messages can be dropped
by the adversary. This \emph{connectivity} assumption is equivalent to assuming
that each honest party is not \emph{eclipsed}~\cite{eclipse,eclipse-ethereum}\index{Eclipse}
from the rest of the network. In practice, this is achieved by ensuring that
every honest party is connected to every other honest party through some path,
although not necessarily directly. Practical peer-to-peer protocols use
gossiping~\cite{gossip}\index{Gossip} to ensure messages reach every participant of the network. The network
model abstracts out such details and treats a round as the unit time which is
needed for a message to reach from every honest party to every other honest
party.

Crucially, because the adversary can reorder messages and inject as many
additional messages as she pleases, she is a
\emph{sybil adversary}~\cite{sybil}\index{Sybil Attack}. This means that the adversary can fake
multiple identities and pretend to produce messages by multiple parties,
potentially more than $t$. It will be the job of our honest protocol to produce
a \emph{sybil resilient} mechanism in which such attacks have no impact on the
protocol's security. Furthermore, the adversary can \emph{split} the view of the
honest parties because she can communicate different messages to different
honest parties and have $C[i] \neq C[j]$ for $i \neq j$ on the same round. For example, the order in which messages are delivered can be
different for every honest party and the adversary may inject different messages
of her own for every honest party.

The requirement that messages diffused at the end of one round are delivered at
the beginning of the next is the \emph{synchronous model}\index{Synchronous}. A
large part of our analysis will be made there.

In addition to the details specified in the environment of
Algorithm~\ref{alg.environment}, we allow the honest parties to receive
auxiliary \emph{input}, distinguished from the network tape. This input is
adversarially chosen and can influence the decisions of the honest parties. Once
these notions have been defined, such input will correspond to
\emph{transactions} that the honest parties will wish to include in their
blockchains.

\begin{figure}[!ht]
\begin{algorithm}[H]
    \caption{\label{alg.delta-environment} The environment and network model in
             the $\Delta$-bounded delay setting.}
    \begin{algorithmic}[1]
        \Function{$\mathcal{Z}^{n,t}_{\Pi,\mathcal{A}}$}{$1^\lambda$}
            \For{$i \gets 1 \text{ to } n - t$}
                \Comment{Boot honest ITMs}
                \Let{P_i}{\textsf{new } \Pi(1^\lambda)}
            \EndFor
            \Let{A}{\textsf{new } \mathcal{A}(1^\lambda, n, t)}
            \Comment{Boot adversarial ITM}
            \Let{\overline{C}}{[\,]}
            \For{$i \gets 1 \text{ to } n - t$}
                \Let{\overline{C}[i]}{[\,]}
            \EndFor
            \Let{\textsf{seen}}{[\,]}
            \Let{\textsf{diffused}}{[\,]}
            \For{$r \gets 1 \text{ to } p(\lambda)$}
                \Let{C}{\emptyset}
                \For{$i \gets 1 \text{ to } n - t$}
                    \Let{\textsf{seen}[i]}{\textsf{seen}[i] \cup \overline{C}[i]}
                    \Let{C}{C \cup \{P_i.\textsf{execute}(\overline{C}[i])\}}
                    \Comment{Execute honest party $i$ for round $r$}
                \EndFor
                \Let{\textsf{diffused}[r]}{C}
                \Let{\overline{C}}{A.\textsf{execute}(C)}
                \Comment{Execute rushing adversary for round $r$}
                \For{$c \in \bigcup_{1 \leq r' \leq r - \Delta}\textsf{diffused}[r']$}\Comment{Ensure $\Delta$-delay}
                    \For{$i \gets 1 \text{ to } n - t$}
                        \State{$\textsf{assert}(c \in \textsf{seen}[i])$}
                    \EndFor
                \EndFor
            \EndFor
        \EndFunction
        \vskip8pt
    \end{algorithmic}
\end{algorithm}
\end{figure}


A relaxation of the synchronous model is the $\Delta$-bounded
delay\index{$\Delta$-bounded Delay} model and is illustrated in Algorithm~\ref{alg.delta-environment}. In this model, the adversary may
delay messages up to $\Delta$ rounds before finally delivering them. Any message
diffused by any honest party at round $r$ must appear in the network tapes of
all other honest parties prior to round $r + \Delta$. This $\Delta$ is unknown
to the honest parties, although the security of the protocol requires that
$\Delta$, together with other protocol parameters, satisfies a number of
conditions. As such, this model stands between the \emph{synchronous} setting
(where $\Delta$ is known by all honest parties beforehand or, equivalently,
$\Delta = 1$) and the \emph{semi-synchronous} setting (where $\Delta$ is
completely unknown) in that, while $\Delta$ itself is unknown, it is governed by
equations which express a trade-off between $\Delta$ and other quantities, and
these equations are known.

Chapters~\ref{chapter:work},
\ref{chapter:stake}, \ref{chapter:superlight}, \ref{chapter:sidechains} are
explored in the synchronous model. We extend our protocol to the $\Delta$-bounded
delay model in Chapter~\ref{chapter:variable}.

\subsection{Evolving Population}
So far, we have defined the environment for executions in which $n$ and $t$ are
fixed throughout the execution. We call this setting the
\emph{static difficulty}\index{Static Difficulty}~\cite{backbone,backbone-new}
setting. The model can be relaxed to allow the population to evolve with time.
This gives rise to the
\emph{variable difficulty}\index{Variable Difficulty}~\cite{varbackbone}
model.
Instead of two fixed values $n$ and $t$, we instead consider two
\emph{sequences} of values
$\{n_r\}_{r \in [p(\lambda)]}$ and $\{t_r\}_{r \in [p(\lambda)]}$.
At round $r$, when $n_r - t_r < n_{r-1} - t_{r-1}$, the number of honest
parties has decreased and $(n_{r-1} - t_{r-1}) - (n_r - t_r)$ honest ITM
instances are killed by the environment. The choice of which instances will be
killed is made by the adversary. When
$n_r - t_r > n_{r-1} - t_{r-1}$, the number of honest parties has increased and
$(n_r - t_r) - (n_{r-1} - t_{r-1})$ new honest instances are spawned up by
cloning the state of some existing honest instances. The choice of which
instances to clone is made by the adversary. An increasing or decreasing $t_r$
does not affect the spawned instances. Finally, the choice of how $n_r$
and $t_r$ evolve is made \emph{adaptively} by the adversary~\cite{varbackbone-new}
based on the execution so far.

\begin{figure}[!ht]
\begin{algorithm}[H]
    \caption{\label{alg.var-environment} The \emph{variable difficulty}
             environment in the $\Delta$-bounded delay setting.}
    \begin{algorithmic}[1]
        \Function{$\mathcal{Z}_{\Pi,\mathcal{A}}$}{$1^\lambda$}
            \Comment{Boot adversarial ITM}
            \Let{A}{\textsf{new } \mathcal{A}(1^\lambda)}
            \Let{(n_1, t_1)}{A.\textsf{init}()}
            \Let{I}{\emptyset}
            \For{$i \gets 1 \text{ to } n_1 - t_1$}
                \Comment{Boot honest ITMs}
                \Let{P_i}{\textsf{new } \Pi(1^\lambda)}
                \Let{I}{I \cup \{i\}}
            \EndFor
            \Let{\overline{C}}{[\,]}
            \For{$i \in I$}
                \Let{\overline{C}[i]}{[\,]}
            \EndFor
            \Let{\textsf{seen}}{[\,]}
            \Let{\textsf{diffused}}{[\,]}
            \For{$r \gets 1 \text{ to } p(\lambda)$}
                \Let{C}{\emptyset}
                \For{$i \in I$}
                    \Let{\textsf{seen}[i]}{\textsf{seen}[i] \cup \overline{C}[i]}
                    \Let{C}{C \cup \{P_i.\textsf{execute}(\overline{C}[i])\}}
                    \Comment{Execute honest party $i$ for round $r$}
                \EndFor
                \Let{\textsf{diffused}[r]}{C}
                \Let{(\overline{C}, \textsf{kill}, \textsf{spawn}, t_r)}{A.\textsf{execute}(C)}
                \For{$(i, j) \in \textsf{spawn}$}\Comment{Spawn new honest parties as clones of old ones}
                    \State{$\textsf{assert}(P_j \neq \bot)$}
                    \Let{P_i}{P_j}
                    \Let{\textsf{seen}[i]}{\textsf{seen}[j]}
                    \Let{I}{I \cup \{i\}}
                \EndFor
                \For{$i \in \textsf{kill}$}\Comment{Kill honest parties of the adversary's choice}
                    \Let{P_i}{\bot}
                    \Let{I}{I \setminus \{i\}}
                \EndFor
                \Let{n_r}{|I| + t_r}
                \For{$c \in \bigcup_{1 \leq r' \leq r - \Delta}\textsf{diffused}[r']$}\Comment{Ensure $\Delta$-delay}
                    \For{$i \in I$}
                        \State{$\textsf{assert}(c \in \textsf{seen}[i])$}
                    \EndFor
                \EndFor
            \EndFor
        \EndFunction
        \vskip8pt
    \end{algorithmic}
\end{algorithm}
\end{figure}


In the variable difficulty setting, we will concern ourselves with protocols in
which the population evolution is gradual and observes certain
bounds~\cite{varbackbone}.

\begin{definition}[Bounded demographic]
  For $\gamma \in \mathbb{R}^+$, a population sequence
  $(n_r)_{r\in\mathbb{N}}$ is called \emph{$(\gamma, s)$-respecting} if for any
  $S$ of at most $s$ consecutive rounds,
  $\max_{r \in S}n_r \leq \gamma\cdot \min_{r \in S}n_r$.
\end{definition}

\subsection{Adaptive Corruption}
\label{sec:prelim-corr}

The environment already allows the adversary to control a certain number
of parties. If the identities of these parties are fixed at the beginning of
the execution, as presented before, we speak of
\emph{static corruption}\index{Static Corruption}. This is sufficient to treat
Proof-of-Work blockchains (Chapters~\ref{chapter:work}, \ref{chapter:variable}
and \ref{chapter:superlight}), but a more nuanced model is required for
Proof-of-Stake (Chapters~\ref{chapter:stake} and \ref{chapter:sidechains}).
In Proof-of-Stake protocols, just \emph{which} parties are
corrupted will be significant, because the corrupted parties will be maintaining
important secrets in their local state (namely, keys controlling money). As
such, it is not sufficient to capture the notion that $t$ of $n$ parties are
corrupted, but the model will require the adversary to specify which parties are
corrupted. At the point of corruption, the honest party $\party_i$ relinquishes
its entire state to the adversary and is killed, while $t$ is incremented by
$1$. In this more detailed model, the adversary first attempts to corrupt an
honest party $P_i$ by requesting to do so from the environment. This permission
is granted after a certain delay of $\cordelay$ rounds, where $\cordelay$ is a
parameter of our model (and can be different from the network delay $\Delta$).
In particular, if $\cordelay=0$ we talk about \emph{fully adaptive
corruptions}\index{Fully Adaptive Corruption} and the corruption is immediate.
The model with $\cordelay>0$ is referred to as allowing \emph{semi-adaptive
corruptions}\index{Semi-Adaptive Corruption}.


\todo{Copy lecture notes from our \emph{Introduction to Blockchains} course for the next sections...}

\section{The Application Layer}
In creating a decentralized cryptocurrency, the goal is to build a monetary
system which is not reliant on any third parties.

Money is moved around by issuing \emph{transactions}, which instruct the
transfer of a certain amount from one party to another. If Alice holds a
certain amount of money and she wishes to give it to Bob, she creates a
transaction which encodes, in some form, the instruction to pay Bob that certain
amount. That transaction is encoded into a string that is then signed by Alice
and transmitted to the network.

Contrary to centrally controlled currencies in which banks or payment processors
are responsible for maintaining balances, decentralized cryptocurrencies allow
any participant to verify the validity of a transaction. In order for this to be
possible, every transaction is transmitted to every interested party on the
network, a so-called \emph{full node}, who validates it. By recording all past
transactions, every participant is aware of \emph{who owns what} and can thereby
determine if an attempt to spend money is legitimate. No special privileged or
trusted nodes exist on the network.

We now formally define what a transaction is and look at the transaction formats
for Bitcoin and Ethereum. In addition to being the largest cryptocurrencies,
these two systems define two prototypal transaction formats known as the
\emph{UTXO model} and the \emph{Account model}. All other cryptocurrencies adopt
either model, or a hybrid of the two~\cite{chimeric}.

\subsection{Transactions}
Transactions are part of the
\emph{application layer}\index{Application Layer}. As this thesis concerns
itself with the \emph{consensus layer} which organizes transactions into
sequences, we will generally not concern ourselves about their format, and we
will allow the application layer to specify any transaction format it wishes.
Therefore, transactions can be any strings that are deemed valid by the
application layer.

\begin{definition}[Transaction]\index{Transaction}
  A predefined language $\Trans$ of strings in $\{0, 1\}^*$ is called
  a \emph{transaction language}. Elements $tx \in \Trans$ are called
  \emph{transactions}.
\end{definition}

While specific applications such as Bitcoin or Ethereum mandate that
transactions follow a certain format and must include, for example, signatures,
we will not impose such requirements on our protocol. As there are
preconceptions about what constitutes a transaction, we feel the need to give
some examples of transaction languages. The set of valid transactions could be
the empty set, the set $\{0, 1\}$ of bits, the set of natural numbers, or the
set of triplets of a message, a public key, and a digital signature that pass
verification under a certain signature scheme. While the latter corresponds more
closely to practical protocols such as Bitcoin, our treatment is quite general
and has no requirements to remain within this strict format.

Once the set of valid transactions $\Trans$ has been defined by the
application layer, it can now specify a \emph{validity language} which specifies
which \emph{sequences} of transactions are valid. This captures what is deemed
to be a valid transaction given a previous history of transactions in the
system and allows the application layer to specify, for example, that double
spending is not allowed.

\begin{definition}[Validity Language]\index{Validity Language}
  Given a transaction language $\Trans$, a predifined set of finite
  transaction sequences $\ValLang \subseteq \Trans^*$ is called its
  \emph{validity language}.
\end{definition}

The validity languages we will concern ourselves with have the property that
they contain the empty transaction sequence $\epsilon$. This is useful because
it allows a node booting up anew to begin with an empty transaction sequence
before it starts receiving and validating transactions. Our validity languages
are also \emph{extensible}: Given a valid transaction sequence $\overline{\tx}
\in \ValLang$ and a new candidate transaction $\tx \in \Trans$, it is
possible to check whether $\overline{\tx} \concat \tx \in \ValLang$ by applying
a predicate $\textsf{extend}(\overline{\tx}, \tx)$. This $\textsf{extend}$
predicate ensures that the transaction only spends money that belongs to it and
exists in the system. Furthermore, once a
transaction which invalidates the sequence has been added to the sequence, the
sequence remains invalid.

In addition to allowing transactions that spend existing money, it must be
possible to also create new money. The macroeconomic rules for money creation
are captured by another application-specific predicate
$\textsf{mints}(\overline{\tx}, \tx)$ which checks whether a transaction $\tx$
is a valid minting transaction. The rules for this can include, for example,
limiting the amount of money generated per block.
In typical cryptocurrencies, there is one minting transaction allowed per block
and the amount that can be generated by this minting transaction has an
upper bound which is algorithmically determined~\cite{equitability}.
We will leave this predicate undefined in our treatment.

Validity by extension is captured by the definition below:

\begin{definition}[Validity by extension]\label{def:lang-extension}
  Given an extension predicate $\textsf{extends}$, and a transaction language
  $\Trans$, the validity language
  $\ValLang^{\textsf{extends},\textsf{mints},\Trans}$ obtained \emph{by extension} is
  the minimum set of transaction sequences which satisfies the following:

  \begin{enumerate}
    \item \textbf{Base.}
          $\epsilon \in \ValLang^{\textsf{extends},\Trans}$
    \item \textbf{Extension.}
          For all $\overline{\tx} \in \ValLang^{\textsf{extends},\textsf{mints},\Trans}$, for all
          $\tx \in \Trans$, if
          $\textsf{extends}(\overline{\tx}, \tx)$ or $\textsf{mints}(\overline{\tx}, \tx)$ then
          $\overline{\tx} \concat \tx \in \ValLang^{\textsf{extends},\textsf{mints},\Trans}$.
  \end{enumerate}
\end{definition}

From the above definition, the following result follows immediately.

\begin{lemma}[Validity Language Monotonicity]
  Consider a validity language $\ValLang$ generated \emph{by extension} of a
  transaction language $\Trans$. For all $w,w'\in \Trans^*$ we have
  $w\not\in\ValLang \Rightarrow w\concat w'\not\in\ValLang$.
\end{lemma}

Monotonicity mandates the natural property that if a sequence of transactions is
invalid, it cannot become valid again by adding further transactions.

Furthermore, it is useful to ensure transactions are unique. This is captured in
the following requirement for the validity languages of our interest.

\begin{definition}[Validity Language Transaction Uniqueness]\label{def:trans-uniqueness}
  A validity language $\ValLang$ pertaining to a transaction language $\Trans$
  has \emph{transaction uniqueness} if it never contain the same transaction
  twice: for any $\tx\in \Trans$ and any
  $w_1,w_2,w_3\in \Trans^*$ we have

  \[ w_1\concat\tx\concat
  w_2\concat\tx\concat w_3\not\in\ValLang \,~. \]
\end{definition}

The natural ``uniqueness'' property of transactions in existing implementations
is not necessary for our treatment, but it allows for some simplifications.

For illustrative purposes, and because we aim our protocols to be deployable to
existing blockchain systems, in particular to Bitcoin-compatible and
Ethereum-compatible chains, we now explore two particular approaches to the
transaction and validity languages employed in the blockchain space: the UTXO
model and the Account model. We note, however, that our consensus protocols
which enable compression and interoperability are not limited to these two
models, but are generic.

\subsection{Keys and addresses}
While the consensus layer does not require this from the application layer,
all known application layer instantiations make use of public/private keys.
These keys are used to identify money holders in the system. The public key is
used to \emph{receive} money, while the private key is used to sign instructions
to \emph{send} money. The public key is publicized, while the private key
remains secret. As the public key can be assumed to be known to anyone, anyone
can \emph{send} money to everyone and this cannot be prevented. The receipient
does not need to authorize a payment to receive it.

To somewhat increase anonymity, it is recommended that public keys used to
receive money are not recycled. In particular, it is recommended that a new
public key is issued every time money is to be received. The set of all
the private keys that belong to a user are known as a
\emph{wallet}\index{Wallet}.

The lifecycle of money is as follows. If Alice wishes to pay Bob, first she
contacts Bob to ask for his public key. Bob generates a new public/private key
pair and send the public key to Alice. Alice then issues a payment instruction
which instructs Bob to be paid and contains the amount payable as well as Bob's
public key. When Bob wishes to spend the money he received from Alice, he uses
the respective private key to sign a message sending out a payment to someone
else.

Public keys used to send money are encoded into \emph{addresses}\index{Address}
using special encodings such as base58\index{base58} or mixed-case~\cite{eip55}.
These addresses can include checksums or other features which make them harder
to miscommunicate or mistype.

\section{Blockchains}
\subsection{The Consensus Problem}
\subsection{Proof-of-Work}
\cite{pow}\index{Proof-of-Work}
The
honest parties try to distinguish between messages diffused by the other honest
parties and the adversary.

\subsection{Blocks}
genesis
\subsection{Chains of Blocks}
longest chain rule
\subsection{Blockchain Addressing}
\subsection{SPV}
the p2p network and types of nodes: full nodes, SPV nodes, ...

\section{Forks}
\subsection{Hard Forks}
\subsection{Soft Forks}

\section{Cryptocurrencies}
\subsection{Bitcoin}
\subsection{Ethereum}

\section{Smart Contracts}
\subsection{Bitcoin Script}
\subsection{Solidity}

\section{Mathematical Background}
\subsection{The Chernoff Bound}

The following well-known theorem is due to
Rubin~\cite{chernoff1952measure,chernoff2014career}. It will help us derive
negligible bounds for probabilities of bad events.

\begin{theorem}[Chernoff Bounds]
  Let $\{X_i: i \in [n]\}$ be mutually independent Boolean random variables
  such that for all $i \in [n]$ it holds that $\Pr[X_i = 1] = p$. Let
  $X = \sum_{i = 1}^n X_i$ and $\mu = \Ex[X] = pn$. Then, for all
  $\delta \in (0, 1]$ it holds that:

  \[
    \Pr[X \leq (1 - \delta)\mu] \leq \exp(-\frac{\delta^2\mu}{2})
    \text{ and }
    \Pr[X \geq (1 + \delta)\mu] \leq \exp(-\frac{\delta^2\mu}{3})
  \]
\end{theorem}

\section{Blockchain Protocols}

In our treatment, we deal with the \emph{consensus layer}\index{Consensus Layer}
of the protocol. The consensus layer attempts to organize
\emph{application layer} transactions into sequences that belong to the
application's validity language by ordering them into a ledger.

\begin{definition}[Ledger]\index{Ledger}
\end{definition}

\begin{description}
  \item[Persistence.]\index{Persistence}
    For any two honest parties $\party_1,\party_2$ and two rounds $r_1\leq r_2$,
    it holds  $\LView{\Ledger}{\party_1}{r_1} \preceq
    \LView{\check{\Ledger}}{\party_2}{r_2}$.

  \item[Liveness.]\index{Liveness}
    If all honest parties in the system attempt to include a  transaction
    then, at any round $t$ after $u$ rounds (called the
    liveness parameter), any honest party $\party$, if queried,
    will report $\tx \in \LView{\Ledger}{\party}{r}$.
\end{description}


\subsection{Blockchain Backbone}
\todo{Restore old NIPoPoWs paper backbone overview section}

\todo{fix citations}\cite{backbone,pass-asynchronous,varbackbone}

The Bitcoin Backbone protocol is illustrated in Algorithm~\ref{alg.backbone}.

\begin{figure}[t]
\begin{algorithm}[H]
    \caption{\label{alg.backbone} The backbone protocol}
    \begin{algorithmic}[1]
     \Statex
     \Let\chain\varepsilon
        \Let{st}\varepsilon
        \Let{round}1
        \Function{$\textsf{Backbone}$}{$1^\lambda$}
            \Let{\tilde\chain}{\mathsf{maxvalid}( \chain, \mbox{any chain $\chain'$ found in  \textsc{Receive}()}) }
            \If{$\textsc{Input}()\mbox{ contains }\textsc{Read}$}
		        \State{ {\bf write} $R(\chain)$ to \textsc{Output}()}
            \EndIf
            \Let{\langle st, x\rangle}{I(st,\tilde\chain, round, \textsc{Input}(), \textsc{Receive}())}
            \Let{\chain_\mathsf{new}}{\Call{\sf pow}{x, \tilde\chain}}
            \If{$\chain\ne\chain_\mathsf{new}$}
                \Let{\chain}{ \chain_\mathsf{new}}
                \State\textsc{Diffuse}{$(broadcast)$}
            \Else
                \State\textsc{Diffuse}{$(\bot)$}
            \EndIf
            \Let{round}{round+1}
        \EndFunction
        \vskip8pt
    \end{algorithmic}
\end{algorithm}
\end{figure}


\begin{remark}[Impossibility of full semi-synchrony]
Revisiting the $\Delta$-bounded delay model, a folklore observation that has not appeared in
the litature is that blockchain protocols are impossible to obtain in the fully
semi-synchronous setting where \emph{no} conditions are imposed on $\Delta$,
because of the anonymous nature of the model. This impossibility stems from the
fact that $n$ is unknown to the honest parties. To see this, consider an
honest majority execution in which an adversary controls $t = (1 - \delta)(n -
t)$ parties for some $\delta > 0$. If the honest parties take a decision of
transaction acceptance within some time $\Delta$, then that $\Delta$ can be used
as network delay in which the messages of a percentage larger than $\delta$ of
the honest parties are delayed. That setting is then indistinguishable to the
honest parties from a setting in which the adversary controls the majority of
the parties $t > n - t$, as there is an honest percentage which is effectively
eclipsed. This is the case regardless of which solution is used to approach the
problem of consensus -- whether it is through blockchains or other means.
Standard dishonest majority attacks therefore become possible avenues to break
the protocol. Hence, the $\Delta$-bounded delay setting in which $\Delta$ is
unknown but some conditions are imposed on it is the best possible model we can
hope for, as further relaxation would not allow for a solution, as long as
dishonest majority breaks the protocol. The model can only be improved by
relaxing the conditions.
\end{remark}

\todo{fast and slow blocks}

% Honest Majority Assumption
% Chain Growth
% Common Prefix
% Chain Quality
% The Constant Difficulty assumption and its relaxation
\subsection{The Common Reference String}
\todo{Move Proof-of-Burn paper appendix notes here}

% Mention that CRS is not needed (due to Bootstrapping the Blockchain - Directly paper)
% Epochs
\subsection{Ouroboros}

%\paragraph{Description.}
The protocol operates (and was analyzed) in the synchronous model with
semi-adaptive corruptions. % described in Section~\ref{sec:model}.
%
In each slot, each of the parties can determine whether she qualifies as a
so-called \emph{slot leader} for this slot.  The event of a particular party
becoming a slot leader occurs with a probability proportional to the stake
controlled by that party and  is independent for two different slots.
It is determined by a public, deterministic computation from the stake
distribution and so-called \emph{epoch randomness} (we will discuss shortly
where this randomness comes from) in such a way that for each slot, exactly one
leader is elected.

If a party is elected to act as a slot leader for the current slot, she is
allowed to create, sign, and broadcast a block (containing transactions that
move stake among stakeholders).  Parties participating in the protocol are
collecting such valid blocks and always update their current state to reflect
the longest chain they have seen so far that did not fork from their previous
state by too many blocks into the past.

Multiple slots are collected into \textit{epochs}, each of which contains
$R\in \mathbb{N}$ slots. The security arguments in~\cite{C:KRDO17} require $R\geq 10k$
for a security parameter $k$; we will consider $R=12k$ as additional $2k$ slots
in each epoch will be useful for our construction.
%(in Cardano, $R=21600$).
Each epoch is indexed by an index $j \in \mathbb{N}$. During an epoch
$j$, the stake distribution that is used for slot leader election corresponds to
the distribution recorded in the ledger up to
a particular slot of epoch $j-1$, chosen in a way that guarantees that by the
end of epoch $j-1$, there is consensus on the chain up to this slot. (More
concretely, this is the latest slot of epoch $j-1$ that appears in the first $4k$
out of its total $R$ slots.)
Additionally, the \emph{epoch randomness $\rnd_j$} for epoch $j$ is derived
during the epoch $j-1$ via a \emph{guaranteed-output delivery coin tossing}
protocol
%SCRAPE~\cite{ACNS:CasDav17}
that is executed by the epoch slot leaders,
and is available after $10k$ slots of epoch $j-1$ have passed.

In our treatment, we will refer to the relevant parts of the above-described protocol as
follows:
\begin{description}
\item
  $\GetDistr(j)$
  returns the stake distribution $\SD_j$ to be used for epoch $j$, as recorded in the
    chain up to slot $4k$ of epoch $j-1$;

\item
  $\GetRandomness(j)$
  returns the randomness $\rnd_j$ for epoch $j$ as derived during epoch $j-1$;
\item
  $\ValidConsLevel(\Chain)$
  checks the consensus-level validity of a given chain $\Chain$: it verifies that all block hashes
    are correct, signatures are valid and belong to eligible slot leaders;
\item
  $\PickWinningChain(\Chain,\mathcal{C})$
    applies the chain-selection rule: from a set of chains $\{\Chain\}\cup\mathcal{C}$ it
    chooses the longest one that does not fork from the current chain $\Chain$
    more than $k$ blocks in the past;
\item
  $\SlotLeader(U,j,sl,\SD_j,\rnd_{j})$
  determines whether a party $U$ is elected a slot leader for the slot $sl$ of
    epoch $j$, given stake distribution $\SD_j$ and randomness $\rnd_{j}$.
\end{description}
Moreover, the function $\EpochIndex$ (resp. $\SlotIndex$) always returns
the index of the current epoch (resp. slot), and the event $\NewEpoch$ (resp.
$\NewSlot$) denotes the start of a new epoch (resp. slot).
Since we use these functions in a black-box manner, our construction can be
readily adapted to PoS protocols with a similar structure that differ in the
details of these procedures.

%\paragraph{Security.}
Ouroboros was shown in~\cite{C:KRDO17} to achieve
both persistence and liveness
%the common prefix, chain growth, and chain quality properties, proposed as
%security desiderata for blockchain protocols in~\cite{EC:GarKiaLeo15}.
%It is well-known that these properties imply both
%persistence and liveness of the resulting ledger.
%These properties are achieved by Ouroboros
under the following assumptions:
(1) synchronous communication;
(2) $2R$-semi-adaptive corruptions;
(3) majority of stake in the stake distribution for each epoch is
always controlled by honest parties during that epoch.
%(4)  the stake shift per epoch is limited.

