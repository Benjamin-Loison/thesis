\subsection{The Random Oracle}\index{Random Oracle}

Real protocols are instantiated using real hash functions. However, as concrete
mathematical objects, these are not possible to analyze cryptographically, as
they do not have a security parameter. For example, \texttt{SHA256} is a
function with a fixed size of $256$ bits. In these terms, one cannot talk about
negligible probability of failure. Keyed hash functions in which the function
depends on the security parameter are possible to analyze in this manner, but
their use is limited in practice. Additionally, many of the guarantees provided
by keyed hash functions discussed above are insufficient for more elaborate
protocols. In particular, collision and preimage resistance are not enough for
our needs. Nevertheless, our intuition is that practical concrete hash functions
behave nicely, and we wish to include this notion in our model.

To bypass these limitations, we model our hash functions in the Random Oracle
model~\cite{ro}. In this model, the hash function behaves like an ideal random
function. This is helpful, because we can argue all of its outputs will be
unbiased and independent. This abstraction also signifies that we are interested
in working in the realm of \emph{protocol design}, and the intricacies of
practical hash function design and implementation, which stands in the realm of
efficient symmetric cryptography, are beyond the scope of our work.

In the Random Oracle model, we assume the existence of a global oracle machine
$H$ to which every party, adversarial or honest, has access to. The machine
models the hash function and hence allows parties to ask for its evaluation at
any input. The Random Oracle machine receives any input in $\{0, 1\}^*$ and
returns an output in $\{0, 1\}^\lambda$.

\begin{figure}[ht]
\begin{algorithm}[H]
    \caption{\label{alg.ro} The Random Oracle model parameterized by security
             parameter $\kappa$.}
    \begin{algorithmic}[1]
        \Let{T}{\emptyset}
        \Function{\sf H$_\kappa$}{$x$}
            \If{$x \not\in T$}
                \State$T[x] \getsrandomly \{0, 1\}^\kappa$
            \EndIf
            \State\Return$T[x]$
        \EndFunction
        \vskip8pt
    \end{algorithmic}
\end{algorithm}
\end{figure}


The output is chosen as illustrated in Algorithm~\ref{alg.ro}. In detail, the
Random Oracle is parameterized by the security parameter $\lambda$. Upon
receiving some input $x$, if the input has not been encountered before, then it
produces a fresh uniformly random $\lambda$-bit string which it stores in a
dictionary $T$ and returns. If on the other hand it receives an input it has
seen before, it answers consistently with its previous answer, giving the same
answer for the same query by consulting the dictionary $T$. Hence, this
functionality is stateful.

It is imperative that the instance of the machine that all parties communicate
with is the same. Hence, if a party makes a particular query $x$ to the oracle
and then another party asks the same query, the answer will be the same. If the
random oracle is modelled as a stateful functionality, communication between the
parties and the Random Oracle machine can be modelled as Interactive Turing
Machines communicating.

Alterantively but equivalently, the Random Oracle can be defined as a shared
oracle which answers queries according to a function selected uniformly at
random at the beginning of the execution. As a random function can neither be
sampled in polynomial time nor represented in polynomial space, this latter
formulation means that, when the oracle is treated that way, it cannot be
modelled as an Interactive Turing Machine, but must remain an oracle. In this
formulation, at the beginning of the execution, the random oracle $H$ is sampled
uniformly at random from the function space $(\{0, 1\}^\lambda)^{\{0, 1\}^*}$,
and the adversary and honest parties are invoked with oracle access to this same
$H$. As all parties will be polynomial in our treatment, this sampling can also
be understood to be done uniformly at random from the function space $(\{0,
1\}^\lambda)^{\{0, 1\}^{p(\kappa)}}$ where $p(\kappa)$ denotes the polynomial
bounding the total execution of all parties together.

An important feature of the Random Oracle model is that the adversary cannot
compute $H$ locally, but must invoke the oracle to do so. This means that, in
our mathematical treatment, we are allowed to speak of the queries the adversary
has made, count them, look at their responses, and so on. This ability, termed
\emph{random oracle observability} will be critical in our analyses. On the
other hand, we will not make use of the ability of a computational reduction to
modify the outputs of the random oracle at will, termed \emph{random oracle
programmability}, which is at the heart of many security proofs in cryptography.
Our results are therefore stronger, as we only assume a \emph{non-programmable}
Random Oracle~\cite{nielsen2002separating,fischlin2010random}, which is a weaker
assumption than usually made.
