\subsection{Blockchain Backbone}

The Bitcoin Backbone protocol is illustrated in Algorithm~\ref{alg.backbone}.

\begin{figure}[t]
\begin{algorithm}[H]
    \caption{\label{alg.backbone} The backbone protocol}
    \begin{algorithmic}[1]
     \Statex
     \Let\chain\varepsilon
        \Let{st}\varepsilon
        \Let{round}1
        \Function{$\textsf{Backbone}$}{$1^\lambda$}
            \Let{\tilde\chain}{\mathsf{maxvalid}( \chain, \mbox{any chain $\chain'$ found in  \textsc{Receive}()}) }
            \If{$\textsc{Input}()\mbox{ contains }\textsc{Read}$}
		        \State{ {\bf write} $R(\chain)$ to \textsc{Output}()}
            \EndIf
            \Let{\langle st, x\rangle}{I(st,\tilde\chain, round, \textsc{Input}(), \textsc{Receive}())}
            \Let{\chain_\mathsf{new}}{\Call{\sf pow}{x, \tilde\chain}}
            \If{$\chain\ne\chain_\mathsf{new}$}
                \Let{\chain}{ \chain_\mathsf{new}}
                \State\textsc{Diffuse}{$(broadcast)$}
            \Else
                \State\textsc{Diffuse}{$(\bot)$}
            \EndIf
            \Let{round}{round+1}
        \EndFunction
        \vskip8pt
    \end{algorithmic}
\end{algorithm}
\end{figure}


\todo{Continuity}

Chains maintained by honest parties running the backbone protocol in an honest
majority setting satisfy the following three properties.

The \emph{chain growth} property states that the chain of an honest party will
keep growing at a certain rate $\tau$. Because honest parties always extend the
longest chain, this property holds even in executions of dishonest majority.

\begin{definition}[Chain Growth]\index{Chain Growth}
  An execution has \emph{chain growth} with parameters $\alpha \in \mathbb{R}$
  (the \emph{chain velocity}) and $s \in \mathbb{N}$ (the \emph{chunk size}) if
  for all honest parties $\party$ and for all rounds $r$ the following holds. If
  $\party$ has adopted chain $\chain$ at round $r$, then for every round $r' > r
  + s$ the chain $\chain'$ adopted by $\party$ at round $r'$ satisfies:

  \[
  |\chain'| \geq |\chain| + \tau s
  \,.
  \]
\end{definition}

The \emph{chain quality} property states that any large enough ($\geq s$) chunk
of an honestly adopted chain will always contain some ($\rho$) honest blocks.

\begin{definition}[Chain Quality]\index{Chain Quality}
  An execution has \emph{chain quality} with parameters $\rho$ (the
  \emph{chain quality parameter}) and $\ell \in \mathbb{N}$ (the \emph{chunk size})
  if for all honest parties $\party$ and for all rounds $r$ during which the
  party has adopted chain $\chain$ the following holds. For any $i, j$ such that
  $|\chain[i:j]| = \ell$, the block set $\chain' = \{B \in \chain[i{:}j]: B \text{
  was honestly generated}\}$ satisfies:

  \[
  \frac{|\chain'|}{\ell} \geq \rho
  \,.
  \]
\end{definition}

The \emph{common prefix} property states that the chains of two honest parties
cannot deviate much. In particular, they will share a large common prefix and
can only differ by up to $k$ blocks near their ends.

\begin{definition}[Common Prefix]\index{Common Prefix}
  An execution has \emph{common prefix} with parameter $k$ (the \emph{common
  prefix parameter}) if for all honest parties $\party_1, \party_2$ and rounds
  $r_1 \leq r_2$ the following holds. If $\party_1$ has adopted $\chain_1$
  during round $r_1$ and $\party_2$ has adopted $\chain_2$ during round $r_2$,
  then:

  \[
  \chain_1[{:}-k] \preceq \chain_2
  \,.
  \]
\end{definition}

The next three theorems establish that these three properties hold in typical
executions and have been proven in the backbone series of papers. These theorems
hold in both the synchronous and the $\Delta$-bounded delay network
models~\cite{pass-asynchronous} as well as in the static~\cite{backbone} and
variable difficulty settings~\cite{varbackbone}. We state them here without
proof.

In the synchronous model it holds that~\cite{backbone}:

\begin{theorem}[Chain Growth]
  A typical execution satisfies \emph{Chain Growth} with
  velocity $\alpha = (1 - \epsilon)f$ and chunk size $s \geq \lambda$.
\end{theorem}

\begin{theorem}[Chain Quality]
  A typical execution satisfies \emph{Chain Quality} with quality
  $\rho = 1 - (1 + \frac{\delta}{2})\frac{t}{n - t} - \frac{\epsilon}{1 - \epsilon}$
  and chunk size $\ell \geq 2\lambda f$.
\end{theorem}

\begin{theorem}[Common Prefix]
  A typical execution satisfies \emph{Common Prefix} with common prefix
  parameter $k \geq 2\lambda f$.
\end{theorem}

With the addition of a $\Delta$-bounded delay it holds that~\cite{backbone-new}:

\begin{theorem}[Chain Growth]
  A typical execution satisfies \emph{Chain Growth} with
  velocity $\alpha = (1 - \epsilon)f(1 - f)^{\Delta - 1}$
  and chunk size $s \geq \lambda$.
\end{theorem}

\begin{theorem}[Chain Quality]
  A typical execution satisfies \emph{Chain Quality} with
  quality
  $\rho = 1 - \frac{1}{(1 - \epsilon)(1 - f)^\Delta}\frac{t}{n - t} - \frac{e}{1 - \epsilon}(1 + \frac{\Delta}{\lambda})$
  and chunk size $\ell \geq 2\lambda f + 2\Delta f$.
\end{theorem}

\begin{theorem}[Common Prefix]
  A typical execution satisfies \emph{Common Prefix} with
  $k \geq 2\lambda f + 2\Delta$.
\end{theorem}

\begin{remark}[Impossibility of full semi-synchrony]
Revisiting the $\Delta$-bounded delay model, a folklore observation that has not appeared in
the litature is that blockchain protocols are impossible to obtain in the fully
semi-synchronous setting where \emph{no} conditions are imposed on $\Delta$,
because of the anonymous nature of the model. This impossibility stems from the
fact that $n$ is unknown to the honest parties. To see this, consider an
honest majority execution in which an adversary controls $t = (1 - \delta)(n -
t)$ parties for some $\delta > 0$. If the honest parties take a decision of
transaction acceptance within some time $\Delta$, then that $\Delta$ can be used
as network delay in which the messages of a percentage larger than $\delta$ of
the honest parties are delayed. That setting is then indistinguishable to the
honest parties from a setting in which the adversary controls the majority of
the parties $t > n - t$, as there is an honest percentage which is effectively
eclipsed. This is the case regardless of which solution is used to approach the
problem of consensus -- whether it is through blockchains or other means.
Standard dishonest majority attacks therefore become possible avenues to break
the protocol. Hence, the $\Delta$-bounded delay setting in which $\Delta$ is
unknown but some conditions are imposed on it is the best possible model we can
hope for, as further relaxation would not allow for a solution, as long as
dishonest majority breaks the protocol. The model can only be improved by
relaxing the conditions.
\end{remark}

\todo{fast and slow blocks}
