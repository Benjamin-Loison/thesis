\section{The Application Layer}
In creating a decentralized cryptocurrency, the goal is to build a monetary
system which is not reliant on any third parties.
Money is moved around by issuing \emph{transactions}, which instruct the
transfer of a certain amount from one party to another. If Alice holds a
certain amount of money and she wishes to give it to Bob, she creates a
transaction which encodes, in some form, the instruction to pay Bob that certain
amount. That transaction is encoded into a string that is then signed by Alice
and transmitted to the network.

Contrary to centrally controlled currencies in which banks or payment processors
are responsible for maintaining balances, decentralized cryptocurrencies allow
any participant to verify the validity of a transaction. In order for this to be
possible, every transaction is transmitted to every interested party on the
network, a so-called \emph{full node}, who validates it. By recording all past
transactions, every participant is aware of \emph{who owns what} and can thereby
determine if an attempt to spend money is legitimate. No special privileged or
trusted nodes exist on the network.

We now formally define what a transaction is and look at the transaction formats
for Bitcoin and Ethereum. In addition to being the largest cryptocurrencies,
these two systems define two prototypal transaction formats known as the
\emph{UTXO model} and the \emph{Account model}. All other cryptocurrencies adopt
either model, or a hybrid of the two~\cite{chimeric}.

\subsection{Transactions}
Transactions are part of the
\emph{application layer}\index{Application Layer}. As this thesis concerns
itself with the \emph{consensus layer} which organizes transactions into
sequences, we will generally not concern ourselves about their format, and we
will allow the application layer to specify any transaction format it wishes.
Therefore, transactions can be any strings that are deemed valid by the
application layer.

\begin{definition}[Transaction]\index{Transaction}
  A predefined language $\Trans$ of strings in $\{0, 1\}^*$ is called
  a \emph{transaction language}. Elements $tx \in \Trans$ are called
  \emph{transactions}.
\end{definition}

While specific applications such as Bitcoin or Ethereum mandate that
transactions follow a certain format and must include, for example, signatures,
we will not impose such requirements on our protocol. As there are
preconceptions about what constitutes a transaction, we feel the need to give
some examples of transaction languages. The set of valid transactions could be
the empty set, the set $\{0, 1\}$ of bits, the set of natural numbers, or the
set of triplets of a message, a public key, and a digital signature that pass
verification under a certain signature scheme. While the latter corresponds more
closely to practical protocols such as Bitcoin, our treatment is quite general
and has no requirements to remain within this strict format.

Once the set of valid transactions $\Trans$ has been defined by the
application layer, it can now specify a \emph{validity language} which specifies
which \emph{sequences} of transactions are valid. This captures what is deemed
to be a valid transaction given a previous history of transactions in the
system and allows the application layer to specify, for example, that double
spending is not allowed.

\glsxtrnewsymbol[description={validity language}]{vallang}{$\ValLang$}\glsadd{vallang}
\begin{definition}[Validity Language]\index{Validity Language}
  Given a transaction language $\Trans$, a predifined set of finite
  transaction sequences $\ValLang \subseteq \Trans^*$ is called its
  \emph{validity language}.
\end{definition}

The validity languages we will concern ourselves with have the property that
they contain the empty transaction sequence $\epsilon$. This is useful because
it allows a node booting up anew to begin with an empty transaction sequence
before it starts receiving and validating transactions. Our validity languages
are also \emph{extensible}: Given a valid transaction sequence $\overline{\tx}
\in \ValLang$ and a new candidate transaction $\tx \in \Trans$, it is
possible to check whether $\overline{\tx} \concat \tx \in \ValLang$ by applying
a predicate $\textsf{extend}(\overline{\tx}, \tx)$. This $\textsf{extend}$
predicate ensures that the transaction only spends money that belongs to it and
exists in the system. Furthermore, once a
transaction which invalidates the sequence has been added to the sequence, the
sequence remains invalid.

In addition to allowing transactions that spend existing money, it must be
possible to also create new money. The macroeconomic rules for money creation
are captured by another application-specific predicate
$\textsf{mints}(\overline{\tx}, \tx)$ which checks whether a transaction $\tx$
is a valid minting transaction. The rules for this can include, for example,
limiting the amount of money generated per block.
In typical cryptocurrencies, there is one minting transaction allowed per block
and the amount that can be generated by this minting transaction has an
upper bound which is algorithmically determined~\cite{equitability}.
We will leave this predicate undefined in our treatment.

Validity by extension is captured by the definition below:

\begin{definition}[Validity by extension]\label{def:lang-extension}
  Given an extension predicate $\textsf{extends}$, and a transaction language
  $\Trans$, the validity language
  $\ValLang^{\textsf{extends},\textsf{mints},\Trans}$ obtained \emph{by extension} is
  the minimum set of transaction sequences which satisfies the following:

  \begin{enumerate}
    \item \textbf{Base.}
          $\epsilon \in \ValLang^{\textsf{extends},\textsf{mints},\Trans}$
    \item \textbf{Extension.}
          For all $\overline{\tx} \in \ValLang^{\textsf{extends},\textsf{mints},\Trans}$, for all
          $\tx \in \Trans$, if
          $\textsf{extends}(\overline{\tx}, \tx)$ or $\textsf{mints}(\overline{\tx}, \tx)$ then
          $\overline{\tx} \concat \tx \in \ValLang^{\textsf{extends},\textsf{mints},\Trans}$.
  \end{enumerate}
\end{definition}

From the above definition, the following result follows immediately.

\begin{lemma}[Validity Language Monotonicity]
  Consider a validity language $\ValLang$ generated \emph{by extension} of a
  transaction language $\Trans$. For all $w,w'\in \Trans^*$ we have
  $w\not\in\ValLang \Rightarrow w\concat w'\not\in\ValLang$.
\end{lemma}

Monotonicity mandates the natural property that if a sequence of transactions is
invalid, it cannot become valid again by adding further transactions.

Furthermore, it is useful to ensure transactions are unique. This is captured in
the following requirement for the validity languages of our interest.

\begin{definition}[Validity Language Transaction Uniqueness]\label{def:trans-uniqueness}
  A validity language $\ValLang$ pertaining to a transaction language $\Trans$
  has \emph{transaction uniqueness} if it never contain the same transaction
  twice: for any $\tx\in \Trans$ and any
  $w_1,w_2,w_3\in \Trans^*$ we have

  \[ w_1\concat\tx\concat
  w_2\concat\tx\concat w_3\not\in\ValLang \,~. \]
\end{definition}

The natural ``uniqueness'' property of transactions holds in existing
implementations, but is not necessary for our treatment, albeit allowing for
some simplifications.

For illustrative purposes, and because we aim our protocols to be deployable to
existing blockchain systems, in particular to Bitcoin-compatible and
Ethereum-compatible chains, we now explore two particular approaches to the
transaction and validity languages employed in the blockchain space: the UTXO
model and the Account model. We note, however, that our consensus protocols
which enable compression and interoperability are not limited to these two
models, but are generic.

\subsection{Keys and addresses}
While the consensus layer does not require this from the application layer,
all known application layer instantiations make use of public/private keys.
These keys are used to identify money holders in the system. The public key is
used to \emph{receive} money, while the private key is used to sign instructions
to \emph{send} money. The public key is publicized, while the private key
remains secret. As the public key can be assumed to be known to anyone, anyone
can \emph{send} money to everyone and this cannot be prevented. The receipient
does not need to authorize a payment to receive it.

To somewhat increase anonymity, it is recommended that public keys used to
receive money are not recycled. In particular, it is recommended that a new
public key is issued every time money is to be received. The set of all
the private keys that belong to a user are known as a
\emph{wallet}\index{Wallet}.

The lifecycle of money is as follows. If Alice wishes to pay Bob, first she
contacts Bob to ask for his public key. Bob generates a new public/private key
pair and send the public key to Alice. Alice then issues a payment instruction
which instructs Bob to be paid and contains the amount payable as well as Bob's
public key. When Bob wishes to spend the money he received from Alice, he uses
the respective private key to sign a message sending out a payment to someone
else.

Public keys used to send money are encoded into \emph{addresses}\index{Address}
using special encodings such as base58\index{base58} or mixed-case~\cite{eip55}.
These addresses can include checksums or other features which make them harder
to miscommunicate or mistype.

\subsection{The UTXO Model}\index{UTXO Model}

The UTXO model was first introduced in Bitcoin. In the UTXO model, each
transaction is a node in a \emph{weighted directed acyclic graph}. Despite being
represented as graph nodes, transactions do not correspond to accounts or
account holders, but denote a payment. An edge incoming to a transaction
designates \emph{who is paying}, and an edge outgoing from a transaction denotes
\emph{who is being paid}. The edge is annotated with two pieces of information:
Its \emph{weight} that designates the \emph{amount} paid; and the \emph{address}
of the payment receipient. The weights are integers denominated in the smallest
denomination of the currency. In Bitcoin, this is the Satoshi\index{Satoshi},
which equals $10^{-8}$ BTC. Denominating amounts in satoshis rather than
bitcoins is helpful because it avoids rounding errors. The data of a transaction
contains the collection of its input and output edges. The hash of that data
gives the \emph{transaction id}, or \emph{txid}\index{txid}, which can be used
to uniquely refer to a transaction. In the case of Bitcoin, the hash function
used for this purpose is $\SHA$.

A directed edge connecting two transactions denotes the transfer of money
from one transaction to the other. It indicates that money was paid to a
beneficiary through the first transaction, and that beneficiary subsequently
spent the money they had received through the second transaction. As such,
a \emph{coin is a chain of transactions}. Transactions can have \emph{dangling
outgoing edges}, which are edges that are outgoing from a transaction but have
not yet been connected as incoming edges into another transaction. These edges
have not yet been spent and are available for spending. They are known as
Unspent Transaction Outputs (UTXOs)\index{UTXO}. The collection of all UTXOs
constitutes the available money in the system.

The transaction DAG is known to all network participants. To determine how much
money Alice owns, she collects the UTXOs of the transaction DAG and filters it
according to the addresses that she is the owner of; that is, addresses that
correspond to a public key for which she owns the corresponding private key. The
sum of the amounts in these UTXOs is the total amount of money she owns. She
does not look at \emph{spent} transaction outputs, because these have already
been consumed and they cannot be spent again.

To send money to Bob, Alice finds a UTXO $e_1$ she owns. She then creates a new
transaction $\tx$ with one incoming edge $e_1$ and one outgoing edge $e_2$ and
connects the UTXO $e_1$ as the incoming edge of $\tx$. As $e_1$ is no longer
a dangling edge since it is now incoming to $\tx$, it is longer in the UTXO set.
The outgoing edge of $\tx$ is now dangling, and so is now part of the UTXO set.
Therefore, with the transaction, the UTXO set is updated so that $e_1$ is
removed from it and $e_2$ is added to it. Edge $e_2$ is weighted by the amount
that Alice wishes to pay and annotated by the address of the beneficiary to be
paid, which is deduced from the public key of the receipient.

To prove that she is the rightful owner of $e_1$, Alice produces a signature
using the private key corresponding to the public key that corresponds to the
address annotated in $e_1$. The contents signed by Alice's signature include the
reference to $e_1$ so that it is clear \emph{which} coin Alice is spending. The
contents also include $e_2$ and in particular the amount and new beneficiary.
This ensures Alice's payment cannot be forged to be made payable to a different
beneficiary.

Alice then diffuses her transaction on the network, which is subsequently
received by Bob.

A transaction can have multiple inputs. This is useful if Alice has received
multiple payments and she wants to make a larger payment. She creates one
transaction with multiple inputs and a single output. The transaction consumes
all inputs and pays them to the given output. A transaction can also have
multiple outputs. Because every transaction fully consumes its inputs, if Alice
has received a large payment in a single UTXO, she can consume it via a
transaction that contains multiple outputs so that she can make multiple
payments with it simultaneously. She can also consume a large UTXO by creating
a transaction that contains two outputs: one output that pays out some amount to
her desired beneficiary, and another that pays the rest of the money back to
Alice. Outputs that pay back the original owner are known as change outputs.
Change outputs are paid out to addresses created for this purpose that are known
as \emph{change addresses}\index{Change Address} and are owned by the original
owner. A typical transaction on a UTXO-based network consumes one or more inputs
all owned by the same owner and contains two outputs: One which indicates the
payment beneficiary, and another which indicates the change.

An edge can be a dangling \emph{outgoing} edge from a transaction, in that it is
not yet an \emph{incoming} edge to any other transaction. However, every edge
must have a transaction from which it is outgoing. As such, every edge can be
associated with a unique transaction that produced it. The outgoing edges of
every transaction are ordered and indexed by the natural numbers. An edge can
therefore be identified by the transaction that produced it together with its
index among the outgoing edges of that transaction. This pair is known as an
\emph{outpoint}.

\begin{definition}[UTXO transaction]
  A \emph{UTXO transaction} $\tx$ is a pair $(in, out)$ of inputs and outputs such
  that $in$ is a sequence of outpoints $(in_1, in_2, \cdots)$ and $out$ is a
  sequence of edges $(out_1, out_2, \cdots)$. It must hold that every input
  $in_i$ is a tuple $(txid_i, j_i, \sigma_i)$ where the pair $(txid_i, j_i)$ is
  an outpoint indicated by the transaction id $txid_i$ and an index $j_i$
  which marks the output index within the transaction
  identified by $txid_i$. The signature $\sigma_i$ is a signature on $\tx$
  in which all $\sigma_k$ have been replaced by $\epsilon$. Additionally, it
  must hold that every output $out_i$ is a pair $(amount, pk)$ indicating the
  amount payable and the beneficiary address $pk$. The set of all syntactically
  valid UTXO transactions is the UTXO transaction language
  $\Trans_{\textsc{UTXO}}$.
\end{definition}

A transaction's signatures must be valid. They must sign the plaintext $\tx$ in
which all $\sigma_k$ have been replaced by $\epsilon$. The replacement is made
so that signatures do not have to sign themselves, which would be an impossible
task. They must also pass the verification using the public key indicated by the
respective outpoint. A transaction must follow the \emph{conservation
principle} which mandates that the sum of output
amounts cannot exceed the sum of input amounts.

These rules define transaction validity inductively. When Alice receives a
transaction from the network, whether it pertains to a payment to her or not,
she needs to validate it. This require Alice to maintain a currently valid UTXO
set against which she will compare the transaction and which she will use to
update this UTXO set. Therefore, to verify a transaction, the procedure followed
is thus. First, Alice already holds some valid UTXO set by the inductive
hypothesis (she starts with the empty UTXO set). First, she checks that the
inputs to the new transaction refer to outpoints that are in her existing UTXO
set. She follows the outpoint pointers to check that the law of conservation
holds for the new transaction. She also verifies all the signatures on the
inputs of the new transaction using the public keys that appear in the
outputs referenced by the outpoints. As long as everything is valid, she updates
her UTXO to remove the outputs referenced by the outpoints and add the outputs
of the new transaction.

We are now ready to define transaction validity formally in the UTXO model.
While the UTXO model should already be straightforward, we go through the
exercise of presenting it precisely to illustrate the expressiveness of validity
language formulations.

First, we define outpoints.

\begin{definition}[Outpoint]\index{Outpoint}
  Consider the UTXO transactions language $\Trans_{\textsc{UTXO}}$ and the transaction id
  hash function $H$.
  Let $\overline{\tx} \in \Trans_{\textsc{UTXO}}^*$ be a transaction sequence
  containing unique transactions.
  Define the transaction lookup function
  $\textsf{lookup-tx}_{\overline{\tx}}(txid)$ to equal $\tx$ if $H(\tx) = txid$ and
  $\tx \in \overline{\tx}$, otherwise set $\textsf{lookup-tx}_{\overline{\tx}}(txid) = \bot$.
  Define the \emph{outpoint lookup function}
  $\textsf{outpoint}_{\overline{\tx}}(txid, j)$ to be the $j^\text{th}$ item of $out$ where
  $(in, out) = \textsf{lookup-tx}(txid)$ if such a $j$ exists. Otherwise
  set $\textsf{outpoint}_{\overline{\tx}}(txid, j) = \bot$.
\end{definition}

We can now define what the UTXO set of a transaction sequence is.

\begin{definition}[UTXO set]
  Consider the transaction language $\Trans_{\textsc{UTXO}}$ and the transaction
  id hash function $H$. Let
  $\overline{\tx} \in \Trans_{\textsc{UTXO}}^*$ be a transaction sequence containing
  unique transactions.
  The \emph{UTXO set} $\textsc{UTXO}(\overline{\tx})$ of $\overline{\tx}$ is defined as
  the set which contains all outpoints $(txid, j)$ with the following properties:

  \begin{enumerate}
    \item \textbf{Unspent.}
          There is no $\tx' \in \overline{\tx}$ with $\tx' = (in', out')$ such that
          $(txid, j) \in in'$.
    \item \textbf{Transaction output.}
          There is some $\tx \in \overline{\tx}$ with $\tx = (in, out)$
          such that $H(\tx) = txid$ and $1 \leq j \leq |out|$.
  \end{enumerate}
\end{definition}

We are now ready to define the validity language for a UTXO system. We will
define our validity language by extension according to
Definition~\ref{def:lang-extension}.
The $\textsf{extends}_{\textsc{UTXO}}(\overline{\tx}, \tx)$ predicate checks that
the transaction $\tx$ can be applied on top of the existing transaction sequence
$\overline{\tx}$ and is defined as follows.

\begin{definition}[UTXO validity]
  Let $\overline{\tx} \in \Trans_{\textsc{UTXO}}^*$ and $\tx \in \Trans_{\textsc{UTXO}}$ with
  $\tx = (in, out)$ and let $\textsf{mint}$ be a \emph{minting predicate}.
  Let $S = (\Gen, \Sig, \Ver)$ be a secure signature scheme.

  We define $\textsf{extends}_{\textsc{UTXO}}(\overline{\tx}, \tx)$ to be $\true$
  if all of the following conditions hold:

  \begin{enumerate}
    \item \textbf{Rightful.}
          For all $(txid, j, \sigma) \in in$ we have that
          $\Ver(pk, \tx', \sigma)$ where
          $(amount, pk) = \textsf{outpoint}_{\overline{\tx}}(txid, j)$
          and $\tx'$ denotes $\tx$ with all signatures replaced by $\epsilon$.
    \item \textbf{Unspent.}
          All $(txid, j, \sigma) \in in$ form unique outpoints $(txid, j)$ and
          for all $(txid, j, \sigma) \in in$ we have
          $(txid, j) \in \textsc{UTXO}(\overline{\tx})$
    \item \textbf{Conserving.}\index{Conservation Principle}
          \[\sum_{in_i \in in} in_i.\textsf{amount} \geq \sum_{out_i \in out} out_i.\textsf{amount}\,.\]
  \end{enumerate}

  The \emph{UTXO validity language} $\ValLang_{\textsc{UTXO}}$ with macroeconomic
  policy $\textsf{mints}$ is defined as
  \[
    \ValLang^{\textsf{extends}_{\textsc{UTXO}},\textsf{mints}}\,.
  \]
\end{definition}

\subsection{The Account Model}\index{Account Model}
A different approach to transactions is followed in the Account Model, which was
first put forth by Ethereum. Instead of maintaining UTXOs, the Account Model
maintains account balances. Transactions are instructions to transfer an amount
from one account to another. Accounts are represented by addresses. A
transaction therefore contains the source account, the target account, the
amount, as well as a signature authorizing the transfer. The conservation
principle here mandates that the source account must have sufficient balance to
cover the amount a transaction wishes to spend.

Contrary to the UTXO model where every UTXO is spent only once, here it is
possible to have multiple transactions which spend from the same source account,
pay into the same target account, and transmit the same amount. As such, these
transactions require a nonce, or counter, which ensures they are unique. This
counter is necessary. If such no such counter was present, the system needing to
support a repeated transfer would accept the same transaction twice. But such
admissibility of transaction duplication is problematic, as an adversary could
replay an existing transaction, with the same signature, benefiting her account
twice, even though no such intention was recorded by the sender. The counter is
therefore required to signal the fact that the sender wishes to initiate yet
another transfer.

\begin{definition}[Account Transaction]
  An \emph{account transaction} $\tx$ is a tuple
  $(\textsf{from},\allowbreak
   \textsf{to},\allowbreak
   \textsf{amount},\allowbreak
   ctr,\allowbreak
   \sigma)$ such that $\textsf{from} \neq \textsf{to}$.
  The signature $\sigma$ is a signature on
  $(\textsf{from},\allowbreak
    \textsf{to},\allowbreak
    \textsf{amount},\allowbreak
    ctr,\allowbreak
    \epsilon)$. The set of all
  syntactically valid account transactions is the account transaction language
  $\Trans_\textsc{account}$.
\end{definition}

Balances can be obtained from a transaction sequence by summing the amounts
transfered. We will then make use of balances to define whether an account
transaction validly extends a transaction sequence.

\begin{definition}[Account Balances]\index{Balance}
  Let $\overline{\tx} \subseteq \Trans_\textsc{account}^*$ be an account
  transaction sequence. We define the \emph{balance}
  $\textsf{balance}_{\overline{\tx}}(\textsf{acc})$
  of an account $\textsf{acc}$
  at the end of the transaction sequence $\overline{\tx}$
  as follows. If $\overline{\tx} = \epsilon$, then define
  $\textsf{balance}_{\overline{\tx}}(\textsf{acc}) = 0$.
  Otherwise, $\overline{\tx}$ is non-empty, so set
  $\overline{\tx} = \overline{\tx}'\concat\tx$
  and let $(\textsf{from}, \textsf{to}, \textsf{amount}, \sigma) = \tx$.
  Recursively let
  $\textsf{balance}' = \textsf{balance}_{\overline{\tx}'}(\textsf{acc})$.
  \begin{itemize}
    \item If $\textsf{from} = \textsf{acc}$, then define
          $\textsf{balance}_{\overline{\tx}}(\textsf{acc}) = \textsf{balance}' - \textsf{amount}$.
    \item If $\textsf{to} = \textsf{acc}$, then define
          $\textsf{balance}_{\overline{\tx}}(\textsf{acc}) = \textsf{balance}' + \textsf{amount}$.
    \item Otherwise
          define $\textsf{balance}_{\overline{\tx}}(\textsf{acc}) = \textsf{balance}'$.
\end{itemize}
\end{definition}

We are now ready to define the validity language for an account-based system.
Again, we will
define our validity language by extension according to
Definition~\ref{def:lang-extension}.

The $\textsf{extends}_{\textsc{account}}(\overline{\tx},\allowbreak \tx)$ predicate checks that
the transaction $\tx$ can be applied on top of the existing transaction sequence
$\overline{\tx}$ and is defined as follows.

\begin{definition}[Account validity]
  Let $\overline{\tx} \in \Trans_\textsc{account}^*$ and $\tx \in \Trans_{\textsc{account}}$.
  Let $\tx = (\textsf{from},\allowbreak \textsf{to},\allowbreak \textsf{amount},\allowbreak \textsf{ctr},\allowbreak \sigma)$
  and let $\textsf{mint}$ be a \emph{minting predicate}.
  Let $S = (\Gen, \Sig, \Ver)$ be a secure signature scheme.

  We define $\textsf{extends}_{\textsc{account}}(\overline{\tx}, \tx)$ to be $\true$
  if:

  \begin{enumerate}
    \item \textbf{Rightful.}
          $\Ver(\textsf{from}, \tx', \sigma)$ where
          and $\tx' = (\textsf{from}, \textsf{to}, \textsf{amount}, \textsf{ctr}, \epsilon)$.
    \item \textbf{Conserving.}\index{Conservation Principle}
          For all $\textsf{acc}$ it holds that
          $\textsf{balances}_{\overline{\tx} \concat \tx}(\textsf{acc}) \geq 0$.
  \end{enumerate}

  The \emph{account validity language} $\ValLang_{\textsc{account}}$ with macroeconomic
  policy $\textsf{mints}$ is defined as
  $\ValLang^{\textsf{extends}_{\textsc{account}},\textsf{mints}}$.
\end{definition}
