\subsection{Ouroboros}

\todo{Re-read this section and clean up}

%\paragraph{Description.}
The protocol operates (and was analyzed) in the synchronous model with
semi-adaptive corruptions. % described in Section~\ref{sec:model}.
%
In each slot, each of the parties can determine whether she qualifies as a
so-called \emph{slot leader} for this slot.  The event of a particular party
becoming a slot leader occurs with a probability proportional to the stake
controlled by that party and  is independent for two different slots.
It is determined by a public, deterministic computation from the stake
distribution and so-called \emph{epoch randomness} (we will discuss shortly
where this randomness comes from) in such a way that for each slot, exactly one
leader is elected.

If a party is elected to act as a slot leader for the current slot, she is
allowed to create, sign, and broadcast a block (containing transactions that
move stake among stakeholders).  Parties participating in the protocol are
collecting such valid blocks and always update their current state to reflect
the longest chain they have seen so far that did not fork from their previous
state by too many blocks into the past.

Multiple slots are collected into \textit{epochs}, each of which contains
$R\in \mathbb{N}$ slots. The security arguments in~\cite{ouroboros} require $R\geq 10k$
for a security parameter $k$; we will consider $R=12k$ as additional $2k$ slots
in each epoch will be useful for our construction.
%(in Cardano, $R=21600$).
Each epoch is indexed by an index $j \in \mathbb{N}$. During an epoch
$j$, the stake distribution that is used for slot leader election corresponds to
the distribution recorded in the ledger up to
a particular slot of epoch $j-1$, chosen in a way that guarantees that by the
end of epoch $j-1$, there is consensus on the chain up to this slot. (More
concretely, this is the latest slot of epoch $j-1$ that appears in the first $4k$
out of its total $R$ slots.)
Additionally, the \emph{epoch randomness $\rnd_j$} for epoch $j$ is derived
during the epoch $j-1$ via a \emph{guaranteed-output delivery coin tossing}
protocol
%SCRAPE~\cite{ACNS:CasDav17}
that is executed by the epoch slot leaders,
and is available after $10k$ slots of epoch $j-1$ have passed.

In our treatment, we will refer to the relevant parts of the above-described protocol as
follows:
\begin{description}
\item
  $\GetDistr(j)$
  returns the stake distribution $\SD_j$ to be used for epoch $j$, as recorded in the
    chain up to slot $4k$ of epoch $j-1$;

\item
  $\GetRandomness(j)$
  returns the randomness $\rnd_j$ for epoch $j$ as derived during epoch $j-1$;
\item
  $\ValidConsLevel(\Chain)$
  checks the consensus-level validity of a given chain $\Chain$: it verifies that all block hashes
    are correct, signatures are valid and belong to eligible slot leaders;
\item
  $\PickWinningChain(\Chain,\mathcal{C})$
    applies the chain-selection rule: from a set of chains $\{\Chain\}\cup\mathcal{C}$ it
    chooses the longest one that does not fork from the current chain $\Chain$
    more than $k$ blocks in the past;
\item
  $\SlotLeader(U,j,sl,\SD_j,\rnd_{j})$
  determines whether a party $U$ is elected a slot leader for the slot $sl$ of
    epoch $j$, given stake distribution $\SD_j$ and randomness $\rnd_{j}$.
\end{description}
Moreover, the function $\EpochIndex$ (resp. $\SlotIndex$) always returns
the index of the current epoch (resp. slot), and the event $\NewEpoch$ (resp.
$\NewSlot$) denotes the start of a new epoch (resp. slot).
Since we use these functions in a black-box manner, our construction can be
readily adapted to PoS protocols with a similar structure that differ in the
details of these procedures.

%\paragraph{Security.}
Ouroboros was shown in~\cite{ouroboros} to achieve
both persistence and liveness
%the common prefix, chain growth, and chain quality properties, proposed as
%security desiderata for blockchain protocols in~\cite{backbone}.
%It is well-known that these properties imply both
%persistence and liveness of the resulting ledger.
%These properties are achieved by Ouroboros
under the following assumptions:
(1) synchronous communication;
(2) $2R$-semi-adaptive corruptions;
(3) majority of stake in the stake distribution for each epoch is
always controlled by honest parties during that epoch.
%(4)  the stake shift per epoch is limited.
