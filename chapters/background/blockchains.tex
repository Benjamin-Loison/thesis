\subsection{Chains and Ledgers}
\index{Genesis}
\index{Anchored Chain}
\index{Blockchain}
\index{Block}
\glsxtrnewsymbol[description={the genesis block}]{genesis}{$\mathcal{G}$}\glsadd{genesis}
\glsxtrnewsymbol[description={blockchain}]{chain}{$\chain$}\glsadd{chain}
Blockchains are finite block sequences obeying the \emph{blockchain property}:
that in every block in the chain there exists a pointer to its previous block.
We denote a blockchain (or simply \emph{chain}) by $\chain$. A special block
generated at the beginning of the protocol execution called the \emph{genesis}
block $\mathcal{G}$ is known by all parties. Every valid chain must begin with
the genesis block. We call such a chain \emph{anchored}.

\index{Block ID}
\glsxtrnewsymbol[description={block}]{block}{$B$}\glsadd{block}
Given chains $\chain_1, \chain_2$ and block $B$ we concatenate them as
$\chain_1 \chain_2$ or $\chain_1 B$. $\chain_2[0]$ must point to $\chain_1[-1]$
and $B$ must point to $\chain_1[-1]$. The \emph{id} function on a block returns
the hash of the block header data $\textsf{id} = H(ctr, x,
h')$.

In blockchain protocols, each honest party $\party$ maintains a currently
adopted chain. We denote $\LView{\chain}{\party}{t}$ the chain adopted by party $\party$ at slot~$t$.

A \emph{ledger} (denoted in bold-face, e.g. $\Ledger$) is a mechanism for
maintaining a sequence of transactions, often stored in the form of a
blockchain.

\glsxtrnewsymbol[description={ledger state}]{ledgerstate}{$\LState$}\glsadd{ledgerstate}
\glsxtrnewsymbol[description={stable ledger view by party $\party$ at time $t$}]{ledgerview}{$\LView{\Ledger}{\party}{t}$}\glsadd{ledgerview}
\glsxtrnewsymbol[description={unstable ledger view by party $\party$ at time $t$}]{ledgerunstable}{$\check{\Ledger}{\party}[t]$}\glsadd{ledgerunstable}
\index{Ledger State}
We call a \emph{ledger state}
%$\LState$
a concrete sequence of transactions
$\tx_1,\tx_2, \ldots$ stored in the \emph{stable} part of a ledger $\Ledger$, typically as viewed by a
particular party.
Hence, in every blockchain-based ledger $\Ledger$, every fixed
chain $\chain$ defines a concrete ledger state % $\LState$
by applying  the interpretation rules given as a part of the description
of~$\Ledger$ (for example, the ledger state is obtained from the
blockchain by dropping the last $k$ blocks and serializing the transactions in
the remaining blocks).
We maintain the typographic convention that a ledger state (e.g. $\LState$) always
belongs to the bold-face ledger of the same name (e.g. $\Ledger$).
We denote by $\LView{\Ledger}{P}{t}$ the ledger state of a ledger $\Ledger$ as viewed by a party
$\party$ at the beginning of a time slot $t$,
%Finally, recall that a transaction is called \emph{stable} if it is contained in a block
%at least $k$ blocks away from the end of the blockchain. For a blockchain
%$\Chain$, we denote by $\stable{\Chain}$ the stable part of $\Chain$ (which
%implicitly depends on $k$). Moreover, for a  chain $\Ledger$ we
%denote by $\Ledger^t$ its form in some past time slot $t$.
and by $\check{\Ledger}^P[t]$ the complete state of the ledger (at time
$t$) including all
pending transactions that are not stable yet.

\glsxtrnewsymbol[description={union of all ledger views}]{ledgercup}{$\Ledger_i^{\cup}$}\glsadd{ledgercup}
\glsxtrnewsymbol[description={intersection of all ledger views}]{ledgercap}{$\Ledger_i^{\cap}$}\glsadd{ledgercap}
For a ledger $\Ledger$ that satisfies persistence at time $t$, we denote by $\LView{\Ledger}{\cup}{t}$ (resp.
$\LView{\Ledger}{\cap}{t}$) the sequence of transactions that are seen as
included in the ledger by \emph{at least one} (resp., \emph{all}) of the honest
parties. Finally, $\length(\Ledger)$ denotes the length of the ledger $\Ledger$,
i.e., the number of transactions it contains.
