\subsection{Blockchains and Ledgers}
A \emph{blockchain} (or a \emph{chain}) (denoted e.g. $\Chain$)  is a sequence
of blocks where each one is connected to the previous one by containing its
hash.

In blockchain protocols, each honest party $\party$ maintains a currently
adopted chain. We denote $\LView{\Chain}{\party}{t}$ the chain adopted by party
$\party$ at slot~$t$.

A \emph{ledger} (denoted in bold-face, e.g. $\Ledger$) is a mechanism for
maintaining a sequence of transactions, often stored in the form of a
blockchain.  In this paper, we slightly abuse the language by letting $\Ledger$
(without further qualifiers) interchangeably refer to the algorithms used to
maintain the sequence, and all the views of the participants of the state of
these algorithms when being executed.  For example, the (existing) ledger
Bitcoin consists of the set of all transactions that ever took place in the
Bitcoin network, %the Bitcoin protocol,
the current UTXO set, as well as the
local views of all the participants.

In contrast, we call a \emph{ledger state}
%$\LState$
a concrete sequence of transactions
$\tx_1,\tx_2, \ldots$ stored in the \emph{stable} part of a ledger $\Ledger$, typically as viewed by a
particular party.
Hence, in every blockchain-based ledger $\Ledger$, every fixed
chain $\chain$ defines a concrete ledger state % $\LState$
by applying  the interpretation rules given as a part of the description
of~$\Ledger$ (for example, the ledger state is obtained from the
blockchain by dropping the last $k$ blocks and serializing the transactions in
the remaining blocks).
We maintain the typographic convention that a ledger state (e.g. $\LState$) always
belongs to the bold-face ledger of the same name (e.g. $\Ledger$).
We denote by $\LView{\Ledger}{P}{t}$ the ledger state of a ledger $\Ledger$ as viewed by a party
$\party$ at the beginning of a time slot $t$,
%Finally, recall that a transaction is called \emph{stable} if it is contained in a block
%at least $k$ blocks away from the end of the blockchain. For a blockchain
%$\Chain$, we denote by $\stable{\Chain}$ the stable part of $\Chain$ (which
%implicitly depends on $k$). Moreover, for a  chain $\Ledger$ we
%denote by $\Ledger^t$ its form in some past time slot $t$.
and by $\check{\Ledger}^P[t]$ the complete state of the ledger (at time
$t$) including all
pending transactions that are not stable yet.
For two ledger states (or, more generally, any sequences), we denote by
$\preceq$ the prefix relation.

Recall the definitions of persistence and liveness of a robust public
transaction ledger given in the most recent version of~\cite{backbone}:

\begin{description}
  \item[Persistence.]
    For any two honest parties $\party_1,\party_2$ and two time slots $t_1\leq t_2$,
    it holds  $\LView{\Ledger}{\party_1}{t_1} \preceq
    \LView{\check{\Ledger}}{\party_2}{t_2}$.

  \item[Liveness.]
    If all honest parties in the system attempt to include a  transaction
    then, at any slot $t$ after $u$ slots (called the
    liveness parameter), any honest party $\party$, if queried,
    will report $\tx \in \LView{\Ledger}{\party}{t}$.
\end{description}

\glsxtrnewsymbol[description={union of all ledger views}]{ledgercup}{$\Ledger_i^{\cup}$}\glsadd{ledgercup}
\glsxtrnewsymbol[description={intersection of all ledger views}]{ledgercap}{$\Ledger_i^{\cap}$}\glsadd{ledgercap}
For a ledger $\Ledger$ that satisfies persistence at time $t$, we denote by $\LView{\Ledger}{\cup}{t}$ (resp.
$\LView{\Ledger}{\cap}{t}$) the sequence of transactions that are seen as
included in the ledger by \emph{at least one} (resp., \emph{all}) of the honest
parties. Finally, $\length(\Ledger)$ denotes the length of the ledger $\Ledger$,
i.e., the number of transactions it contains.
