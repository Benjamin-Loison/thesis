\section{Notation}
We use standard mathematical notation throughout this work. We define all the
non-standard or unusual notation in this section. We use standard cryptographic
notation, which is also introduced here for reference. The reader unfamiliar
with this notation can consult some reference book in the subject such
as~\cite{katz} for a more complete treatment.

Given a distribution $\mathcal{M}$, we denote by $m \gets \mathcal{M}$ the experiment by which the random variable $m$ is chosen according to the distribution $\mathcal{M}$. Given a finite set $M$, we use $\uniform(M)$ to denote the uniform distribution which assigns probability $1 / 2^{|M|}$ to each element $m \in M$. We will use $m \getsrandomly M$ to denote the experiment in which $m$ is sampled from $\uniform(M)$.

As a shorthand for probabilities and to avoid excessive subscripting, we will write the experimental set up (such as the sampling of random variables) within the $\Pr[\cdot]$ prior to the predicate of interest and separated by $;\,$. For example, $\Pr[x \gets \mathcal{D}_1, y \gets \mathcal{D}_2; x + y = 1]$ denotes the experiment of independently sampling two random variables, $x$ and $y$, from the distributions $\mathcal{D}_1$ and $\mathcal{D}_2$ respectively, summing their values, and observing whether their sum is equal to $1$.

\subsection{Asymptotic probabilistic security}
Following the extended Church--Turing definition, we consider the class of
problems in $\textsc{P}$ to be \emph{easy}, and we call \emph{hard} those which
are not easy~\cite{sipser}. We will talk about \emph{honest parties}, Turing
Machines~\cite{turing} which run our code, and the \emph{adversary}, which is an arbitrary
Turing Machine that can run any code. It is assumed that all honest parties and the adversary have polynomial available time. Both the honest parties and the adversary have access to true randomness and are thus probabilistic Turing Machines.
The shorthand \emph{PPT} is used to denote a probabilistic polynomial-time
Turing machine.

Our theorems are by computational reduction, in which we show that bad
events happen only with negligible probability in some security parameter. We
use $\lambda \in \mathbb{N}$ to denote this security parameter. This parameter allows all our
cryptographic primitives to be instantiated with the required level of security;
for example, it provides the number of bits in the output of our hash function.

\begin{definition}[Negligible]\index{Negligible}
  A function $f: \mathbb{N} \longrightarrow \mathbb{R}^+$ is
  \emph{negligible} if for all $k \in \mathbb{N}$ it holds that
  $f \in \mathcal{O}(\frac{1}{\lambda^k})$.
\end{definition}

We will use the notation $\negl$ to denote any negligible function.

\begin{definition}[Overwhelming]\index{Overwhelming}
  A function $f: \mathbb{N} \longrightarrow \mathbb{R}^+$ is
  \emph{overwhelming} if it can be written as $f(n) = 1 - \negl(n)$ for
  some negligible function $\negl$.
\end{definition}

We will define the \emph{security} of various cryptographic protocols by making use of challenger-adversary games in which the challenger is a known Turing Machine defined by us, but the adversary is an \emph{arbitrary} Turing Machine. Herein lies the beauty of cryptography as a science; it allows us to create protocols which we prove secure against \emph{any} adversary, even those we cannot conceive. A protocol will be considered \emph{secure} if no PPT adversary can win the respective game, except with negligible probability.

\subsection{Sequences}
We use $[n]$ to denote the set of natural numbers from $0$ up to and including
$n$. We also use $[\mathcal{M}]$ to denote the support of a distribution
$\mathcal{M}$; the distinction between the two notations will be clear from the
context.
We use $\epsilon$ to denote the empty sequence (or empty string). We write
one sequence next to another to denote string concatenation. Likewise, we
concatenate sequences to symbols by juxtaposition.

Our sequences are indexed starting at $0$. Given a sequence $\chain$, we use
$|\chain|$ to denote its length. We use Python notation to denote sequence
addressing. For $i \in [n - 1]$, we denote the $i^\text{th}$ element from the
beginning as $\chain[i]$. The first element of the sequence is thus $\chain[0]$.
For $i \in [n] \setminus \{0\}$, we denote the $i^\text{th}$ element from the
end as $\chain[-i]$. The last element of the sequence is thus $\chain[-1]$. We
call this the \emph{tip} of the sequence. Given $i \in \mathbb{Z},
j \in \mathbb{Z}$ with $i \leq j$, we denote $\chain[i{:}j]$ the subsequence from
$i$ (inclusive) to $j$ (exclusive), that is the sequence which contains exactly
the elements $\chain[i], \chain[i + 1], \dots, \chain[j - 1]$. If $i > j$, then
by convention we set $\chain[i{:}j] = \epsilon$. We allow this \emph{range}
notation to be used with negative indices as well, indicating ranges starting or
ending (or both) in indices considered from the end of the sequence, hence
allowing for $\chain[-i{:}j], \chain[i{:}-j]$, and $\chain[-i{:}-j]$. The left end of
a range can omitted if it is $i = 0$. The right end of a range can be omitted if
it is $j = |\chain|$. For example, $\chain[:-k]$ is the sequence $\chain$ with
the last $k$ elements excluded. In this example, if $|\chain| < k$, then
$\chain[:-k] = \epsilon$.

If an element $A$ is a member of a sequence $\chain$ we will use the notation $A
\in \chain$ to denote this, i.e. that there exist words $w, v$ such that $\chain
= wAv$. It will be clear from the context whether we are speaking about sequence
or sets. Given $A, Z \in \chain$ such that $A$ and $Z$ exist only once in
$\chain$, we denote by $\chain\{A{:}Z\}$ the subsequence of $\chain$ starting from
$A$ (inclusive) and ending in $Z$ (exclusive). If $A = \chain[0]$, it can be
omitted. Omitting $Z$ denotes the sequence starting with $A$ and containing all
subsequent elements until the end of the sequence.

% TODO: \chain\{A:Z\} notation (inclusive, exclusive). But how to include Z?
