\subsection{Merkle Trees}
Consider a set $S = \{s_1, s_2, \cdots, s_n\}$ of strings
$s_i \in \{0, 1\}^*$. At some initial time, this set is compressed into a
\emph{root} string $s$ which is short ($|s| = \lambda$). This compressed string
is produced honestly and is given to a party called the \emph{verifier}. Given
this short trusted root string, the verifier receives claims from untrusted
\emph{provers} which claim that a certain piece of data $s_i$ existed in $S$.
The verifier's job is to decide whether such claims are truthful or fraudulent.

This protocol is an \emph{authenticated data structure}. It consists of
four algorithms $\mathcal{G}$, $\textsc{compress}$, $\textsc{prove}$,
$\textsc{verify}$. At the beginning of the execution, $\mathcal{G}(1^\lambda)$
is invoked to initialize the protocol parameters. These parameters can be shared
among multiple invocations of the protocol. As these parameters are fixed by the
protocol in its concrete implementations, we will make them implicit from now
on. A set $S$ is compressed by invoking $\textsc{compress}(S)$
which produces the root $s$. When an honest prover wishes to prove that some
$s_i$ exists in $S$, they produce an \emph{inclusion proof} $\pi =
\textsc{prove}(S, s_i)$. When the verifier receives an element $s_i$ together
with a proof of inclusion $\pi$, they check its veracity by invoking
$\textsc{verify}(\pi, s_i, s)$, which returns $\true$ or $\false$.

Authenticated data structure protocols must be correct. This means that honest
executions should always work.

\begin{definition}[Correctness]
  Consider an authenticated data structure protocol
  $\Pi = (\textsc{compress}, \textsc{prove}, \textsc{verify})$.
  We say that $\Pi$ is
  \emph{correct} if

  \[\forall S:
    \forall s_i \in S:
    \textsc{verify}(\textsc{prove}(S, s_i), s_i, \textsc{compress}(S))\,.\]
\end{definition}

Such protocols are useful when $s$ and $\pi$ are short.

\begin{definition}[Succinctness]
  Consider an authenticated data structure protocol
  $\Pi = (\textsc{compress}, \textsc{prove}, \textsc{verify})$.
  We say that $\Pi$ is
  \emph{succinct} if
  for all $S$ it holds that

  \[|\textsc{compress}(S)| \in \bigO(polylog(|S|))
    \land
    \forall s_i \in S:
    |\textsc{prove}(S, s_i)| \in \bigO(polylog(|S|))
    \,.\]
\end{definition}

In the protocols we will explore, we will have $|s| = \lambda \in \bigO(1)$ and
$\pi \in \bigO(\log(|S|))$. Furthermore, $|S|$ will be polynomial in $\lambda$.

An authenticated data structure protocol is \emph{secure} if no adversary can
convince a verifier about the inclusion of an element which is not in the
set. This is made formal in the game illustrated in
Algorithm~\ref{alg.authenticated}.

\begin{figure}[t]
\begin{algorithm}[H]
    \caption{\label{alg.authenticated} The authenticated data structure challenger.}
    \begin{algorithmic}[1]
        \Function{\sf AUTH$_{\Pi,\mathcal{A}}$}{$\lambda$}
            \Let{(S, s_i, \pi)}{\mathcal{A}(1^\lambda)}
            \State\Return{$s_i \not\in S \land \textsc{verify}(\textsc{compress}(S), s_i, \pi)$}
        \EndFunction
        \vskip8pt
    \end{algorithmic}
\end{algorithm}
\end{figure}


\begin{definition}[Security]
  An authenticated data structure protocol $\Pi = (\textsc{compress},\allowbreak \textsc{prove},\allowbreak \textsc{verify})$ is \emph{secure} if for all PPT adversaries $\mathcal{A}$
  there is a negligible function $\negl$ such that

  \[
    \Pr[\textsf{AUTH}_{\Pi,\mathcal{A}}(\lambda)] < negl(\lambda)\,.
  \]
\end{definition}

A construction that solves this problem which is used extensively in blockchain
protocols is the \emph{Merkle Tree}~\cite{merkle}. This construction is
illustrated in Algorithm~\ref{alg.merkle}. It is parameterized by a collision
resistant hash funciton $H$. The construction presented works for $|S|$ equal to
a power of $2$ and assumes the domain of values in $S$ is disjoint from the
domain of $H$.

\begin{algorithm}[H]
    \caption{\label{alg.merkle} The Merkle Tree construction for $|S| = 2^k$ for
                                some $k$.}
    \begin{algorithmic}[1]
        \Function{\sf heapify$^H$}{$S$}
            \Let{Z[|S|{:}]}{\textsf{map}(H, S)}
            \For{$i \gets |S| - 1 \text{ down to } 1$}
                \Let{Z[i]}{H(Z[2i] \concat Z[2i + 1])}
            \EndFor
            \State\Return{$Z$}
        \EndFunction
        \Function{\sf compress$^H$}{$S$}
            \State\Return{$\textsf{heapify}^H(S)$[1]}
        \EndFunction
        \Function{\sf prove$^H$}{$S, i$}
            \Let{Z}{\textsf{heapify}^H(S)}
            \Let{i}{|S| + i}
            \Let{\pi}{[\,]}
            \While{$i > 1$}
                \Let{b}{i \mod 2}
                \Let{\pi}{\pi \concat (b, Z[i \xor 1])}
                \Let{i}{\lfloor i / 2\rfloor}
            \EndWhile
            \State\Return$\pi$
        \EndFunction
        \Function{\sf verify$^H$}{$\pi, s_i, s$}
            \Let{s_i}{H(s_i)}
            \For{$(b, h) \in \pi$}
                \If{$b$}
                    \Let{s_i}{H(s_i \concat h)}
                \Else
                    \Let{s_i}{H(h \concat s_i)}
                \EndIf
            \EndFor
            \State\Return{$s_i = s$}
        \EndFunction
        \vskip8pt
    \end{algorithmic}
\end{algorithm}


It treats $S$ as a sequence and organizes it into a complete binary tree $Z$
using the $\textsf{heapify}$ routine. The routine places the hashes of the
elements of $S$ on the leafs of $Z$ stored at locations $Z[|S|{:}]$. The value
of each internal node is the hash of the concatenation of the values of its
children. The $\textsf{compress}$ function returns the value of the root which
resides at $Z[1]$. To create a proof $\pi$, the $\textsf{prove}$ routine takes
an index of an element $i$ and finds its position in the binary tree, namely the
leaf stored at $Z[|S| + i]$. It then traverses the path from that leaf up to the
root, maintaining the index of the current node in the variable $i$. In every
iteration, it includes a bit indicating whether the current node is a left child
($b = 0$) or a right child ($b = 1$). For each node, it includes the value $Z[i
\xor 1]$ of the node's sibling. To verify a proof, the verifier successively
hashes the element whose inclusion is proven with the hashes $h$ of the siblings
provided in the proof $\pi$ on the correct side indicated by $b$. In the end, it
checks whether it has arrived at the trusted root $s$ and this determines the
result of the verification.

Correctness is achieved because the hash function is deterministic and the
\textsc{verify} applies hashes in the same manner as \textsc{heapify} does.
Succinctness follows because $|s| = \lambda$. Furthermore, the tree contains
$2|S| - 1$ elements and the height of the tree is $\Theta(\log(|S|))$, making
$|\pi| \in \Theta(\log(|S|))$. Lastly, security follows by a direct
computational reduction from the collision resistance of $H$.
