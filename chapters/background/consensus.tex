\subsection{The Consensus Problem}\index{Consensus}
As the peer-to-peer parties in a cryptocurrency system exchange money,
transactions are generated and disseminated on the network. Messages on the
network naturally do not arrive at every party in the same order (as is
formally captured by our network model of Section~\ref{sec:env}).
The parties need to determine whether a newly arriving transaction is valid by
comparing it against the currently adopted transaction sequence.
The order in which transactions are processed is important because it can affect
the validity of the transaction sequence. If every party organizes their
transactions according to the order they receive them from the network, they
will end up disagreeing about whether a transaction is valid. This gives rise to
the \emph{double spending problem}\index{Double Spending}.

Consider the following situation. In the UTXO model, Alice sends 1 coin to Eve via
a legitimate transaction $\tx$ which has only one output. This transaction is
broadcast to the network and every party adopts it as valid. Subsequently, Eve,
who is malicious,
generates two transactions $\tx_1, \tx_2$ each of which consumes the outpoint of
$\tx$ and generates a single new output of the same value, but to a different
receipient each: $\tx_1$ pays Bob, while $\tx_2$ pays back Charlie. Both
$\tx_1$ and $\tx_2$ are broadcast on the network at about the same time, and it
so happens that Bob receives $\tx_1$ before $\tx_2$, while Charlie receives
$\tx_2$ before $\tx_1$. If they adopt transactions in the order they receive
them, Bob will consider $\tx_1$ to be valid and will adopt it, while rejecting
$\tx_2$ as invalid. On the other hand, Charlie will do the opposite. Note that,
without the existence of the other, each of these transactions is individually
valid, as it is rightful and conserving.
When the time comes for Bob to pay Charlie, Charlie will not agree with Bob that
the money belongs to Bob. Without \emph{consensus} on which transaction occurred
first, the economic participants cannot agree who owns what, and the monetary
system breaks down.

Before we describe the solution that proof-of-work offers, let us first observe
that obvious solutions do not work. Consider the following protocol: Since only
a malicious party would create a signature on both transactions $\tx_1$ and
$\tx_2$, we could eliminate both transactions and mark them as invalid as soon
as the double spend has been detected. This is not a good strategy. In this
case, Eve can first pay Bob using $\tx_1$. After she receives the services of
Bob, at a much later time, she can publish $\tx_2$, rendering Bob's money
invalid. While Eve does not gain from this behavior, the trustworthiness of the
monetary system is subverted. Simple adjustments to this strategy do not work
either. Noting that $\tx_2$ would have to be broadcast much later, consider,
for example, a protocol which waits for some time $\Delta$ prior to establishing
whether a double spend has occurred or not. If $\tx_1$ appears at some time
$r_0$ without any double spending transaction appearing prior to time
$r_0 + \Delta$, the transaction $\tx_1$ can be accepted. Any double spending
transaction $\tx_2$ appearing after time $r_0 + \Delta$ is rendered invalid. On
the other hand, if $\tx_2$ appears within time $\Delta$, both $\tx_1$ and
$\tx_2$ are rendered invalid. This protocol can be subverted by Eve as follows.
She initially broadcasts $\tx_1$ at round $r_0$. When time $r_0 + \Delta$
approaches, she broadcasts $\tx_2$ and it so happens that Bob receives $\tx_2$
prior to time $r_0 + \Delta$, while Charlie receives $\tx_2$ after time
$r_0 + \Delta$. This would cause Bob to reject both $\tx_1$ and $\tx_2$, while
Charlie would only reject $\tx_2$. No simple protocol can withstand attacks that
cause such disagreements.

The problem then becomes an issue of coming to an agreement about the
\emph{order} in which transactions have occurred. This problem of agreement is
solved by the \emph{consensus layer}, the part of a decentralized system which
attempts to organize \emph{application layer} transactions into sequences that
belong to the application's validity language by ordering them into transaction
sequences.

As in later chapters we will speak about multiple decentralized systems
interoperating, it is helpful to note which \emph{ledger} $\Ledger$
the party in question maintains. In the simplest case, each ledger is associated
with a unique cryptocurrency, but we will later relax this. The $\Ledger$ is the
protocol which the honest parties execute to maintain their local views.

\begin{definition}[Ledger]\index{Ledger}
  A ledger protocol $\Ledger$ is an honest probabilistic polynomial-time
  algorithm for maintaining a transaction sequence belonging to a validity
  language $\ValLang$.
\end{definition}

To make the problem precise, we will consider the honest parties' beliefs on
\emph{which transactions have occurred and what their order is}. The transaction
sequence reported by an honest party at a particular round is known as the
\emph{ledger view} of the party. An honest party's protocol exposes two
methods related to transaction processing: A method to \emph{read} the current
transaction sequence and a method to \emph{write} a transaction to their ledger.

\begin{definition}[Ledger view]\index{Ledger view}
  The \emph{ledger view} $\LView{\Ledger}{\party}{r} \in \ValLang$ of an honest
  party $\party$ at round $r$ pertaining to ledger $\Ledger$ with validity
  language $\ValLang$ is the transaction sequence reported by the honest party
  when it is given the instruction to \emph{read} the current transaction
  sequence.
\end{definition}

\begin{definition}[Confirmation]\index{Confirmation}
  A transaction $\tx$ of ledger $\Ledger$ is called \emph{confirmed} by honest
  party $\party$ at round $r$ if $\tx \in \LView{\Ledger}{\party}{r}$. We say
  that a transaction is \emph{confirmed} at round $r$ if it is confirmed by all
  honest parties.
\end{definition}

A good ledger protocol will exhibit two virtues: Persistence and Liveness. On
one hand, persistence mandates that the parties eventually come to agreement.
On the other hand, liveness mandates that transactions that occur eventually
appear in everyone's view.

\begin{definition}[Persistence]\index{Persistence}
  A ledger protocol $\Ledger$ has \emph{persistence} with parameter $\lambda$
  (the \emph{persistence parameter}) if for
  any two honest parties $\party_1,\party_2$ and two rounds $r_1\leq r_2 +
  \lambda$, it holds $\LView{\Ledger}{\party_1}{r_1} \preceq
  \LView{\Ledger}{\party_2}{r_2}$.
\end{definition}

Note that we do not require that the ledger views of the honest parties are
equal, but that, after sufficient time, they are a prefix of each other. The
reason for this is that, while both parties will eventually agree on their
ledger view up to a point, each may include a transaction at a later time. Until
an honest party can be certain that a transaction is confirmed and is ready to
report it in a particular position within its ledger view as \emph{stable}, it
will keep the transaction as \emph{unstable} and not report it in its view. The
duration, in number of rounds, during which transactions can remain unstable is
known as the \emph{persistence parameter}.

\begin{definition}[Liveness]\index{Liveness}
  A ledger protocol $\Ledger$ has \emph{liveness} with parameter $u$ (the
  \emph{liveness parameter}) if the following holds.
  If \emph{all} honest parties in the system attempt to include a transaction
  $\tx$ (which validly extends their ledger views) for all rounds $r, r + 1, \cdots, r
  + u$ then, at any round $r' > r + u$, any queried honest party $\party$ will
  report $\tx \in \LView{\Ledger}{\party}{r'}$.
\end{definition}

Liveness ensures that transactions appear on the ledger views of honest parties.
The liveness parameter $u$ denotes the duration for which a party has to wait
until a transaction is confirmed. When we develop our protocols and we require
a transaction to appear in a ledger, this parameter will appear as waiting time
to ensure the transaction has taken its position to all honest parties' ledgers.

We note that it is trivial to design a ledger protocol which has either liveness
or persistence, but not both. To achieve persistence without liveness, the
honest parties always return $\epsilon$ when their ledger views are read. This
trivially satisfies persistence, as the empty sequence as a prefix of itself.
Liveness is not satisfied because transactions are never confirmed. To achieve
liveness without persistence, the honest parties include transactions in their
ledgers in the order they see them on the network. This ensures transactions
appear immediately, which achieves liveness, but the ledgers of different
parties will disagree about their contents and order, and so persistence is
lost. The challenge in creating a \emph{secure ledger} will be to achieve both
properties.

\begin{definition}[Secure ledger]
  A ledger $\Ledger$ is called \emph{secure} with parameters $\lambda$, $u$ if
  it has persistence with parameters $\lambda$ and liveness with parameter $u$.
\end{definition}

To solve the consensus problem and create a secure ledger protocol, the honest
parties must come to agreement and ensure that the adversary does not split
their views while including all transactions. In order to do that, we will allow
parties to \emph{vote} on a transaction order. We will then utilize the majority
of votes to arrive at a conclusion. This requires us to introduce an additional
assumption: That the majority of parties are honest.
